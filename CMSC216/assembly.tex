\documentclass[11pt]{article}
\usepackage{amsmath, amssymb, amsthm}

\usepackage{fancyhdr}

%Listings stuff
\usepackage{listings}
\usepackage{lstautogobble}
\usepackage{color}

\definecolor{gray}{rgb}{0.5,0.5,0.5}
\lstset{
basicstyle={\small\ttfamily},
tabsize=3,
numbers=left,
numbersep=5pt,
numberstyle=\tiny\color{gray},
stepnumber=2,
breaklines=true
}

%Format stuff
\pagestyle{fancy}
\headheight 35pt

%Header info
\chead{\Large \textbf{Assembly}}
\lhead{}
\rhead{}

\begin{document}
\section{Assembler}
	An assembler takes assembly language and turns it into machine code, numeric
	instructions read into the processor. Modern processors break down machine
	code into ``micorcode'' even further.

	It is possible to disassemble machine code back into assembly, but any
	variable names and variables are lost.

	\subparagraph{Makefile} It is possible to write a makefile rule to use yas,
	the assembler, to compile a .ys assembly file to .yo.

	\begin{lstlisting}[autogobble=true]
		all: assembler.yo
			yis $<
		%.yo: %.ys
			yas $<
	\end{lstlisting}

	The machine used in 216 is ``y86'' as a simplified successor to the x86
	Intel processor.

\section{Instructions}
	\subsection{Operands}
		\subparagraph{Flags} There are three bitflags set automatically by the system
		whenever an arithmetic operation is performed: \verb|OF|, \verb|SF|, and
		\verb|ZF|: whether an operation overflowed, returned a negative result, or
		was zero, respectively. All conditional instructions check these flags.

		\subparagraph{Program Counter} The PC points to the address of the next
		instruction to execute.

		\subparagraph{Registers} Registers are very fast memory that hold a single
		word and  can be read or written in one cycle. In y86, e[abcd]x, e[sd]i and e[sb]p. The p's are
		pointers to the stack (top/bottom), the i's are indices (source, destination), and the
		x's are general purpose, defined by convention.

		\subparagraph{Memory} Memory takes several cycles to fetch if not in cache, which is
		much faster.

		\subparagraph{Immediate} These are constants in the instruction itself, in
		the data segment. A pointer to a constant in the data segment would be an
		immediate.

	\subsection{Commands}
	
	\subparagraph{Movement} The \verb|movl| command moves data from one place to another. The first 2 characters specify the source and destination: \verb|i| means immediate, \verb|r| means register, and \verb|m| means memory.
	\begin{center}
	\begin{tabular}{l|l}
		Instruction	&	Effect\\\hline
		
		\verb|irmovl V, R|	& Reg[R]$\leftarrow$ V\\
		\verb|rrmovl rA, rB|	&	Reg[rB]$\leftarrow$ Reg[rA]\\
		\verb|rmmovl rA,D(rB)|	&	Mem[Reg[rB]+D] $\leftarrow$ Reg[rA]\\
		\verb|mrmovl D(rA),rB|	&	Reg[rB]$\leftarrow$Mem[Reg[rA] + D]\\
	\end{tabular}
	\end{center}
	
	\subparagraph{Integer Operations} There are 4 integer operations,
	\verb|addl|, \verb|subl|, \verb|andl|, and \verb|xorl| that only operation
	on register data. The added instructions to the set are \verb|multl|,
	\verb|divl|, and \verb|modl|. They set the three condition codes \verb|ZF|, \verb|SF|,
	and \verb|0F| (zero, sign, overflow) and are in the generic form \verb|OPl rA, rB|. The first register is the source register and the second is the
	destination, and the rule for applying the operation is ``dest <- dest op
	source''.
	
	\subparagraph{Jump} There are seven jump instructions: \verb|jmp|,
	\verb|jle|, \verb|jl|, \verb|je|, \verb|jne|, \verb|jge|, and \verb|jg|. The
	conditions are fairly self-explanatory (compares to 0), and the generic command is \verb|jXX Dest|.
	
	\subparagraph{Conditional Move}6 conditional move instructions: \verb|cmovle|, \verb|cmovl|, \verb|cmove|, \verb|cmovge|, and \verb|cmovg|. They are conditional forms of \verb|rrmovl| and are called like \verb|cmovXX rA, rB|.
	
	\subparagraph{Call and Return} the \verb|call| instruction pushes the return
	address on the stack and advances the PC to the destination: \verb|call Dest|. The \verb|ret| instruction returns from such a call by popping the
	first value it finds off the stack and putting in the PC. This means that
	when doing functions, the stack has to be carefully managed to make sure the
	PC ends up in the correct spot.
	
	\subparagraph{Stack} \verb|pushl rA| and \verb|popl rA| manipulate the stack with register values. The stack holds $2^{12}$ bytes of memory and starts at the top by default at \verb|0x1000|, growing toward lower addresses. The location of the `top' of the stack is kept in the stack pointer, the register designated by \verb|%esp|. 
	
	When a register is pushed, \verb|%esp| is decremented by 4 and the old location it pointed to is overwritten with the register data. When the register is popped, the value on top of the stack is written to the register, and \verb|%esp| increments by 4. 
	
	REMEMBER TO INITIALIZE WITH \verb|irmovl $0x1000, %esp| AND \verb|irmovl $0x1000, %ebp| BEFORE ANY STACK OPERATIONS.
	
	\subparagraph{Miscellaneous} \verb|halt| stops instruction execution, setting the status code to \verb|HLT|. \verb|nop| is no operation, and the processor sleds through it.
	
	\subparagraph{Input/Output} \verb|rdch|, \verb|rdint|, \verb|wrch|, and \verb|wrint| are added commands to Y86 for input/output from standard input, and read/write an integer/character from/to standard input from/to the specified register.
	
\section{Calling Functions}
	When calling functions with arguments, use the C calling style, pushing
	arguments right to left (``backwards'').

	\begin{lstlisting}[autogobble=true]
			push all args
			
			call	//pushes PC
			push ebp	//The caller needs this back
			rrmovl %esp to %ebp
				access arguments at (8+4*i)(%ebp)
			pop ebp
			ret
	\end{lstlisting}

	After the return, you can either pop all the arguments into a junk register, or just add the argument length to \verb|%esp|, which effectively pops them into nothingness. Convention is typically that \verb|%eax| contains the returned value as well.

%	\begin{center}
%	\begin{tikzpicture}
%		[scale=3,line cap=round,
%		%Styles
%		axes/.style=,
%		important line/.style={very thick},
%		information text/.style={rounded corners,fill=red!10,inner sep=1ex},
%		dot/.style={circle,inner sep=1pt,fill,label={#1},name=#1}			
%		]
%		
%		%Colors
%		\colorlet{anglecolor}{green!50!black}	%angle arcs/lines
%		
%		%The graphic
%	\end{tikzpicture}
%	\end{center}

%	\begin{figure}[htb]
%		\centering
%		\includegraphics[width=0.8\textwidth]{filename.eps}
%		\caption{Caption.}
%		\label{fig:figure}
%	\end{figure}

%		\def\enotesize{\normalsize}
%		\theendnotes
\end{document}
