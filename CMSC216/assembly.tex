\documentclass[11pt]{article}
\usepackage{amsmath, amssymb, amsthm}
\usepackage[retainorgcmds]{IEEEtrantools}

\usepackage[pdftex]{graphicx}
\usepackage{tikz}
\usetikzlibrary{intersections}

\usepackage{marginnote}
\usepackage{endnotes}

\usepackage{fancyhdr}

%Listings stuff
\usepackage{listings}
\usepackage{lstautogobble}
\usepackage{color}

\definecolor{gray}{rgb}{0.5,0.5,0.5}
\lstset{
basicstyle={\small\ttfamily},
tabsize=3,
numbers=left,
numbersep=5pt,
numberstyle=\tiny\color{gray},
stepnumber=2,
breaklines=true
}

%Properly formatted differential 'd'
\newcommand{\ud}{\, \mathrm{d}}

%Format stuff
\pagestyle{fancy}
\headheight 35pt

%Header info
\chead{\Large \textbf{Assembly}}
\lhead{}
\rhead{}

\begin{document}
\section{Assembler}
	An assembler takes assembly language and turns it into machine code, numeric
	instructions read into the processor. Modern processors break down machine
	code into ``micorcode'' even further.

	It is possible to disassemble machine code back into assembly, but any
	variable names and variables are lost.

	\subparagraph{Makefile} It is possible to write a makefile rule to use yas,
	the assembler, to compile a .ys assembly file to .yo.

	\begin{lstlisting}[autogobble=true]
		all: assembler.yo
			yis $<
		%.yo: %.ys
			yas $<
	\end{lstlisting}

	The machine used in 216 is ``y86'' as a simplified successor to the x86
	Intel processor.

\section{Instructions}
	\subsection{Operands}

		\subparagraph{Registers} Registers are very fast memory that hold a single
		word and  can be read or written in one cycle. In y86, e[abcd]x, e[sd]i and e[sb]p. The p's are
		pointers to the stack, the i's are indexes (source, destination), and the
		x's are general purpose, defined by convention.

		\subparagraph{Memory} take several cycles to fetch if not in cache, which is
		much faster.

		\subparagraph{Immediate} These are constants in the instruction itself, in
		the data segment. A pointer to a constant in the data segment would be an
		immediate.

	\subsection{Commands}


	irmovl V,R
	Reg[R] ← V
	Immediate-to-register move
	irmovl $55,%edx

	\begin{center}
	\begin{tabular}{cclc}
		Instruction	&	Effect	&	Description	&	Example\\
		irmovl V, R	&	Reg[R]$\leftarrow$ V	&	Immediate-to-register move	&
		irmovl \$55,%edx\\

		rrmovl rA, rB	&	Reg[rB]$\leftarrow$ Reg[rA]	&	Register-to-register
				 move	&	rrmovl %edx,%ebx\\

		rmmovl rA,D(rB)	&	Mem[Reg[rB]+D] ← Reg[rA] 	&	Register-to-memory move
									&	rmmovl	&	%ebx,4(%eax)\\

		mrmovl D(rA),rB	&	Reg[rB]$\leftarrow$Mem[Reg[rA] + D]	&
Memory-to-register move	&	mrmovl 0(%eax),%ecx
	\end{tabular}
	\end{center}

%	\begin{center}
%	\begin{tikzpicture}
%		[scale=3,line cap=round,
%		%Styles
%		axes/.style=,
%		important line/.style={very thick},
%		information text/.style={rounded corners,fill=red!10,inner sep=1ex},
%		dot/.style={circle,inner sep=1pt,fill,label={#1},name=#1}			
%		]
%		
%		%Colors
%		\colorlet{anglecolor}{green!50!black}	%angle arcs/lines
%		
%		%The graphic
%	\end{tikzpicture}
%	\end{center}

%	\begin{figure}[htb]
%		\centering
%		\includegraphics[width=0.8\textwidth]{filename.eps}
%		\caption{Caption.}
%		\label{fig:figure}
%	\end{figure}

%		\def\enotesize{\normalsize}
%		\theendnotes
\end{document}
