\documentclass[11pt]{article}
\usepackage{amsmath, amssymb, amsthm}
\usepackage[retainorgcmds]{IEEEtrantools}

\usepackage[pdftex]{graphicx}
\usepackage{tikz}
\usetikzlibrary{intersections}

\usepackage{marginnote}
\usepackage{endnotes}

\usepackage{fancyhdr}

%Listings stuff
\usepackage{listings}
\usepackage{lstautogobble}
\usepackage{color}

\definecolor{gray}{rgb}{0.5,0.5,0.5}
\lstset{
basicstyle={\small\ttfamily},
tabsize=3,
numbers=left,
numbersep=5pt,
numberstyle=\tiny\color{gray},
stepnumber=2,
breaklines=true
}

%Properly formatted differential 'd'
\newcommand{\ud}{\, \mathrm{d}}

%Format stuff
\pagestyle{fancy}
\headheight 35pt

%Header info
\chead{\Large \textbf{Introduction to C}}
\lhead{}
\rhead{}

\begin{document}
\section{Declarations and the Preprocessor}
	\verb|#include| provides ability to include declarations from other files, usually header files which only include function prototypes. The actual functions are kept in a .c file.
	Two ways to use \verb|#include|: 
		\begin{lstlisting}[autogobble=true]
			#include<stdio.h>
			#include"swap.h"
		\end{lstlisting}
	The first method looks for standard library headers, and the second one looks for user-defined headers beginning in the working directory.
	
\section{Compiling}
	Use \verb|gcc| to compile, options include:
	\begin{description}
		\item[-g:] enable debugging
		\item[-Wall:] warn about common potential problems
		\item[-o filename:] place output in filename
		\item[-c:] only compile to object file, don't link
	\end{description}
	
	The preprocessor makes sure the program sees declarations they need to know (\verb|#define| and \verb|#include| for example). No semicolon is needed.
	
	In \textbf{translation}, the compiler makes sure individual parts are consistent within themselves and creates an object file.
	
	During \textbf{linking}, the compiler joins compiled object files together, including matching function call to callee and matching global variables.
	
\section{Data Types}
	\begin{itemize}
		\parskip=-1pt
		\item \verb|char, short, int, long int, long long int|
		\item \verb|float, double|
		\item \verb|*, []| - pointer and array
		\item \verb|union {}| - variables sharing space
		\item \verb|struct {}| - concatenation of variables
		\item \verb|function()| - address of a beginning of a function
		\item \verb|enum {}| - named integer values
	\end{itemize}
	
	\subsection{Data Sizes}
		\begin{center}
		\begin{tabular}{ccc}
			Type	&	Minimum Size	&	Size on linux.grace\\\hline
			char	&	1				&	1\\
			short	&	2				&	2\\
			int		&	2				&	4\\
			long	&	4				&	8\\
			float	&	4				&	4\\
			double	&	8				&	8\\
			long double	&	10			&	12
		\end{tabular}
		\end{center}
		
	\subsection{Numeric Literals}
	Suffixes allow to specify number types: \verb|10f| is float and \verb|10L| is long. Prefixes specify bases: \verb|0| for octal, and \verb|0x| for hex.
	
%	\begin{center}
%	\begin{tikzpicture}
%		[scale=3,line cap=round,
%		%Styles
%		axes/.style=,
%		important line/.style={very thick},
%		information text/.style={rounded corners,fill=red!10,inner sep=1ex},
%		dot/.style={circle,inner sep=1pt,fill,label={#1},name=#1}			
%		]
%		
%		%Colors
%		\colorlet{anglecolor}{green!50!black}	%angle arcs/lines
%		
%		%The graphic
%	\end{tikzpicture}
%	\end{center}

%	\begin{figure}[htb]
%		\centering
%		\includegraphics[width=0.8\textwidth]{filename.eps}
%		\caption{Caption.}
%		\label{fig:figure}
%	\end{figure}

%		\def\enotesize{\normalsize}
%		\theendnotes
\end{document}