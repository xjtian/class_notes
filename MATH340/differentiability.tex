\documentclass[11pt]{article}
\usepackage{amsmath, amssymb, amsthm}
\usepackage[retainorgcmds]{IEEEtrantools}

\usepackage[pdftex]{graphicx}
\usepackage{tikz}
\usetikzlibrary{intersections}

\usepackage{marginnote}
\usepackage{endnotes}

\usepackage{fancyhdr}

%Listings stuff
\usepackage{listings}
\usepackage{lstautogobble}
\usepackage{color}

\definecolor{gray}{rgb}{0.5,0.5,0.5}
\lstset{
basicstyle={\small\ttfamily},
tabsize=3,
numbers=left,
numbersep=5pt,
numberstyle=\tiny\color{gray},
stepnumber=2,
breaklines=true
}

%Properly formatted differential 'd'
\newcommand{\ud}{\, \mathrm{d}}

%Format stuff
\pagestyle{fancy}
\headheight 35pt

%Header info
\chead{\Large \textbf{Differentiability}}
\lhead{}
\rhead{}

\begin{document}
\section{Limits and Continuity}
	The limit of a function $\lim_{x\rightarrow c} f(x) = l$ means that the function $f(x)$ is arbitrarily close to $l$ as $x$ approaches $c$. Given a function $\mathbb{R}^n \xrightarrow{f} \mathbb{R}^m$, the point $\vec{y}_0 \in \mathbb{R}^m$, $\vec{y}_0$ is the limit if $f$ at $\vec{x}_0$ if:
	
	\[\forall \epsilon > 0, \exists \delta > 0: (|f(\vec{x}) - \vec{y}_0 | < \epsilon \wedge 0 < |\vec{x} - \vec{x}_0| < \delta)\]
	
	Informally speaking, the definition says that $f(\vec{x})$ is arbitrarily close to $\vec{y}_0$ when $\vec{x}$ is sufficiently close to $\vec{x}_0$ and $\vec{x}\neq \vec{x}_0$.
	
	\subparagraph{Neighborhoods} A $\delta$-ball is defined as the set of all points in $\mathbb{R}^n$ that satisfy the inequality $|\vec{x}-\vec{x}_0| < \delta$. $\vec{x}$ is a \textbf{limit point} of a set of points $S$ if $\forall \delta > 0, \exists \vec{y} \in S \mid 0 < |\vec{x} - \vec{y}| < \delta$, or that there are points in $S$ other than $\vec{x}$ that are contained in an arbitrary $\delta$-ball, or \textbf{neighborhood} with $\vec{x}$ at the center.
	
	\subsection{Continuity}
		A function $f$ is continuous at $\vec{x}_0$ if $\vec{x}_0$ is in the domain of $f$ and \[lim_{\vec{x} \rightarrow \vec{x}_0} f(\vec{x}) = f(\vec{x}_0)\]
		At a nonlimit, or \textbf{isoalted} point in the domain of $f$, the limit is meaningless, so the definition of continuity is extended by defining $f$ to be continuous on a subset of its domain $S$ if it's continuous at every point in $S$.
		\begin{itemize}
			\item The functions $\mathbb{R}^n \xrightarrow{P_k} \mathbb{R}$, where $P_k (x_1,\ldots , x_n) = x_k$, are continuous. $P_k$ is called the $k$th coordinate projection.
			\item The functions $S(x, y) = x + y$ and $M(x, y) = xy$ are continuous.
			\item If $f(\vec{x})$ and $g(\vec{x})$ are continuous, then $g(f(\vec{x}))$ is continuous wherever it is defined.
		\end{itemize}
		
\section{Real-Valued Functions}
	\marginnote{Differentiability of real-valued function?}A function $\mathbb{R}^n \xrightarrow{f} \mathbb{R}$ is differentiable at $\vec{x}_0$ if $\vec{x}_0$ is in the domain of $f$ and there exists a vector $\vec{a}$ such that
	\begin{equation}
		\lim_{\vec{x} \rightarrow \vec{x}_0} \frac{f(\vec{x}) - f(\vec{x}_0) - \vec{a}\cdot (\vec{x} - \vec{x}_0)}{|\vec{x} - \vec{x}_0|} = 0
	\end{equation}
	\marginnote{Gradient?}$\vec{\nabla}f(\vec{x}_0)$ is the only vector that satisfies the second condition, called the \textbf{gradient vector}, where the $k$th component of $\vec{\nabla}f$ is the $k$th partial of $f$. defined as If any component of $\vec{\nabla}f$ is not defined at $\vec{x}_0$, then $f$ is not differentiable at that point.
	
	If all of the partials of $f$ are continuous on some open subset of the domain, then $f$ is differentiable at every point in that subset as well. $f$ is \textbf{continuously differentiable} on an open set $D$ if all entries in $\vec{\nabla}f$ are continuous on $D$. Thus, \textit{differentiability of $f$ implies continuity.}
	
	\subparagraph{Tangent Approximations} The \marginnote{Tangent approximation of real-valued function?}tangent approximation of a function $\mathbb{R} \rightarrow \mathbb{R}$ is $T(x) = f(x_0) + f'(x_0)(x - x_0)$. Generalizing to multidimensional real-value functions, it takes the form
	\begin{equation}
		T(\vec{x}) = f(\vec{x}_0) + \nabla f(\vec{x}_0) \cdot (\vec{x}-\vec{x}_0)
	\end{equation}
	
\section{Directional Derivatives}
	\marginnote{Directional derivative?}The directional derivative of a function measures the rate of change in an arbitrary direction $\vec{v}$.
	\begin{equation}
		\frac{\partial f}{\partial \vec{v}} (\vec{x}) = \lim_{t \rightarrow 0} \frac{f(\vec{x} + t\vec{v}) - f(\vec{x})}{t}
	\end{equation}
	\marginnote{Rel. b/w directional and gradient?}If $f$ is differentiable at $\vec{x}$ and $\vec{v} \neq \vec{0}$,
	\begin{equation}
		\frac{\partial f}{\partial v} (\vec{x}) = \nabla f(\vec{x}) \cdot \vec{v}
	\end{equation}
	It's best to choose $|\vec{v}| = 1$ to have a standardized rate of change, in which case the derivative is notated as $\partial f / \partial \vec{u}$. 
	
	\subparagraph{Mean Value Theorem} In \marginnote{MVT for real-valued functions?}one variable, $\exists x_0 \in (x, y) \mid f(x) - f(y) = f'(x_0)(x - y)$. In multiple variables, given $\mathbb{R}^n \xrightarrow{f} \mathbb{R}$ is differentiable on an open set containing the line segment $S$ joining $\vec{x}$ and $\vec{y}$,
	\begin{equation}
		\exists \vec{x_0} \in S \mid f(\vec{y}) - f(\vec{x}) = \nabla f(\vec{x}_0) \cdot (\vec{y} - \vec{x})
	\end{equation}
	The domain of $\mathbb{R}^n \xrightarrow{f} \mathbb{R}$ is \textbf{polygonally connected} if any given pair of points in it can be joined by a polygonal path. If $f$ is differentiable on a polygonally connected domain and $\forall \vec{x} \in D, \nabla f(\vec{x}) = 0$, $f$ is constant.
	
\section{Vector-Valued Functions}
	\marginnote{Differentiability of $\mathbb{R}^n \xrightarrow{f} \mathbb{R}^m$?}$\mathbb{R}^n \xrightarrow{f} \mathbb{R}^m$ is differentiable on its domain if each of its $m$ real-valued functions $f_1, f_2, \ldots f_m$ is differentiable. If all $n$ first-order partials of each of the $m$ coordinate functions are continuous, then $f$ is continuously differentiable. 
	
	\subparagraph{Derivative Matrix} The \marginnote{Derivative matrix?}derivative matrix of a differentiable function at $\vec{x}$ is defined by 
		\begin{equation}
			f'(\vec{x}) = 
			\begin{pmatrix}
				\dfrac{\partial f_1}{\partial x_1} & \dfrac{\partial f_1}{\partial x_2} & \ldots & \dfrac{\partial f_1}{\partial x_n}\\
				\dfrac{\partial f_2}{\partial x_1} & \dfrac{\partial f_2}{\partial x_2} & \ldots & \dfrac{\partial f_2}{\partial x_n}\\
				\vdots & \vdots & \ddots & \vdots \\
				\dfrac{\partial f_m}{\partial x_1} & \dfrac{\partial f_m}{\partial x_2} & \ldots & \dfrac{\partial f_m}{\partial x_n}
			\end{pmatrix}
		\end{equation}
		This matrix can be regarded as having tangent vectors for columns and gradient vectors for rows.
		
	\subparagraph{Tangent Approximations} The \marginnote{Tangent approx. of $\mathbb{R}^n \xrightarrow{f} \mathbb{R}^m$?}first-degree Taylor approximation to $f: \mathbb{R}^n \rightarrow \mathbb{R}^m$ at $\vec{x}_0$ is
		\begin{equation}
			T(\vec{x}) = f(\vec{x}_0) + f'(\vec{x}_0)(\vec{x}-\vec{x}_0)
		\end{equation}
		Where the vector $\vec{x}-\vec{x}_0$ is interpreted as an $n$-by-$1$ column vector. 
		
		Related to tangent approximations, if $\mathbb{R}^n \xrightarrow{f} \mathbb{R}^m$ is differentiable at $\vec{x}_0$, then 
		\begin{equation}
			\lim_{\vec{x} - \vec{x}_0} \frac{f(\vec{x}) - f(\vec{x}_0) - f'(\vec{x}_0)(\vec{x} - \vec{x}_0)}{|\vec{x} - \vec{x}_0|} = 0
		\end{equation}
		and $f'(\vec{x}_0)$ is the unique matrix that satisfies this equation. 
		
		From this theorem follows the corollary that if $A$ is a constant $m$-by-$n$ matrix, then $\mathbb{R}^n \xrightarrow{f} \mathbb{R}^m$ defined by $f(\vec{x}) = A\vec{x}$ has $A$ for its derivative matrix ($f'(\vec{x}) = A$).
		
\section{Newton's Method}
	\marginnote{Newton's method?}Newton's method is used to approximate the solution of an equation $f(\vec{x}) = 0$, where $\mathbb{R}^n \xrightarrow{f} \mathbb{R}^m$ is a nonlinear function. This treats $f$ as an infinite series and gives an equation to compute the next term.
	\begin{equation}
		\vec{x}_{k+1} = \vec{x}_k - [f'(\vec{x}_k)]^{-1} f(\vec{x}_k)
	\end{equation}
	
	Using Newton's method in $\mathbb{R}^n$ for large $n$ can be very time-consuming when inverting the matrix, so the modified Newton method is as follows.
	\begin{equation}
		\vec{x}_{k+1} = \vec{x}_k - [f'(\vec{x}_0)]^{-1} f(\vec{x}_k)
	\end{equation}
	
	\marginnote{Altered Newton's method?}To use Newton's method, simply apply the method until the sequence converges, at which point the resulting vector will be an approximation of the answer. The first method will converge faster, although is computationally expensive, and the second method will take more terms to converge.
	
\section*{Important Concepts}
	\begin{itemize}
		\item Gradient vector is $\vec{a}$ such that \[\lim_{\vec{x} \rightarrow \vec{x}_0} \frac{f(\vec{x}) - f(\vec{x}_0) - \vec{a}\cdot (\vec{x} - \vec{x}_0)}{|\vec{x} - \vec{x}_0|} = 0\]
		\item If all partials of a function are continuous, so is the function.
		\item The tangent approximation for a real-valued function is \[T(\vec{x}) = f(\vec{x}_0) + \nabla f(\vec{x}_0) \cdot (\vec{x}-\vec{x}_0)\]
		\item Directional derivative is derivative in an arbitrary (usually length 1) direction:
			\[\frac{\partial f}{\partial \vec{v}} (\vec{x}) = \lim_{t \rightarrow 0} \frac{f(\vec{x} + t\vec{v}) - f(\vec{x})}{t}\]
		\item Gradient and directional derivative are related as follows: \[\frac{\partial f}{\partial v} (\vec{x}) = \nabla f(\vec{x}) \cdot \vec{v}\]
		\item The derivative matrix of a vector-valued function has gradients for rows and tangents for columns.
		\item For Taylor approximations of vector-valued functions:
			\[T(\vec{x}) = f(\vec{x}_0) + f'(\vec{x}_0)(\vec{x}-\vec{x}_0)\]
		\item $f(\vec{x}) = A\vec{x}$ has $A$ for its derivative matrix.
		\item Apply Newton's method until convergence:
			\[\vec{x}_{k+1} = \vec{x}_k - [f'(\vec{x}_k)]^{-1} f(\vec{x}_k)\]
		\item For large vector spaces, use the alternate form:
			\[\vec{x}_{k+1} = \vec{x}_k - [f'(\vec{x}_0)]^{-1} f(\vec{x}_k)\]
	\end{itemize}
%	\begin{center}
%	\begin{tikzpicture}
%		[scale=3,line cap=round,
%		%Styles
%		axes/.style=,
%		important line/.style={very thick},
%		information text/.style={rounded corners,fill=red!10,inner sep=1ex},
%		dot/.style={circle,inner sep=1pt,fill,label={#1},name=#1}			
%		]
%		
%		%Colors
%		\colorlet{anglecolor}{green!50!black}	%angle arcs/lines
%		
%		%The graphic
%	\end{tikzpicture}
%	\end{center}

%	\begin{figure}[htb]
%		\centering
%		\includegraphics[width=0.8\textwidth]{filename.eps}
%		\caption{Caption.}
%		\label{fig:figure}
%	\end{figure}

%		\def\enotesize{\normalsize}
%		\theendnotes
\end{document}