\documentclass[11pt]{article}
\usepackage{amsmath, amssymb, amsthm}
\usepackage[retainorgcmds]{IEEEtrantools}

\usepackage{fancyhdr}

%Format stuff
\pagestyle{fancy}
\headheight 35pt

%Header info
\chead{\Large \textbf{Multiple Integration}}
\lhead{}
\rhead{}

\begin{document}
\section{Iterated Integrals}
	When integrating a multiple-variable function once, integrate with respect to the differential while keeping all other variables constant (reverse partial derivative). 
	\begin{equation}
		A = \int_c^d G(x)dx = \int_c^d\left[ \int_a^b f(x, y)dy \right]dx
	\end{equation}
	
	\subparagraph{Nonrectangular Regions} Define the boundary curves in the limits of integration (works perfectly fine for iterated integrals) then evaluate in simplest order. 
	\[\int_0^1 dx \int_0^{1-x^2} dy = \frac{2}{3}\]
	
\section{Multiple Integrals}
	Consider a closed \textbf{coordinate rectangle} in $\mathbb{R}^n$: $a_i \leq x_i \leq b_i$. The volume or \textbf{content} of the coordinate rectangle is
	\begin{equation}
		V(R) = (b_1 - a_1)(b_2 - a_2)\ldots (b_n - a_n)
	\end{equation}
	
	Given a subset $B$ of $\mathbb{R}^n$ that is \textbf{bounded} such that $\forall \vec{x}\in B, |\vec{x}| < k$ and a grid composed of the union of a finite set of $(n-1)$-dimensional planes covering $B$, the Reimann sum for $f$ over $B$ is
	\begin{equation}
		\sum_{i=1}^r f_B \cdot (\vec{x}_i)V(R_i)
	\end{equation}
	provided that $f$ is 0 outside of $B$ (simple piecewise modification) and $R$ is not degenerate ($V(R) \neq 0$). The sum depends on the grid chosen to cover $B$, but no matter how a grid is chosen, if the limit exists and is always the same number:
	\begin{equation}
		\lim_{m(G) \rightarrow 0} \sum_{i=1}^r f_B (\vec{x}_i) \cdot V(R_i) = \int_B fdV
	\end{equation}
	
	\subsection{Existence}
		If $\mathbb{R}^n \xrightarrow{f} \mathbb{R}$ is defined and bounded on a bounded set $B$ such that the boundary of $B$ has zero content and $f$ is continuous, then $f$ is Riemann integrable over $B$. Given that $\int_B fdV$ exits and its iterated integrals do as well for some orders of integration, all the integrals are equal. 
		
		Basically, if some order of iteration for integrating over a region exists, then then it is equal to the multiple integral. The converse of this statement is also true.
		
\section{Integration Theorems}
	\subparagraph{Linearity} If $f$ and $g$ are integrable over $B$ and $a$ and $b$ are any two real numbers, then $af + bg$ is integrable over $B$ and
		\begin{equation}
			\int_B (af + bg)dV = a\int_B fdV + b\int_B gdV
		\end{equation}
	\subparagraph{Positivity} If $f$ is nonnegative and integrable over $B$, then
		\begin{equation}
			\int_B fdV \geq 0
		\end{equation}
	\subparagraph{Leibniz Rule} If $(\partial g / \partial y)(x, y)$ is continuous for $a\leq x\leq b$ and $c\leq y\leq d$, then
		\begin{equation}
			\frac{d}{dy}\int_a^b g(x,y)dx = \int_a^b \frac{\partial g}{\partial y}(x, y) dx
		\end{equation}
	\subparagraph{Theorem} If $f \leq g$, then
		\begin{IEEEeqnarray}{rCl}
			\int_B fdV & \leq & \int_B gdV\\
			\left| \int_B fdV \right| & \leq & \int_B |f|dV
		\end{IEEEeqnarray}
		
\section{Change of Variable}
	\subparagraph{Polar Coordinates} Transform the coordinates into polar form if the area of integration is a circle.
		\begin{equation}
			\int_D f(x, y)dA = \int_{\theta_0}^{\theta_1} d\theta \int_{r_0}^{r_1} f(r\cos \theta, r \sin \theta)r\ dr
		\end{equation}
		
	\subparagraph{Spherical Coordinates} $\bar{f} = f(r\sin\phi\cos\theta, r\sin\phi\sin\theta, r\cos\phi)$
		\begin{equation}
			\int_B f(x, y, z)dV = \int_{\theta_0}^{\theta_1} d\theta \int_{\phi_0}^{\phi_1} d\phi \int_{r_0}^{r_1} \bar{f}(r, \phi, \theta)r^2\sin\phi dr
		\end{equation}
		
	\subparagraph{Cylindrical Coordinates}
		\begin{equation}
			\int_B f(x, y, z)dV = \int_{z_0}^{z_1} dz \int_{\theta_0}^{\theta_1} d\theta \int_{r_0}^{r_1} f(r\cos \theta, r\sin \theta, z)r\ dr
		\end{equation}
		
		\subsection{Jacobi's Theorem}
			Let $T$ be a continuously differentiable coordinate transformation and the set that it maps to have a boundary consisting of finitely many smooth sets. Suppose that $R$ and its boundary are contained in the domain of $T$ and that
			\begin{itemize}
				\item $T$ is one-to-one on the interior of $R$.
				\item $det T'$, the Jacobian determinant of $T$, is not zero on the interior of $R$.
			\end{itemize}
			If $f$ is bounded and continuous on the image of $R$ under $T$, denoted by $T(R)$, then
			\begin{equation}
				\int_{T(R)} f(\vec{x})dV_{\vec{x}} = \int_R f(T(\vec{u}))\cdot|\text{det}T'(\vec{u})|dV_{\vec{u}}
			\end{equation}
			
\section{Centroids and Moments}
	The \textbf{center of mass} of a system of masses in space is given by the following.
	\begin{equation}
		\bar{\vec{x}} = \frac{1}{M}\sum_{k=1}^N m_k\vec{x}_k, \qquad \text{where} \qquad M=\sum_{k=1}^N m_k
	\end{equation}
	The \textbf{moment} $M_P$ of the mass system about a plane $P$, defined by the equation $\vec{n} \cdot (\vec{x} - \vec{x}_0) = 0$ normalized so $|\vec{n}| = 1$.
	\begin{equation}
		M_P = \sum_{k=1}^N m_k\vec{n} \cdot (\vec{x}_k - \vec{x}_0)
	\end{equation}
	Note that if $P$ contains the center of mass of the system, then the moment is 0. If the \textbf{total mass}, given by the following expression involving $\mu$, the density distribution,
	\begin{equation}
		M(b) = \int_B \mu(\vec{x})dV
	\end{equation}
	is positive, then the center of mass
	\begin{equation}
		\bar{\vec{x}} = \frac{1}{M(B)} \int_B \mu(\vec{x})\vec{x}dV
	\end{equation}
	The center of mass is called a  \textbf{centroid} if the density function is uniformly 1. 
	
	In mechanical problems, idealize a flat material as a plane region $R$ with a density function $\mu(x, y)$. This forces $\bar{z}=0$, and the remaining two are
	\begin{IEEEeqnarray}{rCl}
		\bar{x} & = & \frac{1}{M} \int_R x\mu(x, y)dxdy \\
		\bar{y} & = & \frac{1}{M} \int_R y\mu(x, y)dxdy, \quad \text{where } M=\int_R \mu(x, y)dxdy
	\end{IEEEeqnarray}
	
	Also, when joining disjoint regions, 
	\begin{equation}
		m(B_1 \cup B_2) = M(B_1) + M(B_2) \text{ and } M_P(B_1 \cup B_2) = M_P(B_1) + M_P(B_2)
	\end{equation}
	
	\subparagraph{Pappus's Theorem} The volume of a solid of revolution $B$ about a line $L$ is proportional to the circumference of the centroid: $V(B) = 2\pi \int_a^b rh(r)dr$.
	
\section{Improper Integrals}
	Standard integrals must be taken on a closed integral, and assume that the function is bounded. However, if either of these conditions is violated, for example if the interval is unbounded or $f$ tends to $\infty$ at a limit of integration, then 
	\begin{equation}
		\int_B f(\vec{x})dV = \lim_{\delta \rightarrow B} \int_{\delta} f(\vec{x})dV
	\end{equation}
	If the limit is finite and exists, the the integral converges to the value, otherwise the integral diverges.
	
	\subparagraph{Probability Densities} Integrals over unbounded regions often occur in statistics, usually in a way like the following.
	\begin{equation}
		\int_S p(\vec{x})dV = 1
	\end{equation}
	In this case, $p(\vec{x})$ is the density of a statistical outcome. The probability that the outcome of an experiment lies in a subset of $S$ is in the form
	\begin{equation}
		Pr[E \in B] = \int_B p(\vec{x})dV
	\end{equation}
	Analogous to the center of mass calculations, the mean of a probability distribution is
	\begin{equation}
		\vec{m}[p] = \int_B \vec{x}p(\vec{x})dV
	\end{equation}
	
\section{Numerical Integration}
	\subsection{Midpoint Approximation}
		For a double integral over a rectangle $R$, impose a grid with intersection points at
		\begin{equation}
			(x_j, y_k) = (a + j(b - a)/p, c + k(d - c)/q)
		\end{equation}
		With $p$ and $q$ being the number of lines in the grid in the $x$ and $y$ directions and $j$ and $k$ both start at 1. Then the midpoint approximation of the integral is
		\begin{equation}
			\int_R f(x, y)dxdy \approx \frac{(b - a)(d - c)}{pq} \sum_{j=0}^{p-1} \sum_{k=0}^{q-1} f(\bar{x}_j, \bar{y}_k)
		\end{equation}
		
	\subsection{Simpson Approximations}
		If $f$ is fairly smooth then it is possible to approximate a multiple integral by iterated Simpson approximations over an \textit{even} number of intervals.
		\begin{equation}
			\int_a^b f(x)dx \approx \frac{b-a}{3p}\sum_{j=0}^p S_j^{(p)} f(x_j)
		\end{equation}
		The Simpson series is defined as follows, essentially as alternating 4's and 2's:
		\begin{IEEEeqnarray}{rCl}
			S_0^{(p)} & = & 1\\
			S_p^{(p)} & = & 1\\
			S_j^{(p)} & = & 3 - (-1)^j
		\end{IEEEeqnarray}
%	\begin{center}
%	\begin{tikzpicture}
%		[scale=3,line cap=round,
%		%Styles
%		axes/.style=,
%		important line/.style={very thick},
%		information text/.style={rounded corners,fill=red!10,inner sep=1ex},
%		dot/.style={circle,inner sep=1pt,fill,label={#1},name=#1}			
%		]
%		
%		%Colors
%		\colorlet{anglecolor}{green!50!black}	%angle arcs/lines
%		
%		%The graphic
%	\end{tikzpicture}
%	\end{center}

%	\begin{figure}[htb]
%		\centering
%		\includegraphics[width=0.8\textwidth]{filename.eps}
%		\caption{Caption.}
%		\label{fig:figure}
%	\end{figure}

%		\def\enotesize{\normalsize}
%		\theendnotes
\end{document}