\documentclass[11pt]{article}
\usepackage{amsmath, amssymb, amsthm}
\usepackage[retainorgcmds]{IEEEtrantools}

\usepackage[pdftex]{graphicx}
\usepackage{tikz}
\usetikzlibrary{intersections}

\usepackage{marginnote}
\usepackage{endnotes}

\usepackage{fancyhdr}

%Listings stuff
\usepackage{listings}
\usepackage{lstautogobble}
\usepackage{color}

\definecolor{gray}{rgb}{0.5,0.5,0.5}
\lstset{
basicstyle={\small\ttfamily},
tabsize=3,
numbers=left,
numbersep=5pt,
numberstyle=\tiny\color{gray},
stepnumber=2,
breaklines=true
}

%Properly formatted differential 'd'
\newcommand{\ud}{\, \mathrm{d}}

%Format stuff
\pagestyle{fancy}
\headheight 35pt

%Header info
\chead{\Large \textbf{Equations and Matrices}}
\lhead{}
\rhead{}

\begin{document}
\section{Gaussian Elimination}
	Given a linear system of equations represented by $\mathbf{A}\vec{x} = \vec{b}$, where the matrix $\mathbf{A}$ represents the coefficients of the unknowns or variables in $\vec{x}$, it is possible with a series of basic manipulations to come up with a solution.
	
	\begin{equation}
		\left(\begin{matrix}
			a & b & c & | & d\\
			e & f & g & | & h\\
			i & j & k & | & l
		\end{matrix}\right)
	\end{equation}
	
	Turn $a$ into 1, and then use the top equation to zero out $e$ and $i$ by combining the rows with a scalar multiple of the first row.
	
	\begin{equation}
		\left(\begin{matrix}
			1 & m & n & | & o\\
			0 & p & q & | & r\\
			0 & s & t & | & u
		\end{matrix}\right)
	\end{equation}
	
	Turn $p$ into 1, and then turn $s$ into 0 by using the same method. Back substitute to obtain either an unique answer, a line, or an empty set of solutions.
	
	\begin{equation}
		\left(\begin{matrix}
			1 & m & n & | & o\\
			0 & 1 & v & | & w\\
			0 & 0 & x & | & y
		\end{matrix}\right)
	\end{equation}
	
	\subsection{Geometric Representation}
		In $\mathbb{R}^3$, the system of equations can be thought of as solving for the intersection of 3 planes. The solution can either be a single point, a line, or nothing. If one equation reduces to $0 = 0$ or some other tautology, then assign one variable (usually $z$) to be a \textbf{free variable} - the parameter - by turning it into $t$, and adding a third equation $z = t$. Then it is possible to express the system like such.
		\begin{IEEEeqnarray}{rCl}
			x & = & k_1t + c_1\\
			y & = & k_2t + c_2\\
			z & = & t
		\end{IEEEeqnarray}

	This is the natural parametric representation for a line.
%	\begin{center}
%	\begin{tikzpicture}
%		[scale=3,line cap=round,
%		%Styles
%		axes/.style=,
%		important line/.style={very thick},
%		information text/.style={rounded corners,fill=red!10,inner sep=1ex},
%		dot/.style={circle,inner sep=1pt,fill,label={#1},name=#1}			
%		]
%		
%		%Colors
%		\colorlet{anglecolor}{green!50!black}	%angle arcs/lines
%		
%		%The graphic
%	\end{tikzpicture}
%	\end{center}

%	\begin{figure}[htb]
%		\centering
%		\includegraphics[width=0.8\textwidth]{filename.eps}
%		\caption{Caption.}
%		\label{fig:figure}
%	\end{figure}

%		\def\enotesize{\normalsize}
%		\theendnotes
\end{document}