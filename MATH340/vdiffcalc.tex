\documentclass[11pt]{article}
\usepackage{amsmath, amssymb, amsthm}
\usepackage[retainorgcmds]{IEEEtrantools}

\usepackage[pdftex]{graphicx}
\usepackage{tikz}
\usetikzlibrary{intersections}

\usepackage{marginnote}
\usepackage{endnotes}

\usepackage{fancyhdr}

%Listings stuff
\usepackage{listings}
\usepackage{lstautogobble}
\usepackage{color}

\definecolor{gray}{rgb}{0.5,0.5,0.5}
\lstset{
basicstyle={\small\ttfamily},
tabsize=3,
numbers=left,
numbersep=5pt,
numberstyle=\tiny\color{gray},
stepnumber=2,
breaklines=true
}

%Properly formatted differential 'd'
\newcommand{\ud}{\, \mathrm{d}}

%Format stuff
\pagestyle{fancy}
\headheight 35pt

%Header info
\chead{\Large \textbf{Vector Differential Calculus}}
\lhead{}
\rhead{}

\begin{document}
\section{Gradient Fields}
	If $f$ is a differential real-valued function, $\vec{\nabla} f$ can be thought of as a \textbf{vector field}. If $\vec{\nabla} f = \vec{F}(x, y) = (f(x, y), g(x, y))$, then to draw the gradient field, at each point $(x, y)$ in the domain is an arrow from $(0,0)$ to $(\vec{F}(x, y)$. The arrow is then translated to start at $(x, y)$ and end at $\vec{F}(x, y) + (x, y)$.
	
	From the gradient field of any differentiable function $f$, it becomes apparent that at each point in the domain for which $\vec{\nabla} f \neq \vec{0}$, $\vec{nabla} f$ points in the direction of maximum increase. The number $|\vec{\nabla} f|$ is the maximum rate of increase of $f$ at $\vec{x}$.

%	\begin{center}
%	\begin{tikzpicture}
%		[scale=3,line cap=round,
%		%Styles
%		axes/.style=,
%		important line/.style={very thick},
%		information text/.style={rounded corners,fill=red!10,inner sep=1ex},
%		dot/.style={circle,inner sep=1pt,fill,label={#1},name=#1}			
%		]
%		
%		%Colors
%		\colorlet{anglecolor}{green!50!black}	%angle arcs/lines
%		
%		%The graphic
%	\end{tikzpicture}
%	\end{center}

%	\begin{figure}[htb]
%		\centering
%		\includegraphics[width=0.8\textwidth]{filename.eps}
%		\caption{Caption.}
%		\label{fig:figure}
%	\end{figure}

%		\def\enotesize{\normalsize}
%		\theendnotes
\end{document}