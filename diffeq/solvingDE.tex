\documentclass{article}
\usepackage{amsmath, amssymb}
\usepackage[retainorgcmds]{IEEEtrantools}
\newcommand{\ud}{\, \mathrm{d}}

\title{Solving Ordinary Differential Equations}
\date{}
\author{}

\begin{document}
\maketitle
\section{Notes}
	\subsection{Separation of Variables}
		Given an equation in the form:
		\begin{equation}
			\frac{\ud y}{\ud t} = \frac{g(t)}{h(y)}
		\end{equation}
		
		Separate all like differentials and variables, then integrate. Like so:
		\begin{align*}
			h(y)\ud y & = g(t)\ud t \nonumber \\
			\int h(y)\ud y & = \int g(t)\ud t + C \nonumber
		\end{align*}
	\subsection{1\textsuperscript{st} Order Linear ODE's}
		The general form of a 1\textsuperscript{st} order linear DE is as follows:
		\begin{equation}
			a_1(x)\frac{\ud y}{\ud x} + a_0(x)y = b(x)
		\end{equation}
		As a shortcut, note that in the special case that $a_0 = a_1\prime$, that becomes
		\begin{equation*}
			\frac{\ud}{\ud x}[a_1(x) \cdot y] = b(x)
		\end{equation*}
		To solve these types of differential equations, use the following steps:
		\begin{enumerate}
			\item Transform the equation into the following \textbf{standard form}:
				\begin{equation}
					\frac{\ud y}{\ud x} + P(x)y = Q(x)
				\end{equation}
			\item Calculate the \textbf{integrating factor}, which is as follows:
				\begin{equation}
					\mu(x) = e^{\int P(x)\ud x}
				\end{equation}
			\item Multiply the standard form by $\mu(x)$:
				\begin{equation}
					\mu(x) \cdot \frac{\ud y}{\ud x} + P(x)\mu(x)y = \mu(x)Q(x)
				\end{equation}
			\item Integrate the expression and separate variables to obtain y(x).
				\begin{equation}
					y = \frac{1}{\mu(x)} \cdot \int \mu(x)Q(x)\ud x
				\end{equation}
		\end{enumerate}
	\subsection{Exact Differential Equations}
		Any ordinary differential equation can be expressed in terms of its composing \emph{partials}, as such:
		\begin{equation}
			\frac{\ud y}{\ud x}(f(x, y)) = - \frac{\dfrac{\partial F}{\partial x}}{\dfrac{\partial F}{\partial y}}
		\end{equation}
		From here, we can derive the general form of a 1st order, exact differential equation:
		\begin{align}	\label{exactgeneral}
			M(x, y)\ud x + N(x, y)\ud y & = 0 \\
			\frac{\partial F}{\partial x}\ud x + \frac{\partial F}{\partial y}\ud y & =
			\ud F(x, y)
		\end{align}
		Solutions to these equations are in the form $F(x, y) = C$. To solve these DE's, use the following steps:
		\begin{enumerate}
			\item Check if the equation truly is exact, which it will be only if
				\begin{equation}
					\frac{\partial M}{\partial y} = \frac{\partial N}{\partial x}, \text{for $\mathbb{R}$}
				\end{equation}
			\item Integrate the general form with respect to either $x$ or $y$:
				\begin{align}	\label{exactstep2}
					F(x, y) & = \int M(x, y)\ud x + g(y) \\
					F(x, y) & = g(x) + \int N(x, y)\ud y
				\end{align}
			\item To obtain $G$, take the partial with respect to the opposite variable the general form was integrated with respect to, then substitute $N(x,y)$ for $\partial F/\partial y$ or $M(x,y)$ for $\partial F/\partial x$ - refer to the general form \eqref{exactgeneral}
				\begin{align}
					N(x, y) & = \frac{\partial \int M(x, y)\ud x}{\partial y} + g\prime(y) \\
					M(x, y) & = g\prime(x) + \frac{\partial \int N(x, y)\ud y}{\partial x}
				\end{align}
			\item Solve for $g\prime$
			\item Integrate to obtain $G$: \[G = \int g + C\]
			\item Plug G into \eqref{exactstep2}, then reorder to obtain the solution $F(x, y) = C$
		\end{enumerate}
		\subsubsection{Integrating Factors}
			Sometimes, the differential equation needs to be transformed by an \textbf{integrating factor} before it can be considered exact. To determine the integrating factor, use the following method:
			\begin{itemize}
				\item If the expression
					\begin{equation} \label{intfactx}
						\frac{\left(\dfrac{\partial M}{\partial y} - \dfrac{\partial N}{\partial x}\right)}{N}
					\end{equation}
					is both continuous and depends \emph{only} on x, -OR-
				\item If the expression
					\begin{equation} \label{intfacty}
						\frac{\left(\dfrac{\partial M}{\partial x} - \dfrac{\partial N}{\partial y}\right)}{M}
					\end{equation}
					is both continuous and depends \emph{only} on y,
			\end{itemize}
			Then the integrating factor is in the following form:
			\begin{equation}
				\mu(x) = e^{\int K}
			\end{equation}
			, where K is the expression in \eqref{intfactx} or \eqref{intfacty} and the integral is taken with respect to x in the case of \eqref{intfactx} and y in the case of \eqref{intfacty}. Multiply the original general form differential equation by $\mu(x)$ and it will be exact.
\section{Examples}
	\subsection{Separation of Variables}
		\begin{IEEEeqnarray}{rCl}
			(y - yx^{2})\frac{\ud y}{\ud x} & = & (y+1)^{2} \\
			(y - yx^{2})\ud y & = & (y+1)^{2}\ud x \nonumber\\
			y(1-x^{2})\ud y & = & (y+1)^{2}\ud x \nonumber\\
			\frac{y\ud y}{(y+1)^{2}} & = & (1-x^{2})^{-1}\ud x \nonumber\\
			\int\frac{y\ud y}{(y+1)^{2}} & = & \int\frac{\ud x}{1 - x^{2}} \nonumber\\
			u & = & y+1 \nonumber\\
			\ud u & = & \ud y \nonumber\\
			\int\frac{(u - 1)\ud u}{u^{2}} & = & \int\frac{\ud x}{1 - x^{2}} \nonumber\\
			\ln|y+1| + \frac{1}{y+1} & = & -\frac{1}{2}\ln\left|\dfrac{x-1}{x+1}\right| + C
		\end{IEEEeqnarray}
	\subsection{Linear Equations}
		\begin{enumerate}
			\item \begin{IEEEeqnarray}{rCl}
				x^{2}\frac{\ud y}{\ud x} + x(x+2)y & = & e^{x} \\
				\frac{\ud y}{\ud x} + \frac{x+2}{x}y & = & \frac{e^{x}}{x^{2}} \nonumber\\
				\mu(x) & = & e^{\int\frac{x+2}{x}\ud x} \nonumber\\
				\mu(x) & = & e^{x}x^{2} \nonumber\\
				e^{x}x^{2}y & = & \int e^{x}x^{2}\cdot\dfrac{e^{x}}{x^{2}}\ud x \nonumber\\
				y & = & \dfrac{e^{x}}{2x^{2}} + \dfrac{c}{e^{x}}{x^{2}}
				\end{IEEEeqnarray}
			\item \begin{IEEEeqnarray}{rClr}
				\frac{\ud y}{\ud x} & = & y\sin x + 2xe^{-\cos x} & y(0) = 1 \\
				\frac{\ud y}{\ud x} - y\sin x & = & 2xe^{-\cos x} \nonumber\\
				\mu(x) & = & e^{\int -\sin x\ud x} \nonumber\\
				\mu(x) & = & e^{\cos x} \nonumber\\
				e^{\cos x} y & = & \int 2xe^{\cos x}\cdot e^{\cos x}\ud x \nonumber\\
				y & = & \frac{x^{2} + c}{e^{\cos x}} \\
				1 & = & \frac{0^{2} + c}{e^{\cos 0}} \nonumber\\
				c & = & e \nonumber\\
				y & = & \frac{x^{2} + e}{e^{\cos x}}
				\end{IEEEeqnarray}
		\end{enumerate}
	\subsection{Exact Differential Equations}
		\begin{enumerate}
			\item \begin{IEEEeqnarray}{rCl}
				(x\sin y - y^2)\frac{\ud y}{\ud x} & = & \cos y \\
				(x\sin y - y^2)\ud y - \cos y\ud x & = & 0 \nonumber\\
				\frac{\partial M}{\partial y} = \frac{\partial N}{\partial x} & = & \sin y \nonumber\\
				-\int\cos y \ud x + g(y) & = & z \nonumber\\
				-x\cos y + g(y) & = & z \nonumber\\
				x\sin y + g'(y) & = & x\sin y - y^2 \nonumber\\
				g'(y) & = & -y^2 \nonumber\\
				g(y) & = & -\frac{1}{3} y^3 \nonumber\\
				-x\cos y - \frac{1}{3} y^3 & = & C
				\end{IEEEeqnarray}
			\item \begin{IEEEeqnarray}{rCl}
				(\cos x\cos y)\frac{\ud y}{\ud x} + \tan x & = & \sin x\sin y \\
				(\cos x\cos y)\ud y & = & (\sin x\sin y - \tan x)\ud x \nonumber\\
				\frac{\partial M}{\partial y} = \frac{\partial N}{\partial x} & = & \sin x\cos y \nonumber\\
				-\int(\cos x\cos y + y)\ud y + g(x) & = & z \nonumber\\
				-\cos x\sin y - \frac{y^2}{2} + g(x) & = & z \nonumber\\
				\sin x\sin y + g'(x) & = & \sin x\sin y - \tan x \nonumber\\
				g(x) & = & \ln|\cos x| \nonumber\\
				-\cos x\sin y - \frac{y^2}{2} + \ln|\cos x| & = & C
				\end{IEEEeqnarray}
		\end{enumerate}
\end{document}