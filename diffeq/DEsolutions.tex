\documentclass{article}
\usepackage{amsmath, amssymb}
\newcommand{\ud}{\, \mathrm{d}}
\newtheorem{picard}{Picard's Theorem}

\title{Solutions of Differential Equations}
\date{}
\author{}

\begin{document}
\maketitle
\section{Notes}
\subsection{General Form}
General forms of solutions to differential equations are as follows:
\begin{equation}
	\begin{cases}
	f(x) & = 0 \\
	f(x, y) & = c
	\end{cases}
\end{equation}
\subsection{Explicit and Implicit Solutions}
\begin{description}
	\item[Explicit Solution]: Substitue $\phi(x)$ for $y$ and satisfy the 
	D.E. for $\mathbb{R}$
	\item[Implicit Solution]: $G(x, y) = 0$ is an explicit solution on some interval
\end{description}
\subsection{Unique Solutions}
\begin{picard}
	Given the following initial value problem:
	\begin{equation}
	\frac{\ud y}{\ud x} = f(x, y), y(x_0) = y_0
	\end{equation}
	If both $f(x, y)$ and $\partial f / \partial y$ are both continuous in a rectangle
	defined by the bounds $R = \left\{(x, y): a<x<b, c<y<d\right\}$ that contains $(x_0, y_0)$,
	then the IVP has a unique solution. \emph{\textbf{In other words, if $f(x, y)$ and $\partial f / \partial y$ are continuous at the point $(x_0, y_0)$, a unique solution exists.}}
\end{picard}
\section{Examples}
\begin{enumerate}
	\item Show $y = (\sqrt{t} + c)^2$ is a solution of 
		\begin{equation}
		\frac{\ud y}{\ud t} = \sqrt{\frac{y}{t}}
		\end{equation}
		for $t = 0$, $c > 0$.

		Begin by substituting $\phi\prime(x)$ for $y\prime$:
		\begin{align*}
		\phi(t) & = (\sqrt{t} + c)^2 \\
		\phi \prime(t) & = \frac{2(\sqrt{t} + c)}{2\sqrt{t}} \\
		& = 1 + ct^{-1/2} \\
		\end{align*}
		Then substitute $\phi(x)$ for $y$ and compare with previous result:
		\begin{align*}
		\frac{\ud y}{\ud t} & = \sqrt{\frac{\phi(t)}{t}} \\
		& = \sqrt{\frac{(\sqrt{t} + c)^2}{t}} \\
		& = \frac{\sqrt{t} + c}{\sqrt{t}} \\
		& = 1 + ct^{-1/2} \\
		\phi \prime(t) & \, \check{=} \, \frac{\ud y}{\ud t}
		\end{align*}
	\item Given:
		\begin{equation}
		\begin{cases}
		y & = c_1 + c_2\cos x + c_3\sin x \\[5pt]
		y\prime\prime\prime + y\prime & = 0
		\end{cases}
		\end{equation}
		Solve for the following initial conditions:
		\begin{equation*}
		\begin{cases}
		y(\pi) = & 0 \\
		y\prime(\pi) = & 2 \\
		y\prime\prime(\pi) = & -1
		\end{cases}
		\end{equation*}
		Begin by differentiating $y(x)$:
		\begin{align*}
		y\prime & = -c_2\sin x + c_3\cos x \\
		y\prime\prime & = -c_2\cos x - c_3\sin x
		\end{align*}
		Then plug in the values given in the initial conditions to arrive at the following answer:
		\[\begin{cases}
		c_1 & = -1 \\
		c_2 & = -1 \\
		c_3 & = 2
		\end{cases}\]
\end{enumerate}
\end{document}