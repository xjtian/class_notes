\documentclass[11pt]{article}
\usepackage{amsmath, amssymb, amsthm}
\usepackage[retainorgcmds]{IEEEtrantools}

\usepackage[pdftex]{graphicx}
\usepackage{tikz}
\usetikzlibrary{intersections}
\usepackage{circuitikz}

\usepackage{fancyhdr}

%Listings stuff
\usepackage{listings}
\usepackage{lstautogobble}
\usepackage{color}

\definecolor{gray}{rgb}{0.5,0.5,0.5}
\lstset{
basicstyle={\small\ttfamily},
tabsize=3,
numbers=left,
numbersep=5pt,
numberstyle=\tiny\color{gray},
stepnumber=2,
breaklines=true,
boxpos=t
}

%Format stuff
\pagestyle{fancy}
\headheight 35pt

%Header info
\chead{\Large \textbf{Phasor Analysis of AC Circuits}}
\lhead{}
\rhead{}

\begin{document}
\section{Basic Principles}
	A \textbf{phasor} is a geometric vector representation of a sinusoidal function with constant amplitude multiplier, frequency, and phase. It is represented as a vector in the imaginary plane that rotates around the origin at a constant frequency. Phasors can be used in the analysis of AC circuits because power sources supply an oscillating voltage at a constant frequency.
	
	For the following analyses, assume that $V_S = V_0 e^{i\omega t}$, that is the voltage supplied from the source is a simple harmonic function.
	
	\subparagraph{Resistors} Current and voltage across a resistor are in phase as long as the resistance is real.
		\begin{IEEEeqnarray}{rCl}
			V_R & = & IR\\
			I & = & \frac{V_0}{R}e^{i\omega t}
		\end{IEEEeqnarray}
		
	\subparagraph{Capacitors} Current and voltage across a capacitor are exactly $\pi/4$ out of phase, with the capacitor voltage lagging behind the current.
		\begin{IEEEeqnarray}{rCl}
			Q & = & CV\\
			I & = & \dot{Q} = C\frac{dV}{dt}\\
			I & = & i\omega CV_0 e^{i\omega t}\\
			V & = & I\left(\frac{1}{i \omega C}\right)\\
			X_C & = & \frac{1}{i \omega C}\\\nonumber\\
			V & = & IX_C
		\end{IEEEeqnarray}
		
	\subparagraph{Inductors} Current and voltage across an inductor are exactly $\pi/4$ out of phase, with the inductor voltage ahead of the current.
		\begin{IEEEeqnarray}{rCl}
			V & = & L\frac{dI}{dt}\\
			I & = & \frac{1}{i\omega L}(V_0e^{i\omega t})\\
			V & = & I(i\omega L)\\
			X_L & = & i\omega L\\
			V & = & IX_L
		\end{IEEEeqnarray}

%	\begin{center}
%	\begin{tikzpicture}
%		[scale=3,line cap=round,
%		%Styles
%		axes/.style=,
%		important line/.style={very thick},
%		information text/.style={rounded corners,fill=red!10,inner sep=1ex},
%		dot/.style={circle,inner sep=1pt,fill,label={#1},name=#1}			
%		]
%		
%		%Colors
%		\colorlet{anglecolor}{green!50!black}	%angle arcs/lines
%		
%		%The graphic
%	\end{tikzpicture}
%	\end{center}

%	\begin{figure}[htb]
%		\centering
%		\includegraphics[width=0.8\textwidth]{filename.eps}
%		\caption{Caption.}
%		\label{fig:figure}
%	\end{figure}

%		\def\enotesize{\normalsize}
%		\theendnotes
\end{document}