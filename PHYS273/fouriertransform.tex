\documentclass[11pt]{article}
\usepackage{amsmath, amssymb, amsthm}
\usepackage[retainorgcmds]{IEEEtrantools}

\usepackage{fancyhdr}

%Format stuff
\pagestyle{fancy}
\headheight 35pt

%Header info
\chead{\Large \textbf{Fourier Transform}}
\lhead{}
\rhead{}

\begin{document}
	Fourier series can be used to represent non-periodic functions if $L$ is taken to $\infty$ - that is, represent a continuous function $f(x)$ as a periodic function with an infinite period.

	First, let $\Delta n = 1$ as the amount $n$ changes per iteration.
	\begin{IEEEeqnarray}{rCl}
		f(x) & = & \sum_{n=-\infty}^\infty c_n e^{in\pi x/L}\\
		& = & \sum_{n=-\infty}^\infty c_n e^{in\pi x/L} \Delta n\\
		& = & \sum_{n=-\infty}^\infty \left(\frac{L}{\pi}\right) c_n e^{in\pi x/L} \left (\frac{\pi \Delta n}{L} \right)
	\end{IEEEeqnarray}
	Define $k = n\pi/L$, which leads obviously to $\Delta k = \pi \Delta n / L$. Also, define $A(k)$ as the following continuous function:
	\begin{equation}
		A(k) = \sqrt{2\pi} \left( \frac{Lc_n}{\pi} \right)
	\end{equation}
	Substitution gives
	\begin{equation}
		f(x) = \sum_{n=-\infty}^\infty \frac{A(k)}{\sqrt{2\pi}} e^{ikx} \Delta k
	\end{equation}
	Taking $\lim_{L\rightarrow \infty}$ gives the \textbf{Fourier Transformation} of $f(x)$:
	\begin{equation}
		f(x) = \frac{1}{\sqrt{2\pi}} \int_{-\infty}^\infty A(k)e^{ikx}dk
	\end{equation}
	
	After the limit, $A(k)$ is a continuous function of expansion coefficients and $e^{ikx}$ is a continuous function of basis vectors. use \textbf{Planchere's Theorem} to determine $A(k)$.
	\begin{IEEEeqnarray}{rCl}
		c_n & = & \frac{1}{2L} \int_{-L}^L f(x)e^{-in\pi x/L} dx\\
		c_n \left( \frac{L}{\pi} \right) \sqrt{2\pi} & = & \frac{\sqrt{2\pi}}{2\pi} \int_{-L}^L f(x)e^{-ikx} dx
	\end{IEEEeqnarray}
	Note that we defined $A(k) = \sqrt{2\pi}(Lc_n/\pi)$ in the first place, so substitute on the left hand side and take $\lim_{L\rightarrow \infty}$ to get the final result:
	\begin{equation}
		A(k) = \frac{1}{\sqrt{2\pi}} \int_{-\infty}^\infty f(x) e^{-ikx} dx
	\end{equation}

%	\begin{center}
%	\begin{tikzpicture}
%		[scale=3,line cap=round,
%		%Styles
%		axes/.style=,
%		important line/.style={very thick},
%		information text/.style={rounded corners,fill=red!10,inner sep=1ex},
%		dot/.style={circle,inner sep=1pt,fill,label={#1},name=#1}			
%		]
%		
%		%Colors
%		\colorlet{anglecolor}{green!50!black}	%angle arcs/lines
%		
%		%The graphic
%	\end{tikzpicture}
%	\end{center}

%	\begin{figure}[htb]
%		\centering
%		\includegraphics[width=0.8\textwidth]{filename.eps}
%		\caption{Caption.}
%		\label{fig:figure}
%	\end{figure}

%		\def\enotesize{\normalsize}
%		\theendnotes
\end{document}