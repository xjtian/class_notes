\documentclass[11pt]{article}
\usepackage{amsmath, amssymb, amsthm}
\usepackage[retainorgcmds]{IEEEtrantools}

\usepackage[pdftex]{graphicx}
\usepackage{tikz}
\usetikzlibrary{intersections,decorations.pathmorphing}

\usepackage{fancyhdr}

%Format stuff
\pagestyle{fancy}
\headheight 35pt

%Header info
\chead{\Large \textbf{Forced Oscillators}}
\lhead{}
\rhead{}

\begin{document}
Assume, to be concrete, that for a forced oscillator the forcing function is itself an oscillating function of the form 
	\begin{equation}
		F(t) = F_0 \cos(\omega_f t)
	\end{equation}
\begin{center}
\begin{tikzpicture}
	[scale=1,line cap=round,
	%Styles
	axes/.style=,
	important line/.style={very thick},
	information text/.style={rounded corners,fill=red!10,inner sep=1ex},
	dot/.style={circle,inner sep=1pt,fill,label={#1},name=#1},
	spring/.style={decorate, draw=blue,decoration={coil,amplitude=4pt, segment length=5pt}}			
	]
	
	%Colors
	\colorlet{anglecolor}{green!50!black}	%angle arcs/lines
	
	%The graphic
	\draw[black] (0,0) -- (0, 4);
	\draw[black] (0, 0) -- (7.5, 0);	
	
	\draw[blue,thick] (0, 1) -- (1, 1);
	\draw[spring] (1, 1) -- node[above=2pt,black] {$k$} (3, 1);
	\draw[blue,thick] (3, 1) -- (3.5, 1);
	
	\draw[orange,thick] (3.5, 0) rectangle (5.5, 2);
	\node (m) at (4.5, 1) {$m$};
	
	\draw[green!80!black, thick, ->] (5.5, 1) -- node[above,black] {$F(t)$} (7, 1);
	
\end{tikzpicture}
\end{center}
\section{Equation of Motion}
	\begin{IEEEeqnarray}{rCl}
		\sum F & = & -kx + F(t)\\
		m\ddot{x} & = & -kx + F(t)\\
		\frac{F(t)}{m} & = & \ddot{x} + \frac{k}{m}x
	\end{IEEEeqnarray}
	
	\begin{equation}
		\ddot{x} + \omega_0^2 x = \frac{F(t)}{m}
	\end{equation}
	
	The solution to this differential equation is as follows.
	\begin{IEEEeqnarray}{rCl}
		x(t) & = & A e^{i(\omega_f t + \delta)}\\
		A & = & \frac{F_0}{m(\omega_0^2 - \omega_f^2)}
	\end{IEEEeqnarray}
	
\section{Analysis}
	The actual value of $A$, computed by assuming both $F(t)$ and $x$ are complex, is 
	\[\frac{F_0 e^{-i\delta}}{m(\omega_0^2 - \omega_f^2)}\]
	But by convention, make $A$ real because we can expand it in polar form and add exponents. This means that the expression should also be real, so $e^{-i\delta} \in \mathbb{R}$ and therefore $\delta = 0$. 
	
	This is important because $\delta = 0$ implies that the \textbf{forcing function and $x(t)$ are in phase}. Then $t=0$ is a arbitrary starting point at a moment when the driving force is maximal, and $\delta$ is the offset (phase) from this starting position.
	
	If $\omega_f < \omega_0$, the force and oscillator are in-phase and motion is described by a constant amplitude. If $\omega_f > \omega_0$, the force and oscillator are out of phase by $180\deg$. If $\omega_f = \omega_0$, then the system is in \textbf{resonance} and $A$ goes to infinity.
%	\begin{center}
%	\begin{tikzpicture}
%		[scale=3,line cap=round,
%		%Styles
%		axes/.style=,
%		important line/.style={very thick},
%		information text/.style={rounded corners,fill=red!10,inner sep=1ex},
%		dot/.style={circle,inner sep=1pt,fill,label={#1},name=#1}			
%		]
%		
%		%Colors
%		\colorlet{anglecolor}{green!50!black}	%angle arcs/lines
%		
%		%The graphic
%	\end{tikzpicture}
%	\end{center}

%	\begin{figure}[htb]
%		\centering
%		\includegraphics[width=0.8\textwidth]{filename.eps}
%		\caption{Caption.}
%		\label{fig:figure}
%	\end{figure}

%		\def\enotesize{\normalsize}
%		\theendnotes
\end{document}