\documentclass[11pt]{article}
\usepackage{amsmath, amssymb, amsthm}
\usepackage[retainorgcmds]{IEEEtrantools}

\usepackage[pdftex]{graphicx}
\usepackage{tikz}
\usepackage{circuitikz}
\usetikzlibrary{intersections}

\usepackage{fancyhdr}

%Listings stuff
\usepackage{listings}
\usepackage{lstautogobble}
\usepackage{color}

\definecolor{gray}{rgb}{0.5,0.5,0.5}
\lstset{
basicstyle={\small\ttfamily},
tabsize=3,
numbers=left,
numbersep=5pt,
numberstyle=\tiny\color{gray},
stepnumber=2,
breaklines=true,
boxpos=t
}

%Format stuff
\pagestyle{fancy}
\headheight 35pt

%Header info
\chead{\Large \textbf{Electrical Impedance}}
\lhead{}
\rhead{}

\begin{document}
Remember that in deriving the voltage equations for resistors, capacitors, and inductors, all were in the general form $V = IZ$, with $Z$ as the impedance of the component.

\section{Series Impedance}
	Combining components in series means that each has a different voltage drop but share the same current.
	\begin{IEEEeqnarray}{rCl}
		V_{total} & = & I\sum_i Z\\
		Z_{series} & = & \sum_i Z
	\end{IEEEeqnarray}		

\section{Parallel Impedance}
	Combining components in parallel means that each share the same voltage drop but the branches have different current.
	
	\begin{IEEEeqnarray}{rCl}
		I_{total} & = & \sum_i I = V \sum_i \frac{1}{Z}\\
		Z_{parallel} & = & \frac{1}{\sum_i (Z^{-1})}
	\end{IEEEeqnarray}
	
\section{Driven RL Parallel}
	\begin{center}
	\begin{circuitikz}
		\draw (0,0) to[vsourcesin] (0,3) -- (2,3) to[R] (2,0) -- (0,0);
		\draw (2,3) -- (4,3) to[L] (4,0) -- (2,0);
	\end{circuitikz}
	\end{center}
	
	\begin{IEEEeqnarray}{rCl}
		Z_{total} & = & \frac{1}{Z_R^{-1} + Z_L^{-1}}\\
		& = & \frac{1}{R^{-1} + (i\omega L)^{-1}}\\
		& = & \frac{(R\omega L)(\omega L + iR)}{(\omega L - iR)(\omega L + iR)}\\
		& = & \frac{R\omega L^2}{(\omega L)^2 + R^2} \left(\omega + i\frac{R}{L}\right)\\
		Z_{total} & = & \frac{R\omega L^2}{(\omega L)^2 + R^2} (\omega + i\gamma)
	\end{IEEEeqnarray}
	
	Given that the current is a real number and $\vec{V} = \vec{I} Z$, the phase difference between $\vec{V}_S$ and $\vec{I}$ is a fairly simple calculation because the impedance cancels out between the numerator and denominator:
	\begin{equation}
		\delta = \tan^{-1}\left(\frac{\gamma}{\omega}\right)
	\end{equation}
	This means that for a very high $\omega$, the phase difference approaches 0 and the system acts like a resistor (current oscillates too fast for inductor to charge), and for a very low $\omega$, the system acts like an inductor.
%	\begin{center}
%	\begin{tikzpicture}
%		[scale=3,line cap=round,
%		%Styles
%		axes/.style=,
%		important line/.style={very thick},
%		information text/.style={rounded corners,fill=red!10,inner sep=1ex},
%		dot/.style={circle,inner sep=1pt,fill,label={#1},name=#1}			
%		]
%		
%		%Colors
%		\colorlet{anglecolor}{green!50!black}	%angle arcs/lines
%		
%		%The graphic
%	\end{tikzpicture}
%	\end{center}

%	\begin{figure}[htb]
%		\centering
%		\includegraphics[width=0.8\textwidth]{filename.eps}
%		\caption{Caption.}
%		\label{fig:figure}
%	\end{figure}

%		\def\enotesize{\normalsize}
%		\theendnotes
\end{document}