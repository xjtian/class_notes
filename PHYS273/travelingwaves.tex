\documentclass[11pt]{article}
\usepackage{amsmath, amssymb, amsthm}
\usepackage[retainorgcmds]{IEEEtrantools}

\usepackage[pdftex]{graphicx}
\usepackage{tikz}
\usepackage{circuitikz}
\usetikzlibrary{intersections}

\usepackage{fancyhdr}

%Listings stuff
\usepackage{listings}
\usepackage{lstautogobble}
\usepackage{color}

\definecolor{gray}{rgb}{0.5,0.5,0.5}
\lstset{
basicstyle={\small\ttfamily},
tabsize=3,
numbers=left,
numbersep=5pt,
numberstyle=\tiny\color{gray},
stepnumber=2,
breaklines=true,
boxpos=t
}

%Format stuff
\pagestyle{fancy}
\headheight 35pt

%Header info
\chead{\Large \textbf{Traveling Waves}}
\lhead{}
\rhead{}

\begin{document}
Rewrite the normal modes of the loaded string, taking only the real parts:
\begin{IEEEeqnarray}{rCl}
	y_n(x, t) & = & A_n\sin\left( \frac{n\pi x}{L} \right) e^{i\omega_n t}\\
	& = & A_n\sin\left( \frac{n\pi x}{L} \right) \cos(\omega_n t)\\
	& = & \frac{A_n}{2}\left[ \sin\left(\frac{n\pi}{L}x - \omega_n t \right) + \sin\left( \frac{n\pi}{L} x + \omega_n t \right) \right]
\end{IEEEeqnarray}
Introduce $k = n\pi / L$, called the \textbf{wave number}, and substitute the actual value for $\omega_n$:
\begin{equation}
	y_n(x, t) = \frac{A_n}{2} \left[ \sin\left( k\left( x - sqrt{\frac{T}{\mu}} t \right) \right) + \sin\left( k\left( x + sqrt{\frac{T}{\mu}} t \right) \right) \right]
\end{equation}

Each of the above sine functions is known as a \textbf{traveling wave}. At any moment in time, each looks like a perfect sine wave, but as time goes forward, the wave moves to the right or left.

To follow the wave as it moves forward, hold the phase constant.
\begin{IEEEeqnarray}{rCl}
	\Delta (phase) = 0 & = & k\left( \Delta x - \sqrt{\frac{T}{\mu}} \Delta t\right)\\
	\frac{dx}{dt} & = & \sqrt{\frac{T}{\mu}} = v_p
\end{IEEEeqnarray}
This velocity is called the \textbf{phase velocity}, and with this expression the normal mode solution is
\begin{equation}
	y_n(x, t) = \frac{A_n}{2} [ \sin((k(x - v_p t)) + \sin(k(x+v_p t)) ]
\end{equation}
Substituting any one of the traveling wave equations into the classical wave equation gives
\begin{equation}
	\frac{\partial^2 y}{\partial x^2} = \frac{1}{v_p^2} \frac{\partial^2 y}{\partial t^2}
\end{equation}
While the traveling wave fits the classical wave equation, it cannot be written as a linear combination of typical normal modes because the normal modes for the loaded string are written for boundary conditions that the traveling wave does not meet.

However, for an infinitely long loaded string with no boundary, an acceptable solution is the traveling wave
\begin{equation}
	y(x, t) = A\sin(kx - \omega t)
\end{equation}
where $\omega = kv$. Note for this solution, $\lambda = 2\pi / k$.

For a fixed string, due to the boundary conditions, $k$ can only be a discrete set $n\pi /L$. This means that $\omega$ must also be discrete. In other words, the boundary conditions pick out a discrete set of $k$ and $\omega$ values which are allowed.
\begin{equation}
	\omega_n = k_n v_p = \left( \frac{n\pi}{L} \right)v_p
\end{equation}

Without boundary conditions, any value of $k$ is allowed so long as $\omega = v_p k$. In this case, the general solution is no longer a discrete sum, but a reverse Fourier Transform.
\begin{equation}
	y(x, t) = \frac{1}{\sqrt{2\pi}} \left[ \int_{-\infty}^{\infty} A(k) e^{ikx} dk \right]e^{i\omega t}
\end{equation}
or
\begin{equation}
	y(x, t) = \frac{1}{\sqrt{2\pi}} \int_{-\infty}^\infty A(k)e^{i(kx + \omega t)}dk
\end{equation}

%	\begin{center}
%	\begin{tikzpicture}
%		[scale=3,line cap=round,
%		%Styles
%		axes/.style=,
%		important line/.style={very thick},
%		information text/.style={rounded corners,fill=red!10,inner sep=1ex},
%		dot/.style={circle,inner sep=1pt,fill,label={#1},name=#1}			
%		]
%		
%		%Colors
%		\colorlet{anglecolor}{green!50!black}	%angle arcs/lines
%		
%		%The graphic
%	\end{tikzpicture}
%	\end{center}

%	\begin{figure}[htb]
%		\centering
%		\includegraphics[width=0.8\textwidth]{filename.eps}
%		\caption{Caption.}
%		\label{fig:figure}
%	\end{figure}

%		\def\enotesize{\normalsize}
%		\theendnotes
\end{document}