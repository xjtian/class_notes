\documentclass[11pt]{article}
\usepackage{amsmath, amssymb, amsthm}
\usepackage[retainorgcmds]{IEEEtrantools}

\usepackage[pdftex]{graphicx}
\usepackage{tikz}
\usepackage{circuitikz}
\usetikzlibrary{intersections}

\usepackage{fancyhdr}

%Format stuff
\pagestyle{fancy}
\headheight 35pt

%Header info
\chead{\Large \textbf{Dispersion}}
\lhead{}
\rhead{}

\begin{document}
For a perfect traveling wave with period $T$ and wavelength $\lambda$, remember that
\begin{equation}
	v_p = \frac{\lambda}{T} = \frac{\omega}{k}
\end{equation}
The traveling wave satisfies the classical wave equation, which means that the ratio between $\omega$ and $k$, or $v_p$, is constant, or $\omega$ is directly proportional to $k$.
	\begin{IEEEeqnarray}{rCl}
		\frac{\partial^2 y}{\partial x^2} & = & \alpha \frac{\partial^2 y}{\partial t^2}\\
		\frac{\omega}{k} & = & \frac{1}{\sqrt{\alpha}}
	\end{IEEEeqnarray}
	
	Recall the general solution to a traveling wave system:
	\begin{equation}
		y(x, t) = \frac{1}{\sqrt{2\pi}} \int_{-\infty}^\infty A(k)e^{ikx}e^{-i\omega t}dk
	\end{equation}
For the perfect traveling wave, $\omega(k) = v_p k$ where $v_p$ is a fixed value, so
	\begin{equation}
		y(x, t) = \frac{1}{\sqrt{2\pi}} \int_{-\infty}^\infty A(k)e^{ik(x-v_p t)}dk
	\end{equation}
In this case, the shape of the original pulse, represented by the Fourier transform $A(k)$, translates across the medium while maintaining its shape because as $t$ changes, every wave number undergoes the same phase change, $v_p \Delta t$.

	For a general $\omega(k)$ for a combination of multiple traveling waves, the general form of the solution is
	\begin{equation}
		y(x, t) = \frac{1}{\sqrt{2\pi}} \int_{-\infty}^\infty A(k)e^{ik(x-(\omega(k)/k) t)}dk
	\end{equation}
	
	The $\omega(k)$ time factor is called the \textbf{dispersion relation}. If $\omega(k)$ is directly proportional to $k$, then there is no dispersion in the wave. If $omega(k)$ is lower for higher wave numbers, then the wave undergoes \textbf{normal dispersion}. If $omega(k)$ is high for higher wave numbers, then the wave undergoes \textbf{anomalous dispersion}.

%	\begin{center}
%	\begin{tikzpicture}
%		[scale=3,line cap=round,
%		%Styles
%		axes/.style=,
%		important line/.style={very thick},
%		information text/.style={rounded corners,fill=red!10,inner sep=1ex},
%		dot/.style={circle,inner sep=1pt,fill,label={#1},name=#1}			
%		]
%		
%		%Colors
%		\colorlet{anglecolor}{green!50!black}	%angle arcs/lines
%		
%		%The graphic
%	\end{tikzpicture}
%	\end{center}

%	\begin{figure}[htb]
%		\centering
%		\includegraphics[width=0.8\textwidth]{filename.eps}
%		\caption{Caption.}
%		\label{fig:figure}
%	\end{figure}

%		\def\enotesize{\normalsize}
%		\theendnotes
\end{document}