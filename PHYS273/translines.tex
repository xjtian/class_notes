\documentclass[11pt]{article}
\usepackage{amsmath, amssymb, amsthm}
\usepackage[retainorgcmds]{IEEEtrantools}

\usepackage[pdftex]{graphicx}
\usepackage{tikz}
\usepackage{circuitikz}
\usetikzlibrary{intersections}

\usepackage{fancyhdr}

%Format stuff
\pagestyle{fancy}
\headheight 35pt

%Header info
\chead{\Large \textbf{Transmission Lines}}
\lhead{}
\rhead{}

\begin{document}
Any two conductors separated by some distance have some capcitance and any length of wire has inductance, so two parallel wires can be modeled by a simple circuit:
\begin{center}
\begin{circuitikz}
	\draw (0, 1) to[L] (3, 1) -- (4, 1);
	\draw (0, 0) -- (4, 0);
	\draw (3, 1) to[C] (3, 0);
\end{circuitikz}
\end{center}
If we define the inductance and capacitance per unit length of the wires as $L_0$ and $C_0$, then 
	\begin{IEEEeqnarray}{rCl}
		I_{out} & = & I + \frac{\partial I}{\partial x}dx\\
		V_{out} & = & V + \frac{\partial V}{\partial x}dx
	\end{IEEEeqnarray}
	
	The voltage drop in the circuit can only be due to the inductor, and the current drop can only be due to the capacitor:
	\begin{IEEEeqnarray}{rCl}
		\frac{\partial V}{\partial x}dx & = & -(L_0dx)\frac{\partial I}{\partial t}\\
		\frac{\partial V}{\partial x} & = & -L_0 \frac{\partial I}{\partial t}\\
	\end{IEEEeqnarray}
	\begin{IEEEeqnarray}{rCl}
		\frac{\partial I}{\partial x}dx = \frac{dQ_c}{dt} & = & \frac{\partial}{\partial t}(CV)\\
		-\frac{\partial I}{\partial x} & = & C_0 \frac{\partial V}{\partial t}
	\end{IEEEeqnarray}
	
	Taking $\partial/\partial t$ of the current differential and $\partial/\partial x$ of the voltage differential leads to two coupled differential equations that satisfy the wave equation;
	\begin{IEEEeqnarray}{rCl}
		\frac{\partial^2 V}{\partial x^2} & = & L_0C_0 \frac{\partial^2 V}{\partial t^2}\\
		\frac{\partial^2 I}{\partial x^2} & = & L_0C_0 \frac{\partial^2 I}{\partial t^2}
	\end{IEEEeqnarray}
	
	Because these satisfy the classical wave equation, traveling waves are a possible solution for the system. In this case, from the classical wave form of the traveling wave,
	\begin{equation}
		v_p = \frac{1}{\sqrt{L_0C_0}}
	\end{equation}
	
	\subparagraph{Characteristic Impedance} The characteristic impedance of a pair of transmission lines can be described by the ratio of peak current to peak voltage. Given the solutions to the differential wave equations, (the $+$ superscript denotes waves flowing in the positive direction)
	\begin{IEEEeqnarray}{rCl}
		V^+(x, t) & = & V_0^+\sin(kx - \omega t)\\
		I^+(x, t) & = & I_0^+\sin(kx - \omega t)	
	\end{IEEEeqnarray}
	substitution back into the wave equation and initial conditions shows that
	\begin{equation}
		Z_0 = \frac{V_0^+}{L_0^+} = \sqrt{\frac{L_0}{C_0}}
	\end{equation}
	
	If the transmission lines end in some device at the end with impedance $Z_L$ called \textbf{load impedance}, then just like the continuous string there is a transmitted wave and reflected wave, with voltage and current ratios as follows:
	\begin{center}
	\begin{circuitikz}[scale=2]
	\draw (0, 1) to[L] node[above] {$V^+, I^+ \rightarrow$} (2, 1) -- (3, 1) to[generic] node[right] {$Z_L$} (3, 0) -- node[below] {$\leftarrow V^-, I^-$} (0, 0);
	\draw (2, 1) to[C] (2, 0);
	\end{circuitikz}
	\end{center}
	
	\begin{IEEEeqnarray}{rCl}
		\frac{V^-}{V^+} & = & \frac{Z_L - Z_0}{Z_L + Z_0}\\
		\frac{V_L}{V_t} & = & \frac{2Z_L}{Z_L + Z_0}
	\end{IEEEeqnarray}
	
	\begin{IEEEeqnarray}{rCl}
		\frac{I^-}{I^+} & = & \frac{Z_0 - Z_L}{Z_L + Z_0}\\
		\frac{I_L}{I_t} & = & \frac{2Z_0}{Z_L + Z_0}
	\end{IEEEeqnarray}
	
%	\begin{center}
%	\begin{tikzpicture}
%		[scale=3,line cap=round,
%		%Styles
%		axes/.style=,
%		important line/.style={very thick},
%		information text/.style={rounded corners,fill=red!10,inner sep=1ex},
%		dot/.style={circle,inner sep=1pt,fill,label={#1},name=#1}			
%		]
%		
%		%Colors
%		\colorlet{anglecolor}{green!50!black}	%angle arcs/lines
%		
%		%The graphic
%	\end{tikzpicture}
%	\end{center}

%	\begin{figure}[htb]
%		\centering
%		\includegraphics[width=0.8\textwidth]{filename.eps}
%		\caption{Caption.}
%		\label{fig:figure}
%	\end{figure}

%		\def\enotesize{\normalsize}
%		\theendnotes
\end{document}