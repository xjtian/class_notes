\documentclass[11pt]{article}
\usepackage{amsmath, amssymb, amsthm}
\usepackage[retainorgcmds]{IEEEtrantools}

\usepackage[pdftex]{graphicx}
\usepackage{tikz}
\usetikzlibrary{intersections}

\usepackage{marginnote}
\usepackage{endnotes}

\usepackage{fancyhdr}

%Listings stuff
\usepackage{listings}
\usepackage{lstautogobble}
\usepackage{color}

\definecolor{gray}{rgb}{0.5,0.5,0.5}
\lstset{
basicstyle={\small\ttfamily},
tabsize=3,
numbers=left,
numbersep=5pt,
numberstyle=\tiny\color{gray},
stepnumber=2,
breaklines=true
}

%Properly formatted differential 'd'
\newcommand{\ud}{\, \mathrm{d}}

%Format stuff
\pagestyle{fancy}
\headheight 35pt

%Header info
\chead{\Large \textbf{Simple Harmonic Oscillators}}
\lhead{}
\rhead{}

\begin{document}
\section{Motion}
	Consider a simple harmonic oscillator (SHO) like a horizontal mass-spring system. Force is given by Hooke's Law, $F=-kx$ where $x$ is the displacement from equilibrium.
	\begin{IEEEeqnarray}{rCl}
		F & = & -kx\\
		ma & = & -kx\\
		m\ddot{x} & = & -kx
	\end{IEEEeqnarray}
	The general equations of motion for a SHO, after solving for the second-order differential equation above, are as follows.
	\begin{IEEEeqnarray}{rCl}
		0  & = & \ddot{x} + \frac{k}{m}x\\
		x(t) & = & A\cos(\omega t + \delta)
	\end{IEEEeqnarray}
	Where $A$ is amplitude, in meters, given by the initial position, $\omega$ is angular frequence, in Hz, and $\delta$ is phase, given by the initial velocity, in radians. Plugging the general solution into the differential equation, we can determine that
	\begin{equation}
		\omega = \sqrt{\frac{k}{m}}
	\end{equation}
	
\section{Energy}
	Straightforward equations for potential and kinetic energy by displacement:
	\begin{IEEEeqnarray}{rCl}
		U(x) & = & \frac{1}{2} kx^2\\
		KE(x) & = & \frac{1}{2}m\dot{x}^2
	\end{IEEEeqnarray}
	
	Substituting for $x$ to express these equations by time:
	\begin{IEEEeqnarray}{rCl}
		U(t) & = & \frac{1}{2}k[A\cos(\omega t + \delta)]^2\\
		KE(t) & = & \frac{1}{2}m[-A\omega \sin(\omega t + \delta)]^2
	\end{IEEEeqnarray}
	
	After adding the above two equations together, the total energy of the system is then
	\begin{equation}
		\sum E = \frac{1}{2} kA^2
	\end{equation}

%	\begin{center}
%	\begin{tikzpicture}
%		[scale=3,line cap=round,
%		%Styles
%		axes/.style=,
%		important line/.style={very thick},
%		information text/.style={rounded corners,fill=red!10,inner sep=1ex},
%		dot/.style={circle,inner sep=1pt,fill,label={#1},name=#1}			
%		]
%		
%		%Colors
%		\colorlet{anglecolor}{green!50!black}	%angle arcs/lines
%		
%		%The graphic
%	\end{tikzpicture}
%	\end{center}

%	\begin{figure}[htb]
%		\centering
%		\includegraphics[width=0.8\textwidth]{filename.eps}
%		\caption{Caption.}
%		\label{fig:figure}
%	\end{figure}

%		\def\enotesize{\normalsize}
%		\theendnotes
\end{document}