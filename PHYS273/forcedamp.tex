\documentclass[11pt]{article}
\usepackage{amsmath, amssymb, amsthm}
\usepackage[retainorgcmds]{IEEEtrantools}

\usepackage[pdftex]{graphicx}
\usepackage{tikz}
\usetikzlibrary{intersections}

\usepackage{fancyhdr}

%Listings stuff
\usepackage{listings}
\usepackage{lstautogobble}
\usepackage{color}

\definecolor{gray}{rgb}{0.5,0.5,0.5}
\lstset{
basicstyle={\small\ttfamily},
tabsize=3,
numbers=left,
numbersep=5pt,
numberstyle=\tiny\color{gray},
stepnumber=2,
breaklines=true,
boxpos=t
}

%Format stuff
\pagestyle{fancy}
\headheight 35pt

%Header info
\chead{\Large \textbf{Forced Oscillators with Damping}}
\lhead{}
\rhead{}

\begin{document}
\section{Long-Term Solution}
	The equation of motion for the forced oscillator with damping is 
	\begin{equation}
		\ddot{x} + \gamma\dot{x} + \omega_0^2 x = \frac{F_0}{m} e^{i\omega_f t}
	\end{equation}
	
	The long-term solution for this differential equation is
	\begin{IEEEeqnarray}{rCl}
		x(t) & = & Ae^{i\omega_f t + \delta}\\
		\delta & = & -\tan^{-1} \left( \frac{\omega_f \gamma}{\omega_0^2 - \omega_f^2}\right)\\
		A & = & \frac{F_0 / m}{\sqrt{(\omega_f \gamma)^2 + (\omega_0^2 + \omega_f^2)^2}}
	\end{IEEEeqnarray}
	
	Note that in this solution, there are no arbitrary constants; the long-term equation of motion does not depend on the initial conditions.
	
\section{Transient Behavior}
	To describe the behavior of the system in a short-term period, note that $x_f + x_d$ is also a solution to the differential equation.
	\begin{equation}
		x_f(t) = A\omega_f e^{i(\omega_f t + \delta_f)}
	\end{equation}
	\begin{equation}
		x_d(t) = Be^{-\gamma t / 2}e^{i(\omega_d t + \delta_d)}
	\end{equation}
	Where they are the solutions to the forced oscillator and dampened oscillator, respectively.
	\begin{equation}
		x(t) = A\omega_f e^{i(\omega_f t + \delta_f)} + Be^{-\gamma t / 2}e^{i(\omega_d t + \delta_d)}
	\end{equation}
	
	In this case, $A$ and $\delta_f$ are determined the same way as in the long-term solution. Now, $B$ and $\delta_d$ are two free parameters in the solution that can be determined by the initial conditions. Note that because of the $-\gamma t / 2$ exponent in the second term, it approaches 0 as $t$ grows, in other words the transient behavior dies out and leaves just the long-term behavior.

%	\begin{center}
%	\begin{tikzpicture}
%		[scale=3,line cap=round,
%		%Styles
%		axes/.style=,
%		important line/.style={very thick},
%		information text/.style={rounded corners,fill=red!10,inner sep=1ex},
%		dot/.style={circle,inner sep=1pt,fill,label={#1},name=#1}			
%		]
%		
%		%Colors
%		\colorlet{anglecolor}{green!50!black}	%angle arcs/lines
%		
%		%The graphic
%	\end{tikzpicture}
%	\end{center}

%	\begin{figure}[htb]
%		\centering
%		\includegraphics[width=0.8\textwidth]{filename.eps}
%		\caption{Caption.}
%		\label{fig:figure}
%	\end{figure}

%		\def\enotesize{\normalsize}
%		\theendnotes
\end{document}