\documentclass[11pt]{article}
\usepackage{amsmath, amssymb, amsthm}
\usepackage[retainorgcmds]{IEEEtrantools}

\usepackage[pdftex]{graphicx}
\usepackage{tikz}
\usetikzlibrary{intersections}

\usepackage{marginnote}
\usepackage{endnotes}

\usepackage{fancyhdr}

%Listings stuff
\usepackage{listings}
\usepackage{lstautogobble}
\usepackage{color}

\definecolor{gray}{rgb}{0.5,0.5,0.5}
\lstset{
basicstyle={\small\ttfamily},
tabsize=3,
numbers=left,
numbersep=5pt,
numberstyle=\tiny\color{gray},
stepnumber=2,
breaklines=true
}

%Properly formatted differential 'd'
\newcommand{\ud}{\, \mathrm{d}}

%Format stuff
\pagestyle{fancy}
\headheight 35pt

%Header info
\chead{\Large \textbf{Sets}}
\lhead{}
\rhead{}

\begin{document}
\section{Discrete Structures}
	``Discrete'' objects:
	\begin{itemize}
		\item Are \textit{enumerable}, so they can be counted.
		\item May (often) be composed to create new structures.
		\item Effectively and naturally capture a spectrum of behaviors essential to all computing.
%		\item Allow for a certain type of constructive reasoning.
	\end{itemize}
	
\section{Definitions and Properties}
	Informally, a \textbf{set} is a collection of \textit{unique} objects called ``elements'' or ``points''. Sets are named with single capital letters, and are notated as such: $S=\{1,2,3\}$, or $S = \{n, \text{where n is an even integer}\}$.
	
	The \textbf{cardinality} of a set is the number of elements it contains and is a property universal to all sets. The following holds true for all sets:
	\begin{equation}
		\forall \text{ sets } S, |S|=m, \text{ where } m\leq N
	\end{equation}
	
	The \textbf{empty set} is the set containing no elements, notated as $\{\}$ or $\emptyset$.
	
	\subsection{Containment}
		Sets allow us to describe \textbf{containment relations} between collections of objects. $\in$ indicates an object is contained within a set, and $\notin$ means the opposite.
		
		A \textbf{subset} relation is defined as follows: The set $S$ is a subset of $T$ iff every element in $S$ is contained within $T$, notated as $S\subseteq T$. $S$ is a \textbf{proper subset} of $T$ iff $S$ is contained in $T$ but $T$ has at least one element not in $S$, notated as $S\subset T$.
		
		A \textbf{superset} relation is defined as such: The set $S$ is a superset iff every element of $T$ is in $S$, notated $S\supset T$. Subset and superset are \textbf{duals}.
		
\section{Constructions Within Sets}
	The \textbf{Cartesian product} of two sets $S$ and $T$, written $S\times T$ and read ``$S$ cross $T$'' is defined as follows.
	\begin{equation}
		S\times T = \{(s,t)\mid s\in S \text{ and } t\in T\}
	\end{equation}

%	\begin{center}
%	\begin{tikzpicture}
%		[scale=3,line cap=round,
%		%Styles
%		axes/.style=,
%		important line/.style={very thick},
%		information text/.style={rounded corners,fill=red!10,inner sep=1ex},
%		dot/.style={circle,inner sep=1pt,fill,label={#1},name=#1}			
%		]
%		
%		%Colors
%		\colorlet{anglecolor}{green!50!black}	%angle arcs/lines
%		
%		%The graphic
%	\end{tikzpicture}
%	\end{center}

%	\begin{figure}[htb]
%		\centering
%		\includegraphics[width=0.8\textwidth]{filename.eps}
%		\caption{Caption.}
%		\label{fig:figure}
%	\end{figure}

%		\def\enotesize{\normalsize}
%		\theendnotes
\end{document}