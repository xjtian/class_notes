\documentclass[11pt]{article}
\usepackage{amsmath, amssymb, amsthm}
\usepackage[retainorgcmds]{IEEEtrantools}

\usepackage[pdftex]{graphicx}
\usepackage{tikz}
\usetikzlibrary{intersections}

\usepackage{marginnote}
\usepackage{endnotes}

\usepackage{fancyhdr}

%Listings stuff
\usepackage{listings}
\usepackage{lstautogobble}
\usepackage{color}

\definecolor{gray}{rgb}{0.5,0.5,0.5}
\lstset{
basicstyle={\small\ttfamily},
tabsize=3,
numbers=left,
numbersep=5pt,
numberstyle=\tiny\color{gray},
stepnumber=2,
breaklines=true
}

%Properly formatted differential 'd'
\newcommand{\ud}{\, \mathrm{d}}

%Format stuff
\pagestyle{fancy}
\headheight 35pt

%Header info
\chead{\Large \textbf{Sets}}
\lhead{}
\rhead{}

\begin{document}
\section{Discrete Structures}
	``Discrete'' objects:
	\begin{itemize}
		\item Are \textit{enumerable}, so they can be counted.
		\item May (often) be composed to create new structures.
		\item Effectively and naturally capture a spectrum of behaviors essential to all computing.
		\item Allow for a certain type of constructive reasoning.
	\end{itemize}
	
	Mathematical structures can be abstracted to a single structure, called a ``semi-group'', the combination of a set and an operator. For example, $(\mathbb{Z}, +)$ means that the operator '+' is closed over the set of all integers ($\mathbb{Z}$). Other basic sets are $\mathbb{Q} \text{ and } \mathbb{N}$, the set of all rational and natural numbers, respectively.
	
\section{Definitions and Properties}
	Modern mathematics (post-1900) is characterized by a critical re-evaluation of mathematical certainty, with a focus on structural and axiomatic approach to the subject.
	
	Informally, a \textbf{set} is a collection of \textit{unique} objects called ``elements'' or ``points''. Sets are named with single capital letters, and are notated as such: $S=\{1,2,3\}$, or $S = \{n, \text{where n is an even integer}\}$.
	
	The \textbf{cardinality} of a set is the number of elements it contains and is a property universal to all sets. The following holds true for all sets:
	\begin{equation}
		\forall \text{ sets } S, |S|=m, \text{ where } m\leq N
	\end{equation}
	
	The \textbf{empty set} is the set containing no elements, notated as $\{\}$ or $\emptyset$.

%	\begin{center}
%	\begin{tikzpicture}
%		[scale=3,line cap=round,
%		%Styles
%		axes/.style=,
%		important line/.style={very thick},
%		information text/.style={rounded corners,fill=red!10,inner sep=1ex},
%		dot/.style={circle,inner sep=1pt,fill,label={#1},name=#1}			
%		]
%		
%		%Colors
%		\colorlet{anglecolor}{green!50!black}	%angle arcs/lines
%		
%		%The graphic
%	\end{tikzpicture}
%	\end{center}

%	\begin{figure}[htb]
%		\centering
%		\includegraphics[width=0.8\textwidth]{filename.eps}
%		\caption{Caption.}
%		\label{fig:figure}
%	\end{figure}

%		\def\enotesize{\normalsize}
%		\theendnotes
\end{document}