\documentclass[11pt]{article}
\usepackage{amsmath, amssymb, amsthm}
\usepackage[retainorgcmds]{IEEEtrantools}

\usepackage[pdftex]{graphicx}
\usepackage{tikz}
\usetikzlibrary{intersections}

\usepackage{marginnote}
\usepackage{endnotes}

\usepackage{fancyhdr}

%Listings stuff
\usepackage{listings}
\usepackage{lstautogobble}
\usepackage{color}

\definecolor{gray}{rgb}{0.5,0.5,0.5}
\lstset{
basicstyle={\small\ttfamily},
tabsize=3,
numbers=left,
numbersep=5pt,
numberstyle=\tiny\color{gray},
stepnumber=2,
breaklines=true
}

%Properly formatted differential 'd'
\newcommand{\ud}{\, \mathrm{d}}

%Format stuff
\pagestyle{fancy}
\headheight 35pt

%Header info
\chead{\Large \textbf{Induction}}
\lhead{}
\rhead{}

\begin{document}
\section{Mathematical Induction}
	Let $P(n)$ be a property that is defined for integers or natural numbers $n$ with a base case at $a$. To prove $\forall n \geq a, P(n)$ by mathematical induction, prove the following true:
	\begin{equation}
		P(a) \wedge ( P(k) \rightarrow P(k + 1) )
	\end{equation}
	There are two steps: the basis step, to prove the base case, and then the \textbf{inductive step}, which shows that supposing that $P(k)$ is true (supposition called the \textbf{inductive hypothesis}) then $P(k + 1)$ is also true.
	
\section{Strong Induction}
	Let $P(n)$ be a property that is defined for integers or natural numbers $n$ and $a, b \in \mathbb{Z} \wedge a \leq b$. To prove $\forall n \geq a, P(n)$ by strong induction, prove the following true:
	\begin{equation}
		(\forall i\in[a, b] \mid P(i)) \wedge \forall k \geq b, (\forall i \in [a, k] \mid P(i)) \rightarrow P(k+1)
	\end{equation}
	In strong induction, prove (possibly) multiple base cases in the basis step and the inductive hypothesis becomes the supposition that $P(i)$ for all integers $i$ from $a$ to $k$.

\section{Second-Order Recurrences}
	A second-order linear homogeneous recurrence relation with constant coefficients is a recurrence relation of the form
	\begin{equation}
		a_k = A a_{k-1} + B a_{k-2} \quad \forall k \geq n, B \neq 0
	\end{equation}
	This kind of recurrence is satisfied by the sequence
	\begin{equation}
		1, t, t^2, \ldots, t^n, \ldots,
	\end{equation}
	where $t$ is z nonzero real number, if and only if $t$ satisfies the \textbf{characteristic equation} of the relation,
	\begin{equation}
		t^2 - At - B = 0
	\end{equation}
	
	\subparagraph{Distinct-Roots Theorem} Given a second-order homogeneous linear recurrence for all $k \geq 2$, if the characteristic equation has two distinct roots $r$ and $s$, then the recurrence is given by 
	\begin{equation}
		a_n = Cr^n + Ds^n
	\end{equation}
	where $C$ and $D$ are determined by plugging in the values $a_0$ and $a_1$.
	
	\subparagraph{Single-Root theorem} If the characteristic equation has a single real root $r$, then the explicit formula is given by
	\begin{equation}
		a_n = Cr^n + Dnr^n
	\end{equation}
	
\section{General Recursion and Structural Induction}
	\subparagraph{Recursively Defined Sets} To define a set recursively, there are 3 necessary components:
	\begin{enumerate}
		\item Base: A statement that certain objects belong to the set.
		\item Recursion: A collection of rules indicating how to form new set objects from those already known to be in the set.
		\item Restriction: A statement that no objects belong to the set other than those coming from 1 and 2.
	\end{enumerate}
	Given a finite non-empty set $S$, a \textbf{string over S} is a finite sequence of elements called \textbf{characters}. The \textbf{null string} is defined to be the string with no characters, denoted by $\epsilon$.
	
	\subparagraph{Structural Induction} to prove that every object in $S$ satisfies some property, prove all the base cases then show that the objects obtained from the recursion satisfy the property if all inputs do.
%	\begin{center}
%	\begin{tikzpicture}
%		[scale=3,line cap=round,
%		%Styles
%		axes/.style=,
%		important line/.style={very thick},
%		information text/.style={rounded corners,fill=red!10,inner sep=1ex},
%		dot/.style={circle,inner sep=1pt,fill,label={#1},name=#1}			
%		]
%		
%		%Colors
%		\colorlet{anglecolor}{green!50!black}	%angle arcs/lines
%		
%		%The graphic
%	\end{tikzpicture}
%	\end{center}

%	\begin{figure}[htb]
%		\centering
%		\includegraphics[width=0.8\textwidth]{filename.eps}
%		\caption{Caption.}
%		\label{fig:figure}
%	\end{figure}

%		\def\enotesize{\normalsize}
%		\theendnotes
\end{document}