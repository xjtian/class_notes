\documentclass[11pt]{article}
\usepackage{amsmath, amssymb, amsthm}
\usepackage[retainorgcmds]{IEEEtrantools}

\usepackage[pdftex]{graphicx}
\usepackage{tikz}
\usetikzlibrary{intersections}

\usepackage{marginnote}
\usepackage{endnotes}

\usepackage{fancyhdr}

%Listings stuff
\usepackage{listings}
\usepackage{lstautogobble}
\usepackage{color}

\definecolor{gray}{rgb}{0.5,0.5,0.5}
\lstset{
basicstyle={\small\ttfamily},
tabsize=3,
numbers=left,
numbersep=5pt,
numberstyle=\tiny\color{gray},
stepnumber=2,
breaklines=true
}

%Properly formatted differential 'd'
\newcommand{\ud}{\, \mathrm{d}}

%Format stuff
\pagestyle{fancy}
\headheight 35pt

%Header info
\chead{\Large \textbf{Binary Relations and Functions}}
\lhead{}
\rhead{}

\begin{document}
\section{Binary Relations}
	A binary relation $R$ is a subset of $A\times B$. It is any arbitrary way of selecting ordered pairs from the Cartesian product of two sets.
	\begin{equation}
		R\subseteq A\times B
	\end{equation}
	
	One way of ``ordering'' a set is through a binary relation defined over that set.
	\begin{equation}
		R\subseteq A\times A
	\end{equation}
	
	\subsection{Equivalence Relations}
		An equivalence relation is a binary relation that enforces reflexivity, transitivity, and symmetry, notated with $\equiv$.
		\begin{itemize}
			\item Reflexivity: $\forall a\in A, a\equiv a$
			\item Transitivity: $\forall a,b,c\in A:a\equiv b \text{ and } b\equiv c \text{ implies } a\equiv c$
			\item Symmetry: $\forall a,b\in A: a\equiv b \text{ implies } b\equiv a$
		\end{itemize}
		
\section{Functions}
	A \textbf{function} is a \textit{unitary} mapping from one set, called the domain, to another set, called the codomain. A \textbf{unitary mapping} is one in which each point in the domain is mapped to one in the codomain.
	
	Functions describe transformations among sets and are the building blocks of more complex relations. The image of the codomain that a function maps to is called the \textbf{range}. A function must map every domain element to only one codomain element.
	
	How many functions exist mapping a domain $S$ of size $n$ and a codomain $T$ of size $m$?

%	\begin{center}
%	\begin{tikzpicture}
%		[scale=3,line cap=round,
%		%Styles
%		axes/.style=,
%		important line/.style={very thick},
%		information text/.style={rounded corners,fill=red!10,inner sep=1ex},
%		dot/.style={circle,inner sep=1pt,fill,label={#1},name=#1}			
%		]
%		
%		%Colors
%		\colorlet{anglecolor}{green!50!black}	%angle arcs/lines
%		
%		%The graphic
%	\end{tikzpicture}
%	\end{center}

%	\begin{figure}[htb]
%		\centering
%		\includegraphics[width=0.8\textwidth]{filename.eps}
%		\caption{Caption.}
%		\label{fig:figure}
%	\end{figure}

%		\def\enotesize{\normalsize}
%		\theendnotes
\end{document}