\documentclass[11pt]{article}
\usepackage{amsmath, amssymb, amsthm}
\usepackage[retainorgcmds]{IEEEtrantools}

\usepackage[pdftex]{graphicx}
\usepackage{tikz}
\usetikzlibrary{intersections}

\usepackage{marginnote}
\usepackage{endnotes}

\usepackage{fancyhdr}

%Listings stuff
\usepackage{listings}
\usepackage{lstautogobble}
\usepackage{color}

\definecolor{gray}{rgb}{0.5,0.5,0.5}
\lstset{
basicstyle={\small\ttfamily},
tabsize=3,
numbers=left,
numbersep=5pt,
numberstyle=\tiny\color{gray},
stepnumber=2,
breaklines=true
}

%Properly formatted differential 'd'
\newcommand{\ud}{\, \mathrm{d}}

%Format stuff
\pagestyle{fancy}
\headheight 35pt

%Header info
\chead{\Large \textbf{Predicate Calculus}}
\lhead{}
\rhead{}

\begin{document}
\section{Predicate Translation}
	A student of mine is wearing a blue shirt.
	\begin{itemize}
		\item Domain: all people $P$
		\item Quantification: There is at least one (existential quantifier)
		\item Predicates: "wearing a blue shirt" and "is my student"
	\end{itemize}
	Let $B(x)$ represent ``$x$ is wearing a blue shirt'' and $S(x)$ represent ``$x$ is my student''.
	\begin{equation}
		(\exists s\in P):[B(x)\wedge S(x)]
	\end{equation}
	
	All good students are in class.
	\begin{itemize}
		\item Domain: all people $P$
		\item Quantification: all of them (universal quantifier)
		\item Predicates: "are in class" and "is a good student"
	\end{itemize}
	In this case, due to the universal quantifier, use a logical implication that goes from least specific to most specific.
	\begin{equation}
		(\forall s\in P):[G(s)\rightarrow C(s)]
	\end{equation}
	
	\subsection{Negation}
		\begin{IEEEeqnarray}{rCl}
			(\forall s\in X:[P(s)]) & \equiv & P(s_1)\wedge P(s_2)\wedge\ldots\wedge P(s_n)\\
			\lnot\forall s\in S:[P(s)] &\equiv & \exists s_i\in S:\lnot[P(s)]
		\end{IEEEeqnarray}
		\begin{IEEEeqnarray}{rCl}
			(\exists s\in X:[P(s)]) & \equiv & P(s_1)\vee P(s_2)\vee\ldots\vee P(s_n)\\
			\lnot\exists s\in S:[P(s)] & \equiv & \forall s\in S:\lnot[P(s)]
		\end{IEEEeqnarray}

%	\begin{center}
%	\begin{tikzpicture}
%		[scale=3,line cap=round,
%		%Styles
%		axes/.style=,
%		important line/.style={very thick},
%		information text/.style={rounded corners,fill=red!10,inner sep=1ex},
%		dot/.style={circle,inner sep=1pt,fill,label={#1},name=#1}			
%		]
%		
%		%Colors
%		\colorlet{anglecolor}{green!50!black}	%angle arcs/lines
%		
%		%The graphic
%	\end{tikzpicture}
%	\end{center}

%	\begin{figure}[htb]
%		\centering
%		\includegraphics[width=0.8\textwidth]{filename.eps}
%		\caption{Caption.}
%		\label{fig:figure}
%	\end{figure}

%		\def\enotesize{\normalsize}
%		\theendnotes
\end{document}