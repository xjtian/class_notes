\documentclass[11pt]{article}
\usepackage{amsmath, amssymb, amsthm}

\usepackage{fancyhdr}

%Format stuff
\pagestyle{fancy}
\headheight 35pt

%Header info
\chead{\Large \textbf{Quantified Statements}}
\lhead{}
\rhead{}

\begin{document}
\section{Predicates and Quantifiers}
	A \textbf{predicate} is a sentence that contains a finite number of variables and becomes a statement when specific values are substituted for the variables. The \textbf{domain} of a predicate variable is the set of all values that may be substituted in place of the variable.
	
	The \textbf{truth set} of a predicate $P(x)$ is the set of all elements in the domain $D$ that make $P(X)$ true: $\{x\in D\mid P(x)\}$.
	
	\subsection{Quantifiers}
		\textbf{Quantifiers} tell for how many elements a given predicate is true. The \textbf{universal quantifier} is the symbol $\forall$ and translates to ``for all'' or ``any''. The \textbf{existential quantifier} is the symbol $\exists$ and translates to ``there exists''.
	
	\subsection{Negation}
		The negation of a universal statement is an existential statement and vice versa. When negating multiply-quantified statements ($\forall\exists$ or $\exists\forall$), the same principle still applies.
		
\section{Arguments with Quantifiers}
	The rule of \textbf{universal instantiation} says that if some property is true of everything in a set, then it is true of any particular thing in the set. This gives rise to two basic argument forms, Modus Ponens (left) and Modus Tollens (right). 
	
	\begin{center}
	\begin{tabular}{ll}
	
		\begin{tabular}{rl}
			& $\forall x, P(x)\rightarrow Q(x)$\\
			& $\exists a\in x\mid P(a)$\\
			$\therefore$ & $Q(a)$
		\end{tabular}
		
		&
		
		\begin{tabular}{rl}
			& $\forall x, P(x)\rightarrow Q(x)$\\
			& $\exists a\in x\mid \lnot Q(a)$\\
			$\therefore$ & $\lnot P(a)$
		\end{tabular}
	
	\end{tabular}
	\end{center}
	
\section{Proof Techniques}
	\begin{description}
		\item[Proof by exhaustion:] show that the predicate is true for all elements in the domain.
		\item[Proof from generic particular:] if all elements in the domain share properties, then take one arbitrary element from the domain and show the predicate is true.
		\item[Proof by contraposition:] prove the contrapositive true. Sometimes this is easier because the negation of a universal is an existential.
		\item[Proof by contradiction:] assume that the statement is false, then show that this assumption leads to a contradiction.
	\end{description}

%	\begin{center}
%	\begin{tikzpicture}
%		[scale=3,line cap=round,
%		%Styles
%		axes/.style=,
%		important line/.style={very thick},
%		information text/.style={rounded corners,fill=red!10,inner sep=1ex},
%		dot/.style={circle,inner sep=1pt,fill,label={#1},name=#1}			
%		]
%		
%		%Colors
%		\colorlet{anglecolor}{green!50!black}	%angle arcs/lines
%		
%		%The graphic
%	\end{tikzpicture}
%	\end{center}

%	\begin{figure}[htb]
%		\centering
%		\includegraphics[width=0.8\textwidth]{filename.eps}
%		\caption{Caption.}
%		\label{fig:figure}
%	\end{figure}

%		\def\enotesize{\normalsize}
%		\theendnotes
\end{document}