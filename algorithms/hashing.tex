\documentclass[11pt]{article}
\usepackage{amsmath, amssymb, amsthm}

\usepackage{fancyhdr}

%Format stuff
\pagestyle{fancy}
\headheight 35pt

%Header info
\chead{\Large \textbf{Hashing}}
\lhead{}
\rhead{}

\begin{document}
\section{Hash Tables}
	Given a universe $U$ of objects, hash tables maintain an evolving set $S\subset U$ by keeping $n$ buckets, where $n\approx |S|$. To map objects to buckets, they use a hash function $H: U \rightarrow {0, 1, 2, \ldots , n-1}$ and an array of length $n$, storing an object $x$ in \verb|A[h(x)]|.
	
	\subsection{Two-Sum Problem}
		\begin{description}
			\item[Input:] Array of integers \verb|A[1\ldots n]|
			\item[Output:] All pairs $(a, b)$ such that $a + b = t$.
		\end{description}
		
		Reducing the problem to sorting yields an $O(n\log n)$ solution - sort the array, then binary search for $t-x$ for each $x$. Using a hash table yields an $O(n)$ solution.
		
\section{Hashing and Pathological Data Sets}
	The \textbf{load} of a hash table is defined as 
	\begin{equation}
		\alpha = \frac{n}{|S|}
	\end{equation}
	$\alpha = O(1)$ is a necessary condition for constant-time hash table operations, and $\alpha << 1$ is necessary for open addressing when using a sequential hash function to deal with collisions. For good performance and efficiency with linked-list buckets, keep $\alpha$ close to 1. 
	
	A \textbf{pathological data set} is a data set that isolates one or very few buckets in a hash table to reduce it to linear efficiency, and exists for every hash function. Pathological data sets can paralyze real-world systems, and there are usually two ways to prevent this exploit:
	\begin{enumerate}
		\item Crytographic hashes are much more immune (practically completely) to pathological data sets (SHA-2, MD5, etc)
		\item Hash randomization: pick a single hash function $h$ in a family $H$ of hash functions at runtime.
	\end{enumerate}
%	\begin{center}
%	\begin{tikzpicture}
%		[scale=3,line cap=round,
%		%Styles
%		axes/.style=,
%		important line/.style={very thick},
%		information text/.style={rounded corners,fill=red!10,inner sep=1ex},
%		dot/.style={circle,inner sep=1pt,fill,label={#1},name=#1}			
%		]
%		
%		%Colors
%		\colorlet{anglecolor}{green!50!black}	%angle arcs/lines
%		
%		%The graphic
%	\end{tikzpicture}
%	\end{center}

%	\begin{figure}[htb]
%		\centering
%		\includegraphics[width=0.8\textwidth]{filename.eps}
%		\caption{Caption.}
%		\label{fig:figure}
%	\end{figure}

%		\def\enotesize{\normalsize}
%		\theendnotes
\end{document}