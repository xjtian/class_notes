\documentclass[11pt]{article}
\usepackage{amsmath, amssymb, amsthm}
\usepackage[retainorgcmds]{IEEEtrantools}

\usepackage{marginnote}
\usepackage{endnotes}

\usepackage{fancyhdr}

%Listings stuff
\usepackage{listings}
\usepackage{lstautogobble}
\usepackage{color}

\definecolor{gray}{rgb}{0.5,0.5,0.5}
\lstset{
basicstyle={\small\ttfamily},
tabsize=3,
numbers=left,
numbersep=5pt,
numberstyle=\tiny\color{gray},
stepnumber=2,
breaklines=true
}

%Properly formatted differential 'd'
\newcommand{\ud}{\, \mathrm{d}}

%Format stuff
\pagestyle{fancy}
\headheight 35pt

%Header info
\chead{\Large \textbf{Analysis of Algorithms}}
\lhead{}
\rhead{}

\begin{document}
\section{Guiding Principles}
	\begin{enumerate}
		\item \marginnote{What are the three guiding principles?} Worst-Case Analysis: analyze algorithms by the upper-bound for \emph{all} inputs, as opposed to average-case or benchmarks, which require domain knowledge.
		\item Disregard constant factors and lower-order terms.
		\begin{itemize}
			\item These factors depend on architecture, compiler, etc$\ldots$
			\marginnote{Why can constant factors be disregarded?}
			\item Very little predictive power is lost
		\end{itemize}
		\item Asymptotic Analysis: focus on running-time for very large inputs.
	\end{enumerate}
	
\section{Asymptotic Analysis}
	\textbf{Big-Oh Notation}\marginnote{What are Big-Oh, Omega, and Theta Notation?}
	\begin{itemize}
		\item $T(n) = O(f(n))$ if $T(n)$ is bound above by constant multiple of $f(n)$ for very large $n$.
		\item $T(n) = O(f(n))$ if only $\exists c$ in $n_0 > 0$ such that $T(n) \leq c\cdot f(n); n\geq n_0$.
	\end{itemize}
	\textbf{Omega Notation}
	\begin{itemize}
		\item $T(n) = \Omega (f(n))$ if $\exists c$ in $n_0 > 0$ such that $T(n) \geq c\cdot f(n), \forall n\geq n_0$.
		\item $T(n)$ is lower-bound by $f(n)$.
	\end{itemize}
	\textbf{Theta Notation}
	\begin{itemize}
		\item $T(n) = \Theta (f(n))$ if $T(n) = O(f(n))$ and $T(n)=\Omega (f(n))$.
		\item $T(n)$ is exactly bound by $f(n)$.
	\end{itemize}
	\textbf{Little-Oh Notation}
	\begin{itemize}
		\item $T(n) = o(f(n))$ if $\forall c > 0, \exists c$ such that $T(n) \leq c\cdot f(n), \forall n\geq n_0$
	\end{itemize}
	
	\line(1,0){250}
	\endnotetext[1]{Worst-case asymptotic analysis with Big-Oh or Theta Notation.}
	\endnotetext[2]{Big-Oh: bounded above by $f(n)$. Omega: bounded below by $f(n)$. Theta: bounded exactly by $f(n)$.}
	\def\enotesize{\normalsize}
	\theendnotes
\end{document}