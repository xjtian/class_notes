\documentclass[11pt]{article}
\usepackage{amsmath, amssymb, amsthm}
\usepackage[retainorgcmds]{IEEEtrantools}

\usepackage{fancyhdr}

%Format stuff
\pagestyle{fancy}
\headheight 35pt

%Header info
\chead{\Large \textbf{Algorithmic Analysis}}
\lhead{}
\rhead{}

\begin{document}
	Generally, it's OK to disregard lower-order terms when doing worst-case analysis on algorithmic complexity. For asymptotically large input values, dropping these terms loses very little predictive power. The following notations are ways to describe the asymptotic behavior of functions, or how a function grows when input size $n$ becomes very large. Big-Oh and Big-Theta notation are most commonly used.
	
	\begin{IEEEeqnarray}{rCl}
		\Theta(g(n)) & = & \{ f(n) \mid \exists (c_1, c_2, n_0) \in \mathbb{R}^+ \mid\\\nonumber 
		&& \forall n \geq n_0, \quad c_1g(n) \leq f(n) \leq c_2g(n) \}\\
		O(g(n)) & = & \{f(n) \mid \exists(c, n_0) \in \mathbb{R}^+ \mid \forall n \geq n_0, \quad f(n) \leq c_1 g(n) \}\\
		\Omega(g(n)) & = & \{f(n) \mid \exists(c, n_0) \in \mathbb{R}^+ \mid \forall n \geq n_0, \quad cg(n) \leq f(n) \}\\
		o(g(n) & = & \left\lbrace f(n) \mid \lim_{n\rightarrow \infty} \frac{f(n)}{g(n)} = 0 \right\rbrace\\
		\omega(g(n)) & = & \left\lbrace f(n) \mid \lim_{n\rightarrow \infty} \frac{f(n)}{g(n)} = \infty \right\rbrace
	\end{IEEEeqnarray}
	
\end{document}