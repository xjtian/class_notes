\documentclass[11pt]{article}
\usepackage{amsmath, amssymb, amsthm}

\usepackage{fancyhdr}
%Format stuff
\pagestyle{fancy}
\headheight 35pt

%Header info
\chead{\Large \textbf{Minimal Spanning Trees}}
\lhead{}
\rhead{}

\begin{document}
\begin{description}
	\item[Informal goal:] connect a set of points as cheaply as possible
	\item[Applications:] clustering, networks
\end{description}

The are multiple greedy algorithms for computing the MST of a given graph, most notably Prim's (1957) and Kruskal's algorithm (1956), both of which achieve $O(m\log n)$ time using suitable data structures.

\section{Problem Definition}
	\begin{description}
		\item[Input:] undirected graph $G=(V,E)$ and a cost $c_e$ for each $e\in E$. (An ``optimal branching problem'' exists as an analog to the MST problem for directed graphs).
		\item[Assume:] adjacency list, all edge costs are distinct (algorithms remain correct if ties exist, but proof is trickier), $G$ is connected (check in preprocessing).
		\item[Output:] minimum cost tree $T\in E$ that spans all vertices such that $T$ has no cycles and is a fully connected subgraph (path exists to/from all vertices).
	\end{description}

%	\begin{center}
%	\begin{tikzpicture}
%		[scale=3,line cap=round,
%		%Styles
%		axes/.style=,
%		important line/.style={very thick},
%		information text/.style={rounded corners,fill=red!10,inner sep=1ex},
%		dot/.style={circle,inner sep=1pt,fill,label={#1},name=#1}			
%		]
%		
%		%Colors
%		\colorlet{anglecolor}{green!50!black}	%angle arcs/lines
%		
%		%The graphic
%	\end{tikzpicture}
%	\end{center}

%	\begin{figure}[htb]
%		\centering
%		\includegraphics[width=0.8\textwidth]{filename.eps}
%		\caption{Caption.}
%		\label{fig:figure}
%	\end{figure}

%		\def\enotesize{\normalsize}
%		\theendnotes
\end{document}