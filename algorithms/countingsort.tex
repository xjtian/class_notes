\documentclass[11pt]{article}
\usepackage{amsmath, amssymb, amsthm}
\usepackage[retainorgcmds]{IEEEtrantools}

\usepackage[pdftex]{graphicx}
\usepackage{tikz}
\usepackage{circuitikz}
\usetikzlibrary{intersections}

\usepackage{fancyhdr}

%Listings stuff
\usepackage{listings}
\usepackage{lstautogobble}
\usepackage{color}

\definecolor{gray}{rgb}{0.5,0.5,0.5}
\lstset{
basicstyle={\small\ttfamily},
tabsize=3,
numbers=left,
numbersep=5pt,
numberstyle=\tiny\color{gray},
stepnumber=2,
breaklines=true,
boxpos=t
}

%Format stuff
\pagestyle{fancy}
\headheight 35pt

%Header info
\chead{\Large \textbf{Counting  Sort}}
\lhead{}
\rhead{}

\begin{document}
For Counting Sort, the input is $n$ integers in some known range $[0, k-1]$.

\begin{lstlisting}[autogobble=true]
	def CountingSort:
		C = [0]*k, B = [0] * n
		for j in [1, n]:
			C[A[j]] += 1	# C = distribution
		for i in [1, k]:
			C[i] = C[i-1] + C[i]	# C = Cumulative distribution
		for j in [n, 1]:
			B[C[A[j]]] = A[j]
			C[A[j]]--
		return B
\end{lstlisting}

The algorithm calculates a cumulative frequency distribution for the numbers in the input array in the first 2 passes, so that \verb|C[i]| is the number of elements less than or equal to \verb|i|. In the final loop, the algorithm puts the last element from \verb|A| into \verb|B| based on its position within the cumulative distribution, then recurses (iteratively) on \verb|A[:-1]|.

Counting sort has $\Theta(n + k)$ runtime and space complexity and is a stable sort. For obvious reasons, it should only be preferred if $k << n\lg n$.

%	\begin{center}
%	\begin{tikzpicture}
%		[scale=3,line cap=round,
%		%Styles
%		axes/.style=,
%		important line/.style={very thick},
%		information text/.style={rounded corners,fill=red!10,inner sep=1ex},
%		dot/.style={circle,inner sep=1pt,fill,label={#1},name=#1}			
%		]
%		
%		%Colors
%		\colorlet{anglecolor}{green!50!black}	%angle arcs/lines
%		
%		%The graphic
%	\end{tikzpicture}
%	\end{center}

%	\begin{figure}[htb]
%		\centering
%		\includegraphics[width=0.8\textwidth]{filename.eps}
%		\caption{Caption.}
%		\label{fig:figure}
%	\end{figure}

%		\def\enotesize{\normalsize}
%		\theendnotes
\end{document}