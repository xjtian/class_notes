\documentclass[11pt]{article}
\usepackage{amsmath, amssymb, amsthm}

\usepackage{fancyhdr}

%Listings stuff
\usepackage{listings}
\usepackage{lstautogobble}
\usepackage{color}

\definecolor{gray}{rgb}{0.5,0.5,0.5}
\lstset{
basicstyle={\small\ttfamily},
tabsize=3,
numbers=left,
numbersep=5pt,
numberstyle=\tiny\color{gray},
stepnumber=2,
breaklines=true
}

%Format stuff
\pagestyle{fancy}
\headheight 35pt

%Header info
\chead{\Large \textbf{Dynamic Connectivity}}
\lhead{}
\rhead{}

\begin{document}
\section{Dynamic Connectivity}

	Given a set of N objects,
	\begin{itemize}
		\item Union command: connect two objects
		\item Find/connected query: is there a path connecting the two objects?
	\end{itemize}
	Connections are equivalency relations:
	\begin{itemize}
		\item Reflexive: p connected to p
		\item Symmetric: p connected to q, q connected to p
		\item Transitive: p $\rightarrow$ q $\rightarrow$ r, then p $\rightarrow$ r
	\end{itemize}
	\textbf{Connected component:} maximal \emph{set} of objects that are mutually connected
	
	\subsection{Algorithm}
	
		Goal: Design efficient data structure for union-find
		\begin{itemize}
			\item Number of objects $N$ can be huge
			\item Number of operations $M$ can be huge
			\item Find queries and union commands may be intermixed
		\end{itemize}
		\textbf{Union-find API:}
		\begin{lstlisting}[autogobble=true]
			public class UF {
				UF(int N)	//initialize with N objects
				
				//add connection b/w p and q
				void union(int p, int q)
				
				//are p and q connected?
				boolean connected(int p, int q)
				
				int find(int p)	//component identifier for p
				int count()	//number of components
			}
		\end{lstlisting}
		\textbf{Dynamic-connectivity client:}
		\begin{lstlisting}[autogobble=true]
			Read in N objects from standard input.
			Loop:
				Read in pair of integers from standard input
				If not connected, connect and print out pair
		\end{lstlisting}
		
	\subsection{Quick-Find}
	
		Data structure:
		\begin{itemize}
			\item Integer array \verb|id[]| of size N that tracks connected component 	ID for every object
			\item Interpretation: p and q are connected if the have the same id
			\item Union: To merge p and q, change all entries with \verb|id == id[p]| to \verb|id[q]|
		\end{itemize}
		Problem: plain union command is O(n) complexity for a single command.
		
	\subsection{Quick-Union}
	
		Data structure:
		\begin{itemize}
			\item Integer array \verb|id[]| of size N
			\item Interpretation: \verb|id[i]| is parent of i
			\item \textbf{Root} of i is \verb|id[id[id[...id[i]...]]]|
		\end{itemize}
		\begin{description}
			\item[Find:] check if p and q have equivalent roots
			\item[Union:] Set \verb|id[p]| to \verb|root(id[q])|
		\end{description}
		Root is a component with \verb|id[i] == i|: is parent to itself
		\begin{lstlisting}[autogobble=true]
			root(i):
				while (i != id[i]) i = id[i];
				return i;
		\end{lstlisting}
		Problem: O(n) worst-case for find \emph{and} union due to complexity of root-finding
	
	\subsection{Improvements to Quick-Union}
		\subsubsection{Weighting}
			\begin{itemize}
				\item Modify quick-union to avoid tall trees
				\item Keep track of size of each tree
				\item Balance by linking root of smaller tree to root of larger tree
			\end{itemize}
			Maintain extra array \verb|sz[i]| to count number of objects at tree rooted at i
			\begin{lstlisting}[autogobble=true]
				Union(p, q):
					i = root(p); j = root(q);
					if sz[i] < sz[j]:
						id[i] = j; sz[j] += sz[i];
					else:
						id[j] = i; sz[i] += sz[j];
			\end{lstlisting}
			Find and union are both O($\log n$)
			
			\begin{proof}
			Find operation is O($\log n$)
				\begin{enumerate}
					\item Depth of x increases by 1 every merge.
					\item Merge means tree $T_1$ containing x at least doubles since $|T_2| \geq |T_1|$
					\item Size of tree containing x can double at most $\log n$ times
				\end{enumerate}
			\end{proof}
		
		\subsubsection{Path Compression}
		
			Just after computing the root of p, set the id of each examined node to point to that root.
			
			Simpler one-pass variant: make every other node in the path point to its grandparent (halves path length)
			
			\begin{lstlisting}[autogobble=true]
				root(i):
					while i != id[i]:
						id[i] = id[id[i]];
						i = id[i];
					return i;
			\end{lstlisting}
			
		\subsubsection{Complexity}
			No algorithm that is completely linear, although weighted, single-pass compressed algorithm is very close in practice.
			
			\begin{center}\begin{tabular}[t]{c c}
			\hline
			\textbf{Algorithm}	&	\textbf{Worst-case Time}	\\\hline
			Quick-Find	&	$MN$	\\
			Quick-Union	&	$MN$	\\\hline
			Weighted QU	&	$N+M\log N$	\\
			QU + Compression	&	$N + M\log N$	\\
			Weighted QU + Compression	&	$N+M\log^* N$\\\hline
			\end{tabular}\end{center}
			
\section{Applications of Union-Find}
	\begin{itemize}
		\item Percolation
		\item Dynamic connectivity
		\item Least common ancestor
		\item Equivalence of FSA
		\item Kruskal's minimum spanning tree algorithm
		\item Hinley-Milner polymorphic type inference
	\end{itemize}
	
	\subsection{Percolation}
		A model for many physical systems:
		\begin{itemize}
			\item $N\times N$ grid of sites
			\item Each site is open with probability $p$
			\item System \textbf{percolates} if top and bottom are connected by open sites
		\end{itemize}
		
		Likelihood of percolation depends strongly on the value of p. When $N$ is large, theory guarantees a sharp threshold $p^*$ such that $p > p^*$ almost certainly percolates and $p < p^*$ almost certainly does not percolate.
		
		It is possible to model this $p^*$ with a Monte Carlo simulation:
		\begin{enumerate}
			\item Initialize $N\times N$ grid to be blocked
			\item Declare random sites open until connected to bottom
			\item Vacancy percentage estimates $p^*$
		\end{enumerate}
		
		To check whether an $N\times N$ system percolates, we initialize a list of structures for each site labeled $0$ to $N^2 - 1$. Sites are in the same component if they are connected by open sites, and the system percolates if any site on the bottom row is connected to any site on the top row.	
		
		Clever trick: Introduce 2 virtual sites, initialize them to be connected to the entire top row and bottom row, respectively. Then the system percolates if the virtual top is connected to the virtual bottom.
		
		To open a site, connect it to all neighboring open sites (quick-union algorithm).
		
		With this simulation, we determine $p^* \approx 0.592746$.
\end{document}