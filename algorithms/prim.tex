\documentclass[11pt]{article}
\usepackage{amsmath, amssymb, amsthm}
\usepackage[retainorgcmds]{IEEEtrantools}
\usepackage{fancyhdr}

%Listings stuff
\usepackage{listings}
\usepackage{lstautogobble}
\usepackage{color}

\definecolor{gray}{rgb}{0.5,0.5,0.5}
\lstset{
basicstyle={\small\ttfamily},
tabsize=3,
numbers=left,
numbersep=5pt,
numberstyle=\tiny\color{gray},
stepnumber=2,
breaklines=true,
boxpos=t
}

%Format stuff
\pagestyle{fancy}
\headheight 35pt

%Header info
\chead{\Large \textbf{Prim's Algorithm}}
\lhead{}
\rhead{}

\begin{document}

	
\section{Algorithm}
	Build the MST by adding leaves one at a time from an arbitrary source vertex. Uses a similar greedy criterion to Dijkstra's, but this one only keys vertices by the weight of the lightest edge coming out from the partial MST.

	\begin{lstlisting}[autogobble=true,mathescape]
		def primMST(G=(V,E), s):
			initialize heap Q with all vertices to $\infty$, s to 0
			pred[s] = null
			while |Q| > 0:
				u = extract-min(Q)
				for (u, v) in E:
					if v undiscovered and w(u, v) < Q[v]:
						update v in Q with w(u, v)
						pred[v] = u
					
					mark u discovered
	\end{lstlisting}
	
	\verb|pred| defines the MST as an inverted tree rooted at $s$.
	
\section{Proof}
	At any time, the graph is partitioned into a cut $(S, V-S)$ where $S$ is the set of all processed vertices. By the definition of the greedy criterion, vertices are added to $S$ by choosing the cheapest edge that crosses this cut. By the safe edge lemma, the algorithm always adds safe edges and therefore correctly computes an MST.
	
\section{Runtime}
	Extractions from the heap are $\Theta(\log n)$ each, while updating the neighbors of an extracted node in the heap is an $O(degree(u) \cdot \log n)$ operation.
	
	\begin{IEEEeqnarray}{rCl}
		T(n, m) & = & \sum_{u \in V} \Big( \log n + degree(u) \cdot \log n \Big)\\
		& = & \log n \sum_{u \in V} (1 + degree(u))\\
		& = & \log n \cdot (n + 2m) \in \Theta(m\log n)
	\end{IEEEeqnarray}

%	\begin{center}
%	\begin{tikzpicture}
%		[scale=3,line cap=round,
%		%Styles
%		axes/.style=,
%		important line/.style={very thick},
%		information text/.style={rounded corners,fill=red!10,inner sep=1ex},
%		dot/.style={circle,inner sep=1pt,fill,label={#1},name=#1}			
%		]
%		
%		%Colors
%		\colorlet{anglecolor}{green!50!black}	%angle arcs/lines
%		
%		%The graphic
%	\end{tikzpicture}
%	\end{center}

%	\begin{figure}[htb]
%		\centering
%		\includegraphics[width=0.8\textwidth]{filename.eps}
%		\caption{Caption.}
%		\label{fig:figure}
%	\end{figure}

%		\def\enotesize{\normalsize}
%		\theendnotes
\end{document}