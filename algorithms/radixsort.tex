\documentclass[11pt]{article}
\usepackage{amsmath, amssymb, amsthm}
\usepackage[retainorgcmds]{IEEEtrantools}

\usepackage[pdftex]{graphicx}
\usepackage{tikz}
\usepackage{circuitikz}
\usetikzlibrary{intersections}

\usepackage{fancyhdr}

%Listings stuff
\usepackage{listings}
\usepackage{lstautogobble}
\usepackage{color}

\definecolor{gray}{rgb}{0.5,0.5,0.5}
\lstset{
basicstyle={\small\ttfamily},
tabsize=3,
numbers=left,
numbersep=5pt,
numberstyle=\tiny\color{gray},
stepnumber=2,
breaklines=true,
boxpos=t
}

%Format stuff
\pagestyle{fancy}
\headheight 35pt

%Header info
\chead{\Large \textbf{Radix Sort}}
\lhead{}
\rhead{}

\begin{document}
Radix sort sorts numbers by sequential digit given an input of $n$ numbers with a maximum of $d$ digits.
\begin{lstlisting}[autogobble=true]
	def RadixSort:
		for i in [0, d-1]:
			StableSort(A, key=digit i)
\end{lstlisting}
A good sort to use to implement Radix sort is Counting sort because it is stable and we can pick the radix $r$ of the numbers to optimize the efficiency of the algorithm. The runtime complexity of Radix sort for radix $r$ is $\Theta(\log_r S (n+r))$, where $S$ is the range of values in the input array ($\log_r S = d$).

\subparagraph{Optimal Radix} Take the derivative of the runtime complexity to optimize $r$.
\begin{IEEEeqnarray}{rCl}
	\frac{d}{dr}\left( \frac{c\ln S}{\ln r} (n + r) \right) & = & 0\\
	c\ln S\left( \frac{\ln r - n/r - 1}{(\ln r)^2} \right) & = & 0
\end{IEEEeqnarray}
The nontrivial solution to this equation is
\begin{equation}
	r = \frac{n}{\ln r - 1}
\end{equation}
Guess $r \stackrel{?}{=} n$, which gives an inconsistent result
\begin{equation}
	r = \frac{n}{\ln n - 1}
\end{equation}
Substituting this value for $r$ and evaluating again yields
\begin{equation}
	r = \frac{n}{\ln n - \ln\ln n - 1}
\end{equation}
The trend continues, and because $\ln\ln n$ is almost negligible, it's enough to say that the best radix to pick for the sort is
\begin{equation}
	r = \frac{n}{\ln n - 1}
\end{equation}

Note that in practice, it's often best to pick a radix that is a power of 2 to take advantage of hardware. If a power of 2 is close to $n/(\ln n - 1)$, then it's the best base to pick.


%	\begin{center}
%	\begin{tikzpicture}
%		[scale=3,line cap=round,
%		%Styles
%		axes/.style=,
%		important line/.style={very thick},
%		information text/.style={rounded corners,fill=red!10,inner sep=1ex},
%		dot/.style={circle,inner sep=1pt,fill,label={#1},name=#1}			
%		]
%		
%		%Colors
%		\colorlet{anglecolor}{green!50!black}	%angle arcs/lines
%		
%		%The graphic
%	\end{tikzpicture}
%	\end{center}

%	\begin{figure}[htb]
%		\centering
%		\includegraphics[width=0.8\textwidth]{filename.eps}
%		\caption{Caption.}
%		\label{fig:figure}
%	\end{figure}

%		\def\enotesize{\normalsize}
%		\theendnotes
\end{document}