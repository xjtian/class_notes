\documentclass[11pt]{article}
\usepackage{amsmath, amssymb, amsthm}
\usepackage[retainorgcmds]{IEEEtrantools}

\usepackage{fancyhdr}

%Format stuff
\pagestyle{fancy}
\headheight 35pt

%Header info
\chead{\Large \textbf{Scheduling Applications}}
\lhead{}
\rhead{}

\begin{document}
\section{Problem Statement}
	\begin{description}
		\item[Input:] A list of jobs $j$ each with a weight $w_j$ and length $l_j$.
		\item[Output:] An ordering of jobs that minimizes the weighted sum of completion times, where completion time is the amount of wall time it takes from the beginning to the completion of job $j$.
		\[C_j = \sum_1^j l_j\]
		\[min(\sum_1^nw_jC_j)\]
	\end{description}
	
\section{Greedy Criterion}
	\subsection{Special Cases}
		\subparagraph{Identical lengths} In this case, we choose the larger-weight jobs first to minimize the term in the sum (when $n$ is smaller).
			\begin{equation}
				\sum_1^n w_jC_j = \sum_1^n w_j \cdot n \cdot L = L \cdot \sum_1^n w_j \cdot n
			\end{equation}
		\subparagraph{Identical weights} In this case, we choose the shortest jobs first to minimize the sum over completion times.
			\begin{equation}
				\sum_1^n WC_j = W\sum_1^n C_j
			\end{equation}
			
	\subsection{Resolving Conflicting Advice}
		The key idea is to assign each job a score that is directly proportional with $w$ and indirectly proportional with $l$. Taking two guesses at such a score, there are 
		\begin{itemize}
			\item $s_j = w_j - l_j$
			\item $s_j = w_j / l_j$
		\end{itemize}
		To distinguish between the two, find the simplest example that produces different outputs. In this case, for the two jobs $j_1(w=3, l=5), j_2(w=1,l=2)$, the second score yields the lower weighted sum, so it is better than the first.
		
\section{Proof of Correctness}
	\begin{description}
		\item[Claim:] sorting by decreasing ratio $w_j / l_j$ is always correct
		\item[Proof:] by exchange argument and contradiction.
		\item[Assume:] no ties in ratios (doesn't change algorithm, only proof)
	\end{description}
	
	Fix an arbitrary input of $n$ jobs with $\sigma$ as the greedy schedule from the algorithm and $\sigma^*$ being a theoretically optimal algorithm. Next, rename the jobs $j_1, j_2, \ldots , j_n$ such that $s_1 > s_2 > \ldots > s_n$, so that $\sigma$ just involves ordering the jobs in numerical order.
	
	Assuming $\sigma^* \neq \sigma$, there must exist two consecutive jobs $(i, j) \in \sigma^* \mid i > j$. Because $\sigma$ is a unique schedule (properly ordered), the only way $\sigma^*$ can be different is if at least one pair of jobs is flipped.
	
	Suppose we exchange the order of these two jobs in $\sigma^*$, leaving all other jobs unchanged. Jobs other than these two have no impact on completion time after this operation, while $C_i$ increases and $C_j$ decreases.
	
	\begin{IEEEeqnarray}{rCl}
		i > j & \rightarrow & \frac{w_i}{l_i} < \frac{w_j}{l_j}\\
		&& w_i l_j < w_j l _i
	\end{IEEEeqnarray}
	
	From the definition of weighted sum, it's clear that the cost associated with this exchange is outweighed by the benefit. Therefore, it is impossible for $\sigma^*$ to be optimal, violating out assumption in the first step, so $\sigma$ is an optimal schedule.
%	\begin{center}
%	\begin{tikzpicture}
%		[scale=3,line cap=round,
%		%Styles
%		axes/.style=,
%		important line/.style={very thick},
%		information text/.style={rounded corners,fill=red!10,inner sep=1ex},
%		dot/.style={circle,inner sep=1pt,fill,label={#1},name=#1}			
%		]
%		
%		%Colors
%		\colorlet{anglecolor}{green!50!black}	%angle arcs/lines
%		
%		%The graphic
%	\end{tikzpicture}
%	\end{center}

%	\begin{figure}[htb]
%		\centering
%		\includegraphics[width=0.8\textwidth]{filename.eps}
%		\caption{Caption.}
%		\label{fig:figure}
%	\end{figure}

%		\def\enotesize{\normalsize}
%		\theendnotes
\end{document}