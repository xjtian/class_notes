\documentclass[11pt]{article}
\usepackage{amsmath, amssymb, amsthm}

\usepackage{fancyhdr}

%Listings stuff
\usepackage{listings}
\usepackage{lstautogobble}
\usepackage{color}

\definecolor{gray}{rgb}{0.5,0.5,0.5}
\lstset{
basicstyle={\small\ttfamily},
tabsize=3,
numbers=left,
numbersep=5pt,
numberstyle=\tiny\color{gray},
stepnumber=2,
breaklines=true
}

%Format stuff
\pagestyle{fancy}
\headheight 35pt

%Header info
\chead{\Large \textbf{Graph Search}}
\lhead{}
\rhead{}

\begin{document}
\section{Generic Search}
	\begin{lstlisting}[autogobble=true]
		def GenericSearch(G, start_vertex):
			mark all G unexplored, S explored
			while unexplored exist:
				choose edge (u, v) with u explored, v unexplored
				mark v explored
	\end{lstlisting}
	
	The difference in how the two most significant graph search algorithms, \textbf{breadth-first search} and \textbf{depth-first search} work is how each chooses the next node to explore.
	
	BFS explores nodes in "layers", exploring all of a vertex's neighbors before moving another level away. DFS explores aggressively, going as deep as possible before backtracking if necessary. BFS is often implemented with a queue (FIFO) structure, while DFS is implemented with a stack (FILO) or recursively. Both algorithms are $O(m+n)$.
	
\section{BFS}
	\begin{lstlisting}[autogobble=true]
		def BFS(G, start):
			mark start explored
			enqueue(start)
			
			while !empty(queue):
				v = dequeue()
				mark v explored
				for (v, w) in G:
					if w unexplored:
						enqueue(w)
	\end{lstlisting}
	
	Runtime is actually $O(m_s + n_s)$, where $s$ means the edges and nodes reachable from the starting vertex.
	
	\subsection{Shortest Path Problem}
		Given an undirected graph with equally-weighted edges, it is simple to adapt BFS to compute the shortest possible path between two vertices. Initialize all vertices with distance $\infty$ except the starting vertex with distance 0, and conduct BFS except if \verb|w| is unexplored, then \verb|dist(w) = dist(v) + 1|.
		
	\subsection{Undirected Connectivity}
		A connected component is an undirected subgraph where any two vertices are connected. They are the "pieces" of an undirected grpah, and there does not exist a path between two connected components. BFS can be adapted to find these subgraphs.
		\begin{lstlisting}[autogobble=true]
			def FindConnections(G):
				for V in G:
					if (V unexplored):
						BFS(G, V)	//Finds the connected vertices
		\end{lstlisting}
		
\section{DFS}
	\begin{lstlisting}[autogobble=true]
		def DFS(G, start):
			start explored
			for (s, v) in G:
				if v unexplored:
					mark v explored
					DFS(G, v)
	\end{lstlisting}
	
	It is also possible to implement DFS with a stack if the recursive version busts the call stack.
	
	\subsection{Topological Sort}
		\textbf{Topological ordering} of a directed graph $G$ is a labeling function $f$ of $G$'s nodes such that
		\begin{equation}
			\forall (u,v)\in G, f(u) < f(v)
		\end{equation}
		Note that this is only possible if there are no cycles present in $G$, as topological sorting turns $G$ into a \textbf{directed acyclic graph}, or DAG.
		
		\begin{center}
		\begin{tabular}{ll}
			\begin{lstlisting}[autogobble=true]
				def DFS-Loop(G):
					label = n
					for v in G:
						if v unexplored:
							DFS_Top(G, v)
			\end{lstlisting} 
			& 
			\begin{lstlisting}[autogobble=true]
				def DFS_Top(G, v):
					v explored
					for (v, w) in G:
						if w unexplored:
							DFS_Topology(G, w)
					f(start) = label--
			\end{lstlisting}
		\end{tabular}
		\end{center}
		
		If we order the graph by increasing $f(v)$, then $G$ is clearly a DAG. This algorithm runs in $O(m+n)$ time if there are no cycles in $G$.

%	\begin{center}
%	\begin{tikzpicture}
%		[scale=3,line cap=round,
%		%Styles
%		axes/.style=,
%		important line/.style={very thick},
%		information text/.style={rounded corners,fill=red!10,inner sep=1ex},
%		dot/.style={circle,inner sep=1pt,fill,label={#1},name=#1}			
%		]
%		
%		%Colors
%		\colorlet{anglecolor}{green!50!black}	%angle arcs/lines
%		
%		%The graphic
%	\end{tikzpicture}
%	\end{center}

%	\begin{figure}[htb]
%		\centering
%		\includegraphics[width=0.8\textwidth]{filename.eps}
%		\caption{Caption.}
%		\label{fig:figure}
%	\end{figure}

%		\def\enotesize{\normalsize}
%		\theendnotes
\end{document}