\documentclass[11pt]{article}
\usepackage{amsmath, amssymb, amsthm}
\usepackage[retainorgcmds]{IEEEtrantools}

\usepackage[pdftex]{graphicx}
\usepackage{tikz}
\usetikzlibrary{intersections}

\usepackage{marginnote}
\usepackage{endnotes}

\usepackage{fancyhdr}

%Listings stuff
\usepackage{listings}
\usepackage{lstautogobble}
\usepackage{color}

\definecolor{gray}{rgb}{0.5,0.5,0.5}
\lstset{
basicstyle={\small\ttfamily},
tabsize=3,
numbers=left,
numbersep=5pt,
numberstyle=\tiny\color{gray},
stepnumber=2,
breaklines=true
}

%Properly formatted differential 'd'
\newcommand{\ud}{\, \mathrm{d}}

%Format stuff
\pagestyle{fancy}
\headheight 35pt

%Header info
\chead{\Large \textbf{Graphs}}
\lhead{}
\rhead{}

\begin{document}
\section{Basics}
	\marginnote{Graph cut?}A graph is a collection of \textbf{nodes} connected by \textbf{vertices} which may or may not have direction and weight. A \textbf{cut} of a graph is a partition of the set of vertices $V$ into non-empty sets $A$ and $B$. When discussing cuts, a \marginnote{Crossing edge?}\textbf{crossing edge} is an edge with a tail in one set and head in the other.
	
\section{Representation}
	\marginnote{Sparse/dense?}A graph is considered \textbf{sparse} if the number of edges $m$ is $O(n^2)$. A \textbf{dense} graph has $m$ on the scale of $O(n)$. In most applications, $m$ is $\Omega(n)$ and $O(n^2)$.
	
	\marginnote{Adj. matrix?}A graph can be represented with an \textbf{adjacency matrix}, an $n\times n$ matrix $A$ in which $A_{ij} \neq 0$ means $G$ has an $ij$ edge. $A_{ij}$ can also represent the number of $ij$ edges, the weight of the $ij$ edge, or the direction of the edge.
	
	\marginnote{Adj. list?}A more popular way of representing graphs is by an \textbf{adjacency list}. In this case, the graph is represented as an array of vertices and an array of edges. Often, each vertex has an array of other incident vertices which represents connections.

%	\begin{center}
%	\begin{tikzpicture}
%		[scale=3,line cap=round,
%		%Styles
%		axes/.style=,
%		important line/.style={very thick},
%		information text/.style={rounded corners,fill=red!10,inner sep=1ex},
%		dot/.style={circle,inner sep=1pt,fill,label={#1},name=#1}			
%		]
%		
%		%Colors
%		\colorlet{anglecolor}{green!50!black}	%angle arcs/lines
%		
%		%The graphic
%	\end{tikzpicture}
%	\end{center}

%	\begin{figure}[htb]
%		\centering
%		\includegraphics[width=0.8\textwidth]{filename.eps}
%		\caption{Caption.}
%		\label{fig:figure}
%	\end{figure}

%		\def\enotesize{\normalsize}
%		\theendnotes
\end{document}