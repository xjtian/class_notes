\documentclass[11pt]{article}
\usepackage{amsmath, amssymb, amsthm}
\usepackage[retainorgcmds]{IEEEtrantools}

\usepackage[pdftex]{graphicx}
\usepackage{tikz}
\usetikzlibrary{intersections}

\usepackage{fancyhdr}

%Listings stuff
\usepackage{listings}
\usepackage{lstautogobble}
\usepackage{color}

\definecolor{gray}{rgb}{0.5,0.5,0.5}
\lstset{
basicstyle={\small\ttfamily},
tabsize=3,
numbers=left,
numbersep=5pt,
numberstyle=\tiny\color{gray},
stepnumber=2,
breaklines=true
}

%Format stuff
\pagestyle{fancy}
\headheight 35pt

%Header info
\chead{\Large \textbf{Bubblesort}}
\lhead{}
\rhead{}

\begin{document}
\section{Algorithm}
	"Bubble" the largest value in the array to the top by a series of exchanges, then recurse (so to say) on the subarray \verb|A[:-1]|.

	\begin{lstlisting}[autogobble=true]
		for i in [n, 2]:
			for j in [1, i-1]:
				if A[j] > A[j+1]:
					A[j], A[j+1] = A[j+1], A[j]
	\end{lstlisting}
	
\section{Analysis}
	Bubblesort is $O(n^2)$ in its best, worst, and average cases.
	\subsection{Comparisons}
		\begin{equation}
			\sum_{i=2}^n \sum_{j=1}^{i-1} 1 = \sum_{i=2}^{n-1} i = \frac{n(n-1)}{2}
		\end{equation}
		
	\subsection{Exchanges}
		\subparagraph{Best case} The best case for bubblesort is a sorted array, which means $0$ exchanges.
		\subparagraph{Worst case} The worst case for bubble sort is a reverse-sorted array, which means every element goes through the maximum number of exchanges (equal to the number of comparisons).
			\begin{equation}
				\frac{n(n-1)}{2}
			\end{equation}
		\subparagraph{Average case} Because the number of exchanges to conduct is equal to the number of transpositions in the array (reverse-sorted pairs) and all permutations are equally likely, picking a random permutation would expect to yield $max/2$ transpositions.
			\begin{equation}
				\frac{n(n-1)}{4}
			\end{equation}

%	\begin{center}
%	\begin{tikzpicture}
%		[scale=3,line cap=round,
%		%Styles
%		axes/.style=,
%		important line/.style={very thick},
%		information text/.style={rounded corners,fill=red!10,inner sep=1ex},
%		dot/.style={circle,inner sep=1pt,fill,label={#1},name=#1}			
%		]
%		
%		%Colors
%		\colorlet{anglecolor}{green!50!black}	%angle arcs/lines
%		
%		%The graphic
%	\end{tikzpicture}
%	\end{center}

%	\begin{figure}[htb]
%		\centering
%		\includegraphics[width=0.8\textwidth]{filename.eps}
%		\caption{Caption.}
%		\label{fig:figure}
%	\end{figure}

%		\def\enotesize{\normalsize}
%		\theendnotes
\end{document}