\documentclass[11pt]{article}
\usepackage{amsmath, amssymb, amsthm}
\usepackage[retainorgcmds]{IEEEtrantools}

\usepackage{marginnote}
\usepackage{endnotes}

\usepackage{fancyhdr}

%Listings stuff
\usepackage{listings}
\usepackage{lstautogobble}
\usepackage{color}

\definecolor{gray}{rgb}{0.5,0.5,0.5}
\lstset{
basicstyle={\small\ttfamily},
tabsize=3,
numbers=left,
numbersep=5pt,
numberstyle=\tiny\color{gray},
stepnumber=2,
breaklines=true
}

%Properly formatted differential 'd'
\newcommand{\ud}{\, \mathrm{d}}

%Format stuff
\pagestyle{fancy}
\headheight 35pt

%Header info
\chead{\Large \textbf{Simple Divide and Conquer Algorithms}}
\lhead{}
\rhead{}

\begin{document}
\section{Counting Inversions}
	\textbf{Inversion:} a pair indices \verb|i, j| such that \verb|i < j| and \verb|A[i] > A[j]|.
	\marginnote{How is mergesort used to count inversions?}
	Possible to solve problem in $O(n\log n)$ time by using the Mergesort algorithm:
	\begin{lstlisting}[autogobble=true]
		count_and_split(A):
			if n == 1: return A, 0
			else:
				(B, x) = count_and_split(left)
				(C, y) = count_and_split(right)
				(D, z) = merge_and_count(B, C)
				
				return D, x+y+z
				
		merge_and_count(B, C):
			i = j = inversions = 0
			D = []
			while:
				if B[i] < C[j]:
					D += B[i++]
				else:
					D += C[j++]
					inversions += (len(B) - i)
			return D, inversions
	\end{lstlisting}
				
\section{Strassen's Subcubic Matrix Multiplication}
	Traditional matrix multiplication is done as such:
	\marginnote{Why is Strassen's algorithm faster?}
	\[ X = \left( \begin{array}{cc}
	a & b\\
	c & d	
	\end{array} \right),
	Y = \left( \begin{array}{cc}
	e & f\\
	g & h
	\end{array} \right);
	X\cdot Y = Z = \left( \begin{array}{cc}
	ae+bg & af+bh\\
	ce+dg & cf + dh
	\end{array} \right) \]
	
	The na\"{i}ve implementation of this algorithm requires 8 recursive calls and runs in $O(n^3)$ time. Strassen's algorithm computes the product of two square matrices with 7 recursive calls and accomplishes a subcubic running time that is still polynomial.
	
	The 7 products $p_1 \ldots p_7$ are:
	\begin{IEEEeqnarray*}{rCl}
		p_1 & = & a(f-h)\\
		p_2 & = & (a+b)h\\
		p_3 & = & (c+d)e\\
		p_4 & = & d(g-e)\\
		p_5 & = & (a+d)(e+h)\\
		p_6 & = & (b-d)(g+h)\\
		p_7 & = & (a-c)(e+f)
	\end{IEEEeqnarray*}
	
	And the final product is:
	\[Z = \left(\begin{array}{cc}
		p_5+p_4-p_2+p_6 & p_1+p_2\\
		p_3+p_4 & p_3-p_7
	\end{array} \right)\]

	\endnotetext[1]{The merge step in Mergesort can be used to count inversions}
	\endnotetext[2]{Strassen's algorithm improves on na\"{i}ve matrix multiplication by removing 1 recursive call}
	\def\enotesize{\normalsize}
	\theendnotes
\end{document}