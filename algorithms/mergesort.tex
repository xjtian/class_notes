\documentclass[11pt]{article}
\usepackage{amsmath, amssymb}

\usepackage{fancyhdr}
\pagenumbering{gobble}

\usepackage{listings}
\usepackage{lstautogobble}
\usepackage{color}

\definecolor{gray}{rgb}{0.5,0.5,0.5}

\lstset{
basicstyle={\small\ttfamily},
tabsize=3,
numbers=left,
numbersep=5pt,
numberstyle=\tiny\color{gray},
stepnumber=2
}

%Properly formatted differential 'd'
\newcommand{\ud}{\, \mathrm{d}}

%Format stuff
%\reversemarginpar
\pagestyle{fancy}
\headheight 35pt

%Header info
\chead{\Large \textbf{Mergesort}}
\rhead{}

\begin{document}
\section{Algorithm}
	\begin{itemize}
		\itemsep=-2pt
		\item Input: array of $n$ numbers, unsorted.
		\item Assume: all elements distinct
	\end{itemize}
	
	\textbf{Approach:}
	\begin{lstlisting}
		Split array in half
		Recursively sort halves
		Merge halves and return
	\end{lstlisting}
	
	\textbf{Merge Subprocedure:}
	\begin{lstlisting}
		C = output array [n]
		A = left array [n/2]
		B = right array [n/2]
		
		i = counter for A
		j = counter for B
		
		while i and j are both inside:
			Put the smallest element in C
			Increment counter of array that element is from
		Append leftover elements in each array
	\end{lstlisting}
		
\section{Runtime}
	\begin{itemize}
		\itemsep=-2pt
		\item Merge: $O(n)$
		\item Mergesort: $O(n\log_2 n)$
	\end{itemize}
	
	\subsection{Analysis}
		\begin{itemize}
			\item Maximum number of levels is $\log_2 n$
			\item At any level $j$, there are $2^j$ subproblems
			\item At each level $j$, the size of the subproblem is $\dfrac{n}{2^j}$
			\item The work at level $j$ is:
				\begin{equation}
				2^j \times \frac{n}{2^j} = n
				\end{equation}
			\item $Work = Levels * work/level = n \cdot \log_2 n$
		\end{itemize}
		
\section{Practical Improvements}

	\begin{itemize}
		\item Use insertion sort for small subarrays, at a cutoff of $\approx 7$. Eliminates unecessary overhead, improves $\approx 20\%$.
		\item Stop if already sorted: is the biggest item in first half $\leq$ smallest item in second half?
	\end{itemize}
	
\section{Bottom-up Mergesort}
	Basic plan:
		\begin{enumerate}
			\item Pass through array, merging subarrays of size 1
			\item Repeat for size 2, 4, 8, 16, etc\ldots
		\end{enumerate}
\end{document}