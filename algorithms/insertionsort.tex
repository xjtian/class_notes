\documentclass[11pt]{article}
\usepackage{amsmath, amssymb, amsthm}
\usepackage[retainorgcmds]{IEEEtrantools}

\usepackage{fancyhdr}

%Listings stuff
\usepackage{listings}
\usepackage{lstautogobble}
\usepackage{color}

\definecolor{gray}{rgb}{0.5,0.5,0.5}
\lstset{
basicstyle={\small\ttfamily},
tabsize=3,
numbers=left,
numbersep=5pt,
numberstyle=\tiny\color{gray},
stepnumber=2,
breaklines=true,
boxpos=t
}

%Format stuff
\pagestyle{fancy}
\headheight 35pt

%Header info
\chead{\Large \textbf{Insertion Sort}}
\lhead{}
\rhead{}

\begin{document}
\section{Algorithm}
	The approach is to sort the first \verb|i-1| elements, then insert element \verb|i| into its proper location in that subarray. Repeat until the array is sorted.
	
	There are two versions of the algorithm: first one assumes a sentinel value \verb|A[0] = -inf|, and the second one doesn't use this sentinel.
	
	\subparagraph{}	%hack to add a little extra padding above the table.
	
	\begin{tabular}{ll}
		\begin{lstlisting}[autogobble=true]
			A[0] = -inf
			for i in [2, n]:
				t, j = A[i], i-1
				while A[j] > t:
					A[j+1] = A[j]
					j--
				A[j+1] = T
		\end{lstlisting}
		&
		\begin{lstlisting}[autogobble=true]
			for i in [2, n]:
				t, j = A[i], i-1
				while j > 0 and A[j] > t:
					A[j+1] = A[j]
					j--
				A[j+1] = t
		\end{lstlisting}
	\end{tabular}
	
\section{Analysis}
	\subsection{Comparisons}
		The following calculations are for the sentinel version. They can be redone for the non-sentinel version pretty simply by counting how many fewer operations are done every iteration. 
		
		\subparagraph{Best case} In the best case, $n-1$ comparisons are done because the while loop checks once every time the for loop runs.
		\subparagraph{Worst case} Because of the presence of the sentinel value, the worst case is a reverse-sorted array in which case there are $i$ comparisons done in each iteration of the for loop.
			\begin{equation}
				\sum_{i=2}^n i = \left(\sum_{i=1}^n i\right) - 1 = \frac{(n-1)(n+2)}{2}
			\end{equation}
		\subparagraph{Average case} Consider a snapshot of the algorithm at iteration $i,j$. \verb|t| can potentially be inserted into any of $i$ possible spots, and at the $j$\textsuperscript{th} iteration of the \verb|while|-loop there will have been $i - j + 1$ comparisons conducted.
			\begin{IEEEeqnarray}{rCl}
				\sum_{i=2}^n \sum_{j=1}^i \frac{1}{i}(i-j+1) & = & \sum_{i=2}^n\frac{1}{i} \cdot \sum_{j=1}^i (i-j+1)\\
				& = & \sum_{i=2}^n\frac{1}{i} \cdot \sum_{j=1}^i j\\
				& = & \sum_{i=2}^n \frac{1}{i} \cdot \frac{i(i+1)}{2}\\
				& = & \frac{1}{2} \sum_{i=2}^n (i+1)\\
				& = & \frac{(n-1)(n+4)}{4}
			\end{IEEEeqnarray}
			
	\subsection{Exchanges}
		\subparagraph{Best case} $2n-1$, with the extra $1$ being from moving the sentinel into position $0$ at the beginning of the algorithm.
		\subparagraph{Worst case} The worst case is a reverse-sorted array, which means there will be $i+1$ exchanges done every iteration.
		\begin{IEEEeqnarray}{rCl}
			1 + \sum_{i=2}^n (i + 1) & = & 1 + \sum_{i=1}^{n-1} (i + 2)\\
			& = & \frac{(n-1)(n+4)}{2} + 1
		\end{IEEEeqnarray}
		\subparagraph{Average case} It's not worth it to do this calculation.

%	\begin{center}
%	\begin{tikzpicture}
%		[scale=3,line cap=round,
%		%Styles
%		axes/.style=,
%		important line/.style={very thick},
%		information text/.style={rounded corners,fill=red!10,inner sep=1ex},
%		dot/.style={circle,inner sep=1pt,fill,label={#1},name=#1}			
%		]
%		
%		%Colors
%		\colorlet{anglecolor}{green!50!black}	%angle arcs/lines
%		
%		%The graphic
%	\end{tikzpicture}
%	\end{center}

%	\begin{figure}[htb]
%		\centering
%		\includegraphics[width=0.8\textwidth]{filename.eps}
%		\caption{Caption.}
%		\label{fig:figure}
%	\end{figure}

%		\def\enotesize{\normalsize}
%		\theendnotes
\end{document}