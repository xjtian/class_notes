\documentclass[11pt]{article}
\usepackage{amsmath, amssymb, amsthm}
\usepackage[retainorgcmds]{IEEEtrantools}

\usepackage[pdftex]{graphicx}
\usepackage{tikz}
\usetikzlibrary{intersections}

\usepackage{marginnote}
\usepackage{endnotes}

\usepackage{fancyhdr}

%Listings stuff
\usepackage{listings}
\usepackage{lstautogobble}
\usepackage{color}

\definecolor{gray}{rgb}{0.5,0.5,0.5}
\lstset{
basicstyle={\small\ttfamily},
tabsize=3,
numbers=left,
numbersep=5pt,
numberstyle=\tiny\color{gray},
stepnumber=2,
breaklines=true
}

%Properly formatted differential 'd'
\newcommand{\ud}{\, \mathrm{d}}

%Format stuff
\pagestyle{fancy}
\headheight 35pt

%Header info
\chead{\Large \textbf{Randomized Contraction}}
\lhead{}
\rhead{}

\begin{document}
\section{Introduction}
	\marginnote{Min Cut?}The \textbf{minimum cut} problem is defined as follows: Given an undirected graph $G=(V,E)$ possibly containing parallel edges, compute the cut with the fewest number of crossing edges, called the minimum cut.
	
\section{Algorithm}
	\marginnote{Algorithm?}\begin{lstlisting}[autogobble=true]
		def MinCut(G):
			while len(V) > 2:
				randomly select remaining (u,v) edge
				merge/contract into single vertex
				remove self loops
			return cut represented by remaining 2 vertices.
	\end{lstlisting}
	
	It is important to note that this is a randomized algorithm that has a non-negligible chance of failure - thus, it is necessary to run the algorithm many times with different RNG seeds to guarantee a correct result.
	
\section{Analysis}
	To analyze the success rate and runtime of the randomized contraction algorithm, we need to start with three givens. First, fix a graph $G=(V,E)$ with $n$ vertices and $m$ edges, with a min-cut $(A,B)$, and let $k$ be the number of crossing edges forming the set $F$. Two important observations arise immediately:
	\begin{enumerate}
		\item If we contract an edge in $F$, the algorithm will not return the correct result.
		\item If we only contract edges contained within $A$ and $B$, we will get the min-cut of the graph.
	\end{enumerate}
	
	Let $S_i$ represent the event that an edge in $F$ is contracted in iteration $i$. Observing that the degree of each vertex is at least $k$,\marginnote{$m\geq ?$}
	\begin{IEEEeqnarray}{rCl}
		\sum_v degree(v) & = & 2m\\
		\sum_v degree(v) & \geq & kn\\
		m &\geq & \frac{kn}{2}\\\nonumber\\
		Pr[S_1] & = & \frac{2}{n}
	\end{IEEEeqnarray}
	
	For the second iteration:
	\begin{IEEEeqnarray}{rCl}
		Pr[\lnot S_1\wedge\lnot S_2] & = & Pr[\lnot S_1 | \lnot S_2]\times Pr[\lnot S_1]\\
		Pr[\lnot S_1\wedge\lnot S_2] & \geq & 1 - \frac{k}{\text{remaining edges}}\times (1 - \frac{2}{n})
	\end{IEEEeqnarray}
	
	Remember that $m \geq \frac{kn}{2}$.
	\begin{equation}
		Pr[\lnot S_1 | \lnot S_2] = 1 - \frac{2}{n - 1}
	\end{equation}
	
	Now we can generalize for all iterations.
	\begin{IEEEeqnarray}{rCl}
		Pr[\lnot S_1\wedge\ldots\wedge\lnot S_{n-2}] & = & \left(1 - \frac{2}{n}\right)\left(1 - \frac{2}{n - 1}\right)\ldots\left(1-\frac{2}{n-(n-3)}\right)\\\nonumber\\
		& = & \frac{n-2}{n}\cdot\frac{n-3}{n-1}\cdot\frac{n-4}{n-2}\cdots\frac{2}{4}\cdot\frac{1}{3}\\\nonumber\\
		& = & \frac{2}{n(n-1)}\\\nonumber\\
		&\geq & \frac{1}{n^2}
	\end{IEEEeqnarray}
	
	\marginnote{Success rate?}Thus, the chance of success on a single attempt with randomized selection is very unlikely.
	
	\subsection{Repeated Trials}
		Due to the significant failure rate, finding the minimum cut actually involves running the algorithm many times and remembering the smallest cut. Let $T_i$ be the event that $(A,B)$ is found on the $i\textsuperscript{th}$ try and $N$ be the number of times the algorithm is run.
		\begin{IEEEeqnarray}{rCl}
			Pr[fail] & = & Pr[\lnot T_1\wedge\lnot T_2\wedge\ldots\wedge\lnot T_N]\\\nonumber\\
			& = & \prod_{i=1}^N Pr[\lnot T_i] \leq \left(1 - \frac{1}{n^2}\right)^N
		\end{IEEEeqnarray}
		
		\marginnote{How many trials?}Using the fact that $\forall x\in\mathbb{R}, 1+x\leq e^x$:
		\begin{itemize}
			\item For $N = n^2, Pr[fail]\leq e\left(e^{-\dfrac{1}{n^2}}\right)^{n^2} = \dfrac{1}{e} \approx 30\%$
			\item For $N=n^2\cdot\ln (n), Pr[fail]\leq\dfrac{1}{n}$
		\end{itemize}
		
		\marginnote{Runtime?}Runtime is therefore $\Omega(n^2 m)$ with slow constants.
		
\section*{Important Ideas}
	\begin{itemize}
		\item Contract random edges until graph has only 2 vertices, which possibly represent minimum cuts.
		\item Success rate is $\approx\frac{1}{n^2}$
		\item Running algorithm $n^2\cdot\ln (n)$ times almost guarantees success.
	\end{itemize}

%	\begin{center}
%	\begin{tikzpicture}
%		[scale=3,line cap=round,
%		%Styles
%		axes/.style=,
%		important line/.style={very thick},
%		information text/.style={rounded corners,fill=red!10,inner sep=1ex},
%		dot/.style={circle,inner sep=1pt,fill,label={#1},name=#1}			
%		]
%		
%		%Colors
%		\colorlet{anglecolor}{green!50!black}	%angle arcs/lines
%		
%		%The graphic
%	\end{tikzpicture}
%	\end{center}

%	\begin{figure}[htb]
%		\centering
%		\includegraphics[width=0.8\textwidth]{filename.eps}
%		\caption{Caption.}
%		\label{fig:figure}
%	\end{figure}

%		\def\enotesize{\normalsize}
%		\theendnotes
\end{document}