\documentclass[11pt]{article}
\usepackage{amsmath, amssymb, amsthm}
\usepackage[retainorgcmds]{IEEEtrantools}

\usepackage{fancyhdr}

%Listings stuff
\usepackage{listings}
\usepackage{lstautogobble}
\usepackage{color}

\definecolor{gray}{rgb}{0.5,0.5,0.5}
\lstset{
basicstyle={\small\ttfamily},
tabsize=3,
numbers=left,
numbersep=5pt,
numberstyle=\tiny\color{gray},
stepnumber=2,
breaklines=true
}

%Format stuff
\pagestyle{fancy}
\headheight 35pt

%Header info
\chead{\Large \textbf{Selection}}
\lhead{}
\rhead{}

\begin{document}
\section{Problem}
	\begin{description}
		\item[Input:] \verb|A[...]| and an integer \verb|i|$\epsilon \{1,...,n\}$
		\item[Output:] i\textsuperscript{th} order statistic of \verb|A| (i\textsuperscript{th} smallest element)
	\end{description}
	
\section{QuickSelect}
	The algorithm for randomized selection is an adaptation of the partition subroutine used in quicksort. The high-level idea is to partition a random element from the list, and if it ends up in the proper index, return it, or otherwise recurse on the appropriate half of the array.
	
	\begin{lstlisting}[autogobble=true]
		QuickSelect(A, i):
			if n==1: return A[0]
			p = rand(0, n)
			Partition(A, p)
			j = index_p	//After pivot has been moved
			
			if j==i: return pivot
			if j > i: return RSelect(left, i)
			if j < i: return RSelect(right, i-j)
	\end{lstlisting}
	
	
	
\subsection{Analysis}
	If we let the pivot element be random on every partition but assume that the element is never found until the end of the algorithm \textit{and} the algorithm always recurses on the larger half, we can get a reasonable upper bound for the average case of the algorithm. The recurrence is almost identical to the Quicksort recurrence (one less recursive call), so see the Quicksort notes for a more in-depth explanation.
	
	For the constructive induction, we guess $T(n) \leq an$.
	\begin{IEEEeqnarray}{rCl}
		T(n) & = & \frac{1}{n}\sum_{q=1}^n T(\verb|max(q-1, n-q)|) \quad + \quad n - 1\\
		& = & \frac{2}{n} \sum_{q=n/2}^{n-1} T(q) \quad + \quad n - 1\\
		& \leq & \frac{2}{n} \sum_{q=n/2}^{n-1} aq \quad + \quad n - 1\\
		& = & \frac{2a}{n}\left( \sum_{q=1}^{n-1} q - \sum_{q=1}^{\frac{n}{2} - 1} q \right) + n-1\\
		& = & \left(\frac{3a}{4} + 1\right)n - \frac{a}{2} - 1
	\end{IEEEeqnarray}
	
	For the constructive induction to hold,
	\begin{IEEEeqnarray}{rCl}
		\frac{3a}{4} + 1 & \leq & a\\
		a & \geq & 4
	\end{IEEEeqnarray}
	This pessimistic analysis yields $T(n) \leq 4n \in \Theta(n)$.
			
\section{Deterministic Selection}
	Deterministic selection is not as efficient as QuickSelect in practice, and is much trickier to implement. It is still $\Theta(n)$, and is guaranteed to be so for all cases, but in practice has much slower constants than QuickSelect.
	\begin{lstlisting}[autogobble=true]
		DSelect(A, n, i):
			C = Choosepivot(A, n)
			p = DSelect(C, n/5, n/10)	//Recursively find median
			Partition(A, p)
			if j == i: return p
			if j < i: return DSelect(left, j-1, i)
			if j > i: return DSelect(right, n-j, i-j)
			
		ChoosePivot(A, n):
			Break A into groups of size 5 each
			Sort each group
			Copy n/5 medians into new array C
			return C
	\end{lstlisting}
	
	\subsection{Analysis}
		The key insight is that when partitioning in the final step, we are guaranteed to get a 30-70 split or better. To explain this, imagine the \verb|ChoosePivot| subroutine as conceptually dividing \verb|A| into a 5-by-$n/5$ matrix. 
		
		Each column is sorted in the subroutine so that when the median-of-medians is found (median of the center row of the matrix), we know for sure that the first three elements in all sorted columns with median lower than $m$ are less than $m$, and that the last three elements in all sorted columns with median high than $m$ are greater than $m$. This gives $3n/10$ elements that we are unsure about, so we get at worst a 30-70 split.
		
		If we use a quadratic sort like Selection Sort to sort the columns, we can expect 10 comparisons per column. There are $n/5$ columns, so \verb|ChoosePivot| uses $2n$ comparisons. The last partition takes again $n - 1$ comparisons, and the recurrence is now fairly straightforward.
		\begin{equation}
			T(n) \leq T\left( \frac{7}{10}n \right) + T\left( \frac{n}{5} \right) + 3n - 1
		\end{equation}
		
		Use constructive induction to guess that $T(n) \leq an$:
		\begin{IEEEeqnarray}{rCl}
			T(n) & \leq & \frac{an}{5} + \frac{7an}{10} + 3n - 1\\
			& = & \left( \frac{9a}{10} + 3 \right) n - 1
		\end{IEEEeqnarray}
		For the induction to hold,
		\begin{IEEEeqnarray}{rCl}
			\frac{9a}{10} + 3 & \leq & a\\
			a & \geq & 30
		\end{IEEEeqnarray}
		
		So we get that the worst-case for deterministic selection by median-of-medians is $T(n) \leq 30n \in \Theta(n)$. This is a significantly worse comparison-based runtime than randomized selection, and worse even than sorting when $n \leq 2^{30}$. Practically, deterministic selection is used over randomized selection when the worst case must still be fast (QuickSelect is $O(n^2)$ in the absolute worst-case).
		
		Note that because the algorithm can terminate early, this is still preferred over an $n\lg n$ sort.
\end{document}