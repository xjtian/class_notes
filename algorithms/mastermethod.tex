\documentclass[11pt]{article}
\usepackage{amsmath, amssymb, amsthm}
\usepackage[retainorgcmds]{IEEEtrantools}

\usepackage{fancyhdr}

%Format stuff
\pagestyle{fancy}
\headheight 35pt

%Header info
\chead{\Large \textbf{Master Method}}
\lhead{}
\rhead{}

\begin{document}
\section{Definition}
	\begin{itemize}
		\item $T(n)$: upper-bound of function at level $n$
		\item \textbf{Recurrance:} expression of $T(n)$ in terms of runtime of recursive calls.
		\item \textbf{Base case:} $T(1) \leq k$
		\item $\forall n > 1: T(n) \leq ($recursive work$) + ($current work$)$
	\end{itemize}
	
	For larger, non-base case sizes of n:
	\begin{equation}
		T(n)\leq aT\left(\frac{n}{b}\right) + O(n^d)	
	\end{equation}
	
	There are $a$ recursive calls on inputs of size $n/b$, where the work done in the \emph{combination step} of the algorithm is of the magnitude $n^d$.
	
	This leads to the generic \textbf{Master Method} for divide-and-conquer algorithms:
	\begingroup
	\renewcommand*{\arraystretch}{1.5}
	\large
	\begin{equation}
		T(n) = \left\{
			\begin{array}{lr}
				O(n^d\cdot \log n) & : a = b^d\\
				O(n^d) & : a < b^d\\
				O(n^{\log_b a}) & : a > b^d
			\end{array}\right.
	\end{equation}
	\endgroup
	
	Note: In case 1, the base of the logarithm doesn't matter because the difference is a constant factor.
	
\section{Proof}
	\subsection{Exact Analysis}
		Given the following recurrence:
		\begin{IEEEeqnarray}{rCl}
			T(n) & = & aT\left(\frac{n}{b}\right) + cn^d\\
			T(1) & = & f
		\end{IEEEeqnarray}
		expanding it out a few iterations for $T(n)$ reveals the pattern.
		
		\begin{IEEEeqnarray}{rCl}
			T(n) & = & aT\left(\frac{n}{b}\right) + cn^d\\
			& = & a\left[aT\left(\frac{n}{b^2}\right) + c\left(\frac{n}{b}\right)^d\right] + cn^d\\
			& = & a\left[a\left[aT\left(\frac{n}{b^3}\right) + c\left(\frac{n}{b^2}\right)^d\right] + c\left(\frac{n}{b}\right)^d \right] + cn^d\\
			& = & a^3T\left(\frac{n}{b^3}\right) + a^2c\left(\frac{n}{b^2}\right)^d + cn^d
		\end{IEEEeqnarray}
		
		Let $k = \log_b n$ be the number of iterations in the recurrence before $T$ reaches its base case.
		
		\begin{IEEEeqnarray}{rCl}
			T(n) & = & a^k f + \sum_{i=0}^{k-1} a^i c \left(\frac{n}{b^i}\right)^d\\
			& = & a^k f + cn^d \sum_{i=0}^{k-1} \frac{a^i}{b^{i^d}}\\
			& = & a^k f + cn^d \sum_{i=0}^{k-1} \left(\frac{a}{b^d}\right)^i\\
			& = & a^k f + cn^d \left(\frac{\displaystyle \left(\frac{a}{b^d}\right)^k - 1}{\displaystyle \frac{a}{b^d} - 1}\right)\\
			& = & n^{\log_b a}f + cn^d\left(\frac{\displaystyle \frac{a^{\log_b n}}{n^d} - 1}{\displaystyle \frac{a}{b^d} - 1}\right)\\
			& = & n^{\log_b a}f + \frac{\displaystyle ca^{\log_b n} - cn^d}{\displaystyle \frac{a}{b^d} - 1}\\
			T(n) & = & \left( \frac{\displaystyle c}{\displaystyle \frac{a}{b^d} - 1} + f \right)n^{\log_b a} - \frac{\displaystyle cn^d}{\displaystyle \frac{a}{b^d} - 1}
		\end{IEEEeqnarray}
		
		Note that the geometric sum formula is not valid when $a/b^d = 1$ so the final formula doesn't work for that case, but it is a special case that makes the original summation trivial to solve. The full definition of the Master Theorem, accounting for all 3 possible cases, is as follows:
		
		\begin{equation}
			T(n) = \left\lbrace
			\begin{array}{ll}
				\left( f + \frac{\displaystyle c}{\displaystyle ab^{-d} - 1}\right) n^{\log_b a} - \frac{\displaystyle cn^d}{\displaystyle ab^{-d} - 1} = & \left\lbrace
				\begin{array}{ll}
					\Theta(n^{\log_b a}) & a > b^d\\
					\Theta(n^d) & a < b^d
				\end{array}\right.
				\\
				n^d(f + c\log_b n) = \Theta(n^d \log_b n) & a = b^d
			\end{array}\right.
		\end{equation}
		
		\subparagraph{Superposition} If the work done in each recursive call isn't perfectly modeled by $cn^d$ because there are extra terms, use the theorem on each part of the recurrence letting $f=0$, add the solutions together, and then finally add $fn^{\log_b a}$.
		
	\subsection{Inexact Analysis}
		Assume the following:
		\begin{enumerate}
			\item $T(1)\leq c$
			\item $T(n)\leq aT\left(\dfrac{n}{b}\right) + cn^d$
		\end{enumerate}
		
		On the recursion tree, at any level $j$, there are $a^j$ subproblems of $\dfrac{n}{b^j}$ size each ($j\leq log_b n$). The total work done at each level is the product of the number of subproblems and the amount of work done in each subproblem.
		\begin{equation}
			T(j)\leq a^j\cdot c\cdot\left[\frac{n}{b^j}\right]^d = cn^d\cdot\left[\frac{a}{b^d}\right]^j
		\end{equation}
		
		The total work then is the sum of the work done at each level $j$ in the recursion tree:
		\begin{equation}
			T(n)\leq cn^d\cdot \sum_{j=0}^{\log_b n}\left[\frac{a}{b^d}\right]^j
		\end{equation}
		
		The ratio of $a$ to $b^d$ determines the final result of $T(n)$. There exist three cases:
		
		\begin{enumerate}
			\item
				\[a=b^d, \frac{a}{b^d} = 1\]
				\begin{equation}
					T(n) \approx cn^d\cdot\log_b n
				\end{equation}
			\item
				\[a<b^d, \frac{a}{b^d}<1\]
				\[\sum_{j=0}^{\log_b n}\left[\frac{a}{b^n}\right]^j \leq \text{constant}\]
				\begin{equation}
					T(n) \approx cn^d
				\end{equation}
			\item
				\[a>b^d, \frac{a}{b^d} > 1\]
				\[\sum_{j=0}^{\log_b n}\left[\frac{a}{b^n}\right]^j \leq \frac{r^k+1 -1}{r-1} = r^k\left(1 + \frac{1}{r-1}\right)\]
				\begin{equation}
					T(n)=O\left(n^d\cdot\left[\frac{a}{b^n}\right]^{\log_b n}\right) = O(n^{\log_b a})
				\end{equation}
		\end{enumerate}
\end{document}