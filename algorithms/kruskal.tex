\documentclass[11pt]{article}
\usepackage{amsmath, amssymb, amsthm}
\usepackage[retainorgcmds]{IEEEtrantools}

\usepackage[pdftex]{graphicx}
\usepackage{tikz}
\usepackage{circuitikz}
\usetikzlibrary{intersections}

\usepackage{fancyhdr}

%Listings stuff
\usepackage{listings}
\usepackage{lstautogobble}
\usepackage{color}

\definecolor{gray}{rgb}{0.5,0.5,0.5}
\lstset{
basicstyle={\small\ttfamily},
tabsize=3,
numbers=left,
numbersep=5pt,
numberstyle=\tiny\color{gray},
stepnumber=2,
breaklines=true,
boxpos=t
}

%Format stuff
\pagestyle{fancy}
\headheight 35pt

%Header info
\chead{\Large \textbf{Kruskal's Algorithm}}
\lhead{}
\rhead{}

\begin{document}
\section{Algorithm}
	Basically, add edges to a subset of the viable solution by increasing weight, given that no cycles are induced by the addition. Merges disjoint forests into a single MST.
	
	\begin{lstlisting}[autogobble=true]
		def kruskalMST(G = (V, E)):
			A = {}
			Place each vertex in its own set
			Sort E by increasing order by weight
			
			for (u, v) in sorted(E):
				if find(u) != find(v):	# different trees
					A += (u, v)
					union(u, v)		# merge trees
	\end{lstlisting}

\section{Proof}
	Consider at any point in the algorithm, with the partial solution $A'$, that we are considering the edge $(u, v) \mid u \in A' \subset A, v \notin A'$ and $(u, v)$ does not induce any cycles in $A'$. Consider the cut $(A', V - A')$:
	\begin{itemize}
		\item Every crossing edge is not in $A'$ by definition of the cut
		\item $(u, v)$ must cross the cut by definition of the edge being considered. Because edges are considered by weight order, $(u, v)$ must be the cheapest such crossing edge.
	\end{itemize}
	By the safe spanning edge lemma, $(u, v)$ is a safe edge. Therefore, Kruskal's algorithm correctly generates an MST.
	
\section{Runtime}
	With an efficient union-find data structure to back the algorithm, it runs in $O(m\log n)$. The initial sorting is $O(m \log m)$. Given an efficient Union Find structure, there are $\Theta(n)$ operations on a total input size of $n$ elements to give $\Theta(m \log n)$ for the loop. Therefore, the algorithm is $\Theta(m \log n)$.

%	\begin{center}
%	\begin{tikzpicture}
%		[scale=3,line cap=round,
%		%Styles
%		axes/.style=,
%		important line/.style={very thick},
%		information text/.style={rounded corners,fill=red!10,inner sep=1ex},
%		dot/.style={circle,inner sep=1pt,fill,label={#1},name=#1}			
%		]
%		
%		%Colors
%		\colorlet{anglecolor}{green!50!black}	%angle arcs/lines
%		
%		%The graphic
%	\end{tikzpicture}
%	\end{center}

%	\begin{figure}[htb]
%		\centering
%		\includegraphics[width=0.8\textwidth]{filename.eps}
%		\caption{Caption.}
%		\label{fig:figure}
%	\end{figure}

%		\def\enotesize{\normalsize}
%		\theendnotes
\end{document}