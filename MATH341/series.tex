\documentclass[11pt]{article}
\usepackage{amsmath, amssymb, amsthm}
\usepackage[retainorgcmds]{IEEEtrantools}

\usepackage{fancyhdr}

%Format stuff
\pagestyle{fancy}
\headheight 35pt

%Header info
\chead{\Large \textbf{Series}}
\lhead{}
\rhead{}

\begin{document}
\section{Taylor Series}
	The $n$th degree \textbf{Taylor polynomial} of an arbitrary function $f(x)$ at $x=a$ is given by
	\begin{equation}
		T_n(x) = \sum_{k=0}^n \frac{1}{k!}f^{(k)}(a)(x-a)^k
	\end{equation}
	The difference $f(x) - T_n(x) = R_n(x)$ is the \textbf{Taylor remainder}. If $f$ has $n+1$ continuous derivatives on some interval containing $a$, then for all $x$ in the interval,
	\begin{equation}
		f(x) = T_n(x) + R_n(x)
	\end{equation}
	where for some $x < c < a$,
	\begin{equation}
		R_n(x) = \frac{1}{(n+1)!}f^{(n+1)}(c)(x-a)^{n+1}
	\end{equation}
	
	\subparagraph{Convergence} To test convergence of an infinite Taylor series, show that $\lim_{n\rightarrow \infty} R_n(x) = 0$. for all values of $x$ in question.
	
\section{Convergence Criteria}
	\subparagraph{Limit Test} If a series $\sum_{k=1}^\infty a_k$ converges, then $\lim_{k\rightarrow \infty} a_k = 0$. Otherwise, the series diverges. Note that the converse is not true.
	
	\subparagraph{Integral Test} If $f(k) = a_k$ for all terms of $k$, then the series and improper integral either both converge or diverge.
	\begin{equation}
		\sum_{k=1}^\infty a_k, \quad \int_1^\infty f(x)dx
	\end{equation}
	
	\subparagraph{Comparison Test} If $0 \leq a_k \leq b_k$, if $\sum_{k=1}^\infty b_k$ converges, then so does $\sum{k=1}^\infty a_k$. If $a$ diverges, then $b$ diverges.
	
	\subparagraph{Absolution Convergence} If the series $\sum_{k=1}^\infty |a_k|$ converges, then the series $\sum_{k=1}^\infty a_k$ is \textbf{absolutely convergent}. If $a$ is absolutely convergent, then $a$ is also convergent.
	
	\subparagraph{Ratio Test} If $\lim_{k\rightarrow \infty} |a_{k+1}|/|a_k|$ exists for a series, the series converges absolutely if
	\begin{equation}
		\lim_{k\rightarrow \infty} \left| \frac{a_{k+1}}{a_k} \right| < 1
	\end{equation}
	and diverges if
	\begin{equation}
		\lim_{k\rightarrow \infty} \left| \frac{a_{k+1}}{a_k} \right| > 1
	\end{equation}
	and no conclusions can be drawn if the limit is 1 or fails to exist.
	
	\subparagraph{Leibniz Test} If $\sum_{k=1}^\infty a_k$ is an alternating series such that $|a_k| \geq |a_{k+1}|$ and $\lim_{k\rightarrow \infty} a_k = 0$, then the partial sums $s_n$ converge to a sum $s$, with error $|s - s_n| \leq |a_{n+1}|$.
	
\section{Uniform Convergence}
	Let $f_k(\mathbf{x})$ be a sequence of real-valued functions defined for all $\mathbf{x}$ in some set $S$. The series \textbf{converges pointwise} to $f(\mathbf{x})$ if
	\begin{equation}
		\forall \mathbf{x} \in S, \quad \lim_{n\rightarrow \infty} \sum_{k=1}^\infty f_k(\mathbf{x}) = f(\mathbf{x})
	\end{equation}
	
	An infinite functional series that converges pointwise defines a function $f$ on $S$. The same series \textbf{converges uniformly} to a function $f$ on a set $S$, if for all points in $S$,
	\begin{equation}
		\exists \epsilon > 0, n_0 > 0 \mid \forall N > n_0, \left| \sum_{k=1}^N f_k(\mathbf{x}) - f(\mathbf{x}) \right| < \epsilon
	\end{equation}
	
	\subparagraph{Weierstrass Test} If there exists a convergent constant series $\sum_{k=1}^\infty p_k$ such that $|f_k(\mathbf{x})| \leq p_k$, then $\sum_{k=1}^\infty f_k(\mathbf{x})$ converges uniformly.
	
	\subsection{Related Theorems}
		\begin{itemize}
			\item If $\lim_{\mathbf{x} \rightarrow \mathbf{x_0}} f_k(\mathbf{x})$ exists, then
				\begin{equation}
					\lim_{\mathbf{x} \rightarrow \mathbf{x_0}} \sum_{k=1}^\infty f_k(\mathbf{x}) = \sum_{k=1}^\infty \lim_{\mathbf{x} \rightarrow \mathbf{x_0}} f_k(\mathbf{x})
				\end{equation}
			\item If $\sum f_k$ is a uniformly convergent series of continuous functions on $S \in \mathbf{R}^n$, then the function it uniformly converges to is continuous as well.
			\item If $\sum f_k$ converges uniformly on some interval $[a, b]$ and $f_k$ are all continuous on the same interval, then
			\begin{equation}
				\sum_{k=1}^\infty \int_a^b f_k(x)dx = \int_a^b \left[ \sum_{k=1}^\infty f_k(x) \right] dx
			\end{equation}
			\item If $\sum f_k$ converges pointwise in an interval to $f$ and $\sum f_k'$ converges uniformly on the same interval, then $f$ is continuously differentiable and
			\begin{equation}
				\frac{d}{dx} \sum_{k=1}^\infty f_k(x) = \sum_{k=1}^\infty \frac{df_k}{dx}(x)
			\end{equation}
		\end{itemize}
		
\section{Power Series}
	A power series is a series of the form
	\begin{equation}
		\sum_{k=0}^\infty a_k (x-a)^k
	\end{equation}
	The \textbf{radius of convergence} of the series is half the length of the set of $x$ with midpoint at $x=a$ for which the series converges.
	
	In the interior of the interval of convergence for a power series, it can be integrated or differentiated term-by-term.  Additionally, if
	\begin{equation}
		f(x) = \sum_{k=0}^\infty a_k (x-a)^k
	\end{equation}
	then the series is the Taylor series of $f$ about $x=a$: that is, a function $f$ can be represented by a power series uniquely. From this, it follows that we can compute a limit of $f(x)$ as $x$ approaches $a$ by setting $x=a$ in the series to get
	\begin{equation}
		\lim_{x\rightarrow a} f(x) = a_0
	\end{equation}
	
	Power series can be multiplied and divided like polynomials. The product of two power series about the same point gives a third series called a \textbf{Cauchy product}:
	\begin{eqnarray}
		(a_0 + a_1 x + a_2 x^2 + \ldots)(b_0 + b_1 x + b_2 x^2 + \ldots)\\
		= a_0b_0 + (a_1b_0 + a_0b_1)x + (a_0b_2 + a_1b_1 + a_2b_0)x^2 + \ldots
	\end{eqnarray}

%	\begin{center}
%	\begin{tikzpicture}
%		[scale=3,line cap=round,
%		%Styles
%		axes/.style=,
%		important line/.style={very thick},
%		information text/.style={rounded corners,fill=red!10,inner sep=1ex},
%		dot/.style={circle,inner sep=1pt,fill,label={#1},name=#1}			
%		]
%		
%		%Colors
%		\colorlet{anglecolor}{green!50!black}	%angle arcs/lines
%		
%		%The graphic
%	\end{tikzpicture}
%	\end{center}

%	\begin{figure}[htb]
%		\centering
%		\includegraphics[width=0.8\textwidth]{filename.eps}
%		\caption{Caption.}
%		\label{fig:figure}
%	\end{figure}

%		\def\enotesize{\normalsize}
%		\theendnotes
\end{document}