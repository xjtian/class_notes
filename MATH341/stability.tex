\documentclass[11pt]{article}
\usepackage{amsmath, amssymb, amsthm}
\usepackage[retainorgcmds]{IEEEtrantools}

\usepackage[pdftex]{graphicx}
\usepackage{tikz}
\usepackage{circuitikz}
\usetikzlibrary{intersections}

\usepackage{fancyhdr}

%Format stuff
\pagestyle{fancy}
\headheight 35pt

%Header info
\chead{\Large \textbf{Equilibrium and Stability}}
\lhead{}
\rhead{}

\begin{document}
	Consider the behavior of solutions $\mathbf{x} = \mathbf{x}(t)$ of autonomous systems $\dot{\mathbf{x}} = \mathbf{F}(\mathbf{x})$ near an equilibrium point $\mathbf{x_0} \mid \mathbf{F}(\mathbf{x_0}) = 0$. For the following types of behavior, in the notation $\mathbf{x}$ represents any solution to the autonomous system. $\mathbf{x_0}$ is
	\begin{description}
		\item[Asymptotically stable] if 
		\begin{equation}
			\exists d_0 > 0 \mid \forall \mathbf{x} \mid ||\mathbf{x}(0) - \mathbf{x_0}|| < d_0, \lim_{t\rightarrow \infty} \mathbf{x}(t) = \mathbf{x_0}
		\end{equation} 
		\item[Stable] if
		\begin{equation}
			\exists d_0 > 0 \mid \forall \mathbf{x}: ||\mathbf{x}(0) - \mathbf{x_0}|| = d_1 < d_0, \forall t > 0, ||\mathbf{x}(t) - \mathbf{x_0}|| \leq d_1
		\end{equation}
		\item[Unstable] if the equilibrium point is not stable.
	\end{description}
	
	\section{Linear Systems}
		For an autonomous linear system $\dot{\mathbf{x}} = A\mathbf{x} + \mathbf{b}$, an equilibrium solution is some constant $\mathbf{x_0}$ satisfying $A\mathbf{x_0} + \mathbf{b} = 0$. If $A^{-1}$ exists, then $\mathbf{x_0} = -A^{-1}\mathbf{b}$. Otherwise, either there is no equilibrium point or the is a line or plane consisting entirely of such points. 
	
	In either case, for any equilibrium solution, the general solution $\mathbf{x}(t) = \mathbf{x_h}(t) + \mathbf{x_0}$ behaves near $\mathbf{x_0}$ the same as the homogeneous solution $\mathbf{x_h}$ near $\mathbf{x} = 0$.
	
	Because the homogeneous solutions are given by $t^ke^{\lambda t}$, if there is at least one eigenvalue of $A$ with positive real part, in two dimensions, the equilibrium is necessarily unstable. If one eigenvalue is negative and one is positive, the equilibrium is a saddle point.
	
	For $A: \mathbb{R}^n \times \mathbb{R}^n$, the equilibrium solution $\mathbf{x_0} = 0$ for the homogeneous system with $A$ is asymptotically stable if every eigenvalue of $A$ has negative real part and is unstable if there is at least one eigenvalue with positive real part.
	
	In two dimensions, here are all the possibilities of behavior near an equilibrium point:
	\begin{enumerate}
		\item $0 < \lambda_1 < \lambda_2$: Unstable node
		\item $\lambda_1 < 0 < \lambda_2$: Saddle (unstable)
		\item $\lambda_1 < \lambda_2 < 0$: Asymptotically stable node
		\item $\lambda = p \pm iq, p>0, q \neq 0$: Unstable spiral
		\item $\lambda = \pm iq, q\neq 0$: Stable center
		\item $\lambda = -p \pm iq, p > 0, q \neq 0$: Asymptotically stable spiral
		\item $\lambda_1 = \lambda_2 > 0$: Unstable star
		\item $\lambda_1 = \lambda_2 < 0$: Asymptotically stable star
	\end{enumerate}
	
	
	\section{Nonlinear Systems} 
	To extend the eigenvalue analysis of equilibrium solutions to autonomous nonlinear systems, after finding the equilibrium points, introduce the \textbf{linearization} of $\dot{\mathbf{x}} = \mathbf{F}(\mathbf{x})$ using the derivative matrix:
	\begin{equation}
		\dot{\mathbf{x}} = \mathbf{F'}(\mathbf{x_0})(\mathbf{x} - \mathbf{x_0}) + \mathbf{F}(\mathbf{x_0})
	\end{equation}
	
	The equilibrium solution is asymptotically stable if every eigenvalue of the derivative matrix $\mathbf{F}(\mathbf{x_0})$ has negative real part. The point is unstable if at least one eigenvalue has positive real part, and is a saddle point if both signs occur. If both signs occur, no conclusions can be drawn.

%	\begin{center}
%	\begin{tikzpicture}
%		[scale=3,line cap=round,
%		%Styles
%		axes/.style=,
%		important line/.style={very thick},
%		information text/.style={rounded corners,fill=red!10,inner sep=1ex},
%		dot/.style={circle,inner sep=1pt,fill,label={#1},name=#1}			
%		]
%		
%		%Colors
%		\colorlet{anglecolor}{green!50!black}	%angle arcs/lines
%		
%		%The graphic
%	\end{tikzpicture}
%	\end{center}

%	\begin{figure}[htb]
%		\centering
%		\includegraphics[width=0.8\textwidth]{filename.eps}
%		\caption{Caption.}
%		\label{fig:figure}
%	\end{figure}

%		\def\enotesize{\normalsize}
%		\theendnotes
\end{document}