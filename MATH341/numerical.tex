\documentclass[11pt]{article}
\usepackage{amsmath, amssymb, amsthm}
\usepackage[retainorgcmds]{IEEEtrantools}

\usepackage[pdftex]{graphicx}
\usepackage{tikz}
\usetikzlibrary{intersections}

\usepackage{fancyhdr}

%Listings stuff
\usepackage{listings}
\usepackage{lstautogobble}
\usepackage{color}

\definecolor{gray}{rgb}{0.5,0.5,0.5}
\lstset{
basicstyle={\small\ttfamily},
tabsize=3,
numbers=left,
numbersep=5pt,
numberstyle=\tiny\color{gray},
stepnumber=2,
breaklines=true
}

%Format stuff
\pagestyle{fancy}
\headheight 35pt

%Header info
\chead{\Large \textbf{Numerical Methods}}
\lhead{}
\rhead{}

\begin{document}
Both the following methods consider the initial value problem
\begin{IEEEeqnarray}{rCl}
	y' & = & F(x, y)\\
	y(a) & = & b
\end{IEEEeqnarray}
\section{Euler's Method}
	\begin{equation}
		y(x) \approx y'(a) + (x - a) \cdot F(a, b)
	\end{equation}
	
	Applied iteratively with step size $h$,
	\begin{IEEEeqnarray}{rCl}
		y(x_k) & = & y(x_{k-1}) + hF(x_{k-1}, y(x_{k-1}))\\
		x_k & = & x_{k - 1} + h
	\end{IEEEeqnarray}
	
\section{Prediction-Correction Euler's Method}
	\begin{IEEEeqnarray}{rCl}
		y(x_k) & = & y(x_{k-1}) + h\frac{F(x_k, P_k) + F(x_{k-1}, y_{k-1})}{2}\\
		P_k & = & y(x_{k-1}) + h F(x_{k-1}, y(x_{k-1}))
	\end{IEEEeqnarray}

%	\begin{center}
%	\begin{tikzpicture}
%		[scale=3,line cap=round,
%		%Styles
%		axes/.style=,
%		important line/.style={very thick},
%		information text/.style={rounded corners,fill=red!10,inner sep=1ex},
%		dot/.style={circle,inner sep=1pt,fill,label={#1},name=#1}			
%		]
%		
%		%Colors
%		\colorlet{anglecolor}{green!50!black}	%angle arcs/lines
%		
%		%The graphic
%	\end{tikzpicture}
%	\end{center}

%	\begin{figure}[htb]
%		\centering
%		\includegraphics[width=0.8\textwidth]{filename.eps}
%		\caption{Caption.}
%		\label{fig:figure}
%	\end{figure}

%		\def\enotesize{\normalsize}
%		\theendnotes
\end{document}