\documentclass[11pt]{article}
\usepackage{amsmath, amssymb, amsthm}
\usepackage[retainorgcmds]{IEEEtrantools}

\usepackage[pdftex]{graphicx}
\usepackage{tikz}
\usetikzlibrary{intersections}

\usepackage{fancyhdr}

%Listings stuff
\usepackage{listings}
\usepackage{lstautogobble}
\usepackage{color}

\definecolor{gray}{rgb}{0.5,0.5,0.5}
\lstset{
basicstyle={\small\ttfamily},
tabsize=3,
numbers=left,
numbersep=5pt,
numberstyle=\tiny\color{gray},
stepnumber=2,
breaklines=true
}

%Format stuff
\pagestyle{fancy}
\headheight 35pt

%Header info
\chead{\Large \textbf{Numerical Methods}}
\lhead{}
\rhead{}

\begin{document}
\section{First-Order}
	Both the following methods consider the initial value problem
	\begin{IEEEeqnarray}{rCl}
		y' & = & F(x, y)\\
		y(a) & = & b
	\end{IEEEeqnarray}
	\subsection{Euler's Method}
		\begin{equation}
			y(x) \approx y'(a) + (x - a) \cdot F(a, b)
		\end{equation}
		
		Applied iteratively with step size $h$,
		\begin{IEEEeqnarray}{rCl}
			y(x_k) & = & y(x_{k-1}) + hF(x_{k-1}, y(x_{k-1}))\\
			x_k & = & x_{k - 1} + h
		\end{IEEEeqnarray}
		
	\subsection{Prediction-Correction Euler's Method}
		\begin{IEEEeqnarray}{rCl}
			y(x_k) & = & y(x_{k-1}) + h\frac{F(x_k, P_k) + F(x_{k-1}, y_{k-1})}{2}\\
			P_k & = & y(x_{k-1}) + h F(x_{k-1}, y(x_{k-1}))
	\end{IEEEeqnarray}
	
\section{Second-Order}
	Introduce an artificial second dimension for the generic second-order ODE
	\begin{equation}
		y'' = f(x, y, y')
	\end{equation}
	so it becomes
	\begin{IEEEeqnarray}{rCl}
		z & = & y'\\
		z' & = & f(x,y,z)
	\end{IEEEeqnarray}
	Both of the following methods consider the initial value problem $y_0 = y(x_0), z_0 = y'(x_0)$.
	
	\subsection{Euler's Method}
		\begin{IEEEeqnarray}{rCl}
			z_{k+1} & = & z_k + hf(x_k, y_k, z_k)\\
			y_{k+1} & = & y_k + hz_k
		\end{IEEEeqnarray}
		
	\subsection{Improved Euler's Method}
		\begin{IEEEeqnarray}{rCl}
			\tilde{z}_{k+1} & = & z_k + hf(x_k, y_k, z_k)\\
			\tilde{y}_{k+1} & = & y_k + hz_k\\
			y_{k+1} & = & y_k + \frac{h}{2}(z_k + \tilde{z}_{k+1})\\
			z_{k+1} & = & z_k + \frac{h}{2}(f(x_k,y_k,z_k) + f(x_k, \tilde{y}_{k+1}, \tilde{z}_{k+1}))
		\end{IEEEeqnarray}

%	\begin{center}
%	\begin{tikzpicture}
%		[scale=3,line cap=round,
%		%Styles
%		axes/.style=,
%		important line/.style={very thick},
%		information text/.style={rounded corners,fill=red!10,inner sep=1ex},
%		dot/.style={circle,inner sep=1pt,fill,label={#1},name=#1}			
%		]
%		
%		%Colors
%		\colorlet{anglecolor}{green!50!black}	%angle arcs/lines
%		
%		%The graphic
%	\end{tikzpicture}
%	\end{center}

%	\begin{figure}[htb]
%		\centering
%		\includegraphics[width=0.8\textwidth]{filename.eps}
%		\caption{Caption.}
%		\label{fig:figure}
%	\end{figure}

%		\def\enotesize{\normalsize}
%		\theendnotes
\end{document}