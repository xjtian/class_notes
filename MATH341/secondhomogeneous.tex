\documentclass[11pt]{article}
\usepackage{amsmath, amssymb, amsthm}
\usepackage[retainorgcmds]{IEEEtrantools}

\usepackage[pdftex]{graphicx}
\usepackage{tikz}
\usetikzlibrary{intersections}

\usepackage{fancyhdr}

%Listings stuff
\usepackage{listings}
\usepackage{lstautogobble}
\usepackage{color}

\definecolor{gray}{rgb}{0.5,0.5,0.5}
\lstset{
basicstyle={\small\ttfamily},
tabsize=3,
numbers=left,
numbersep=5pt,
numberstyle=\tiny\color{gray},
stepnumber=2,
breaklines=true,
boxpos=t
}

%Format stuff
\pagestyle{fancy}
\headheight 35pt

%Header info
\chead{\Large \textbf{Second-Order Homogeneous ODE}}
\lhead{}
\rhead{}

\begin{document}
\section{Differential Operators}
	The differential operator $D$ is a \textbf{linear operator} $D: C^1 \rightarrow C$, where $C^1$ is the set of all continuously differentiable functions and $C$ is the set of all continuous functions.
	\begin{equation}
		f' = Df
	\end{equation}
	
	Differential operators are necessary for finding the solution to a second-order homogeneous ODE.
	
	\subparagraph{Linearity} The following two equations express the linearity of $D$.
		\begin{equation}
			D(y_1 + y_2) = Dy_1 + Dy_2
		\end{equation}
		\begin{equation}
			D(cy) = cDy
		\end{equation}
		
\section{Second-Order Homogeneous ODE's}
	\subsection{Example}
		Consider the following problem.
		\begin{IEEEeqnarray}{rCl}
			y'' + 5y' + 6y & = & 0\\
			(D^2 + 5D + 6)y & = & 0\\
			(D+3)(D+2)y & = & 0
		\end{IEEEeqnarray}
		
		At this stage, the next key step is to perform a change of variables.
		\begin{IEEEeqnarray}{rCl}
			(D+2)y & = & u\\
			(D+3)u & = & 0
		\end{IEEEeqnarray}
		
		Now the equation is a first-order linear ODE, which can be solved by multiplying by an integrating factor.
		\begin{IEEEeqnarray}{rCl}
			e^{3x}Du + 3e^{3x}u & = & 0\\
			D(e^{3x}u) & = & 0\\
			e^{3x}u & = & c_1\\
			u & = & c_1e^{-3x}
		\end{IEEEeqnarray}
		
		Substituting back in for $y$ after the change in variables.
		\begin{equation}
			(D+2)y = c_1e^{-3x}
		\end{equation}
		
		This is another first-order linear ODE.
		\begin{IEEEeqnarray}{rCl}
			e^{2x}Dy + 2e^{2x}y & = & c_1e^{-x}\\
			D(e^{2x}y) & = & c_1e^{-x}\\
			e^{2x}y = -c_1e^{-x} \quad & \text{or} & \quad y = -c_1e^{-3x} + c_2e^{-2x}
		\end{IEEEeqnarray}
		
		Because the constants are arbitrary, we change the sign on the first one to get
		\begin{equation}
			y = c_1e^{-3x} + c_2e^{-2x}
		\end{equation}
		
	\subsection{Theorem}
	Consider the \textbf{characteristic roots} of the second-order homogeneous ODE with constant coefficients $y'' + ay' + by = 0$ to be $r_1$ and $r_2$.
		\begin{equation}
			(D - r_1)(D - r_2)y = 0
		\end{equation}
		
		If $r_1$ and $r_2$ are \textbf{unequal}, then the general solution to the ODE is of the form
		\begin{equation}
			y = c_1e^{r_1x} + c_2e^{r_2x}
		\end{equation}
		
		If $r_1 = r_2$, then the solution is in the form
		\begin{equation}
			y = c_1e^{r_1x} + c_2xe^{r_1x}
		\end{equation}
		
	\subsection{Complex Solutions}
		If $r_1$ and $r_2$ are found to be complex, they must also be complex conjugates, as dictated by the quadratic formula. The general form of the solution still remains
		\begin{equation}
			y = c_1e^{r_1x} + c_2e^{r_2x}
		\end{equation}
		but because the roots are complex, after expanding them out and using Euler's formula on the result, the general form for the solution becomes
		\begin{equation}
			y = c_1 e^{\alpha x}\cos \beta x + c_2 i e^{\alpha x}\sin\beta x
		\end{equation}

%	\begin{center}
%	\begin{tikzpicture}
%		[scale=3,line cap=round,
%		%Styles
%		axes/.style=,
%		important line/.style={very thick},
%		information text/.style={rounded corners,fill=red!10,inner sep=1ex},
%		dot/.style={circle,inner sep=1pt,fill,label={#1},name=#1}			
%		]
%		
%		%Colors
%		\colorlet{anglecolor}{green!50!black}	%angle arcs/lines
%		
%		%The graphic
%	\end{tikzpicture}
%	\end{center}

%	\begin{figure}[htb]
%		\centering
%		\includegraphics[width=0.8\textwidth]{filename.eps}
%		\caption{Caption.}
%		\label{fig:figure}
%	\end{figure}

%		\def\enotesize{\normalsize}
%		\theendnotes
\end{document}