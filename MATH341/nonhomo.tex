\documentclass[11pt]{article}
\usepackage{amsmath, amssymb, amsthm}
\usepackage[retainorgcmds]{IEEEtrantools}

\usepackage[pdftex]{graphicx}
\usepackage{tikz}
\usetikzlibrary{intersections}

\usepackage{fancyhdr}

%Listings stuff
\usepackage{listings}
\usepackage{lstautogobble}
\usepackage{color}

\definecolor{gray}{rgb}{0.5,0.5,0.5}
\lstset{
basicstyle={\small\ttfamily},
tabsize=3,
numbers=left,
numbersep=5pt,
numberstyle=\tiny\color{gray},
stepnumber=2,
breaklines=true,
boxpos=t
}

%Format stuff
\pagestyle{fancy}
\headheight 35pt

%Header info
\chead{\Large \textbf{Nonhomeogeneous Linear ODE}}
\lhead{}
\rhead{}

\begin{document}
\section{Superposition}
	Given a linear operator $L$ in terms of the differential operator $D$ (e.g. $L=D^2 + D + 1$), because of its linearity the solution to the nonhomogeneous equation $L(y) = f$ can be written as the sum of $L(y) = 0$ and $L(y) = f$. Thus, the solution to a nonhomogeneous ODE
	\begin{equation}
		L(y) = f
	\end{equation}
	can be written as
	\begin{equation}
		y(x) = y_h + y_p
	\end{equation}
	$y_h$ is the general solution to the homogeneous case, and $y_p$ is the particular solution to the non-homogeneous case.
	
\section{Method of Undetermined Coefficients}
	\subsection{Factored Operators}
		For a nonhomogeneous equation $L(y) = f(x)$, if there exists a linear differential operator such that $M(f(x)) = 0$, then the equation can be expressed and solved as
		\begin{equation}
			M(L(y)) = 0
		\end{equation}
		
		This will yield a solution as the sum of two parts, $y_p$ and $y_h$. Substitute $y(x)=y_p$ into the original ODE to determine the constants for the particular solution.
		
	\subsection{Particular Guess}
		It is possible to guess a general form of a particular solution given $f(x)$, then substitute to determine the coefficients.
		
		\[\]
		
		\begin{tabular}{c|c}
		$f(x)$ & Guessed $y_p$\\\hline
		$k$ & $c$\\
		$x^m$ & $c_mx^m + c_{m-1}x^{m-1} + \ldots + c_1x + c_0$\\
		$e^{\gamma x}$ & $ce^{\gamma x}$\\
		$\cos(\alpha x + \beta)$ or $\sin(\alpha x + \beta)$ & $c_1 \cos(\alpha x + \beta) + c_2\sin(\alpha x + \beta)$
		\end{tabular}
		
		By superposition, if $f(x)$ is a linear combination (addition, multiplication) of any of the above, simply combine the guessed solutions in the same way to get the particular solution and substitute back in to determine the constants.
		
		The above table only holds when \textbf{no term in the particular solution is in the homogeneous solution}. If this is not true, then multiply the the particular solution by $x$ until all terms in the solution are linearly independent again.

%	\begin{center}
%	\begin{tikzpicture}
%		[scale=3,line cap=round,
%		%Styles
%		axes/.style=,
%		important line/.style={very thick},
%		information text/.style={rounded corners,fill=red!10,inner sep=1ex},
%		dot/.style={circle,inner sep=1pt,fill,label={#1},name=#1}			
%		]
%		
%		%Colors
%		\colorlet{anglecolor}{green!50!black}	%angle arcs/lines
%		
%		%The graphic
%	\end{tikzpicture}
%	\end{center}

%	\begin{figure}[htb]
%		\centering
%		\includegraphics[width=0.8\textwidth]{filename.eps}
%		\caption{Caption.}
%		\label{fig:figure}
%	\end{figure}

%		\def\enotesize{\normalsize}
%		\theendnotes
\end{document}