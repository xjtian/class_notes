\documentclass[11pt]{article}
\usepackage{amsmath, amssymb, amsthm}
\usepackage[retainorgcmds]{IEEEtrantools}

\usepackage{fancyhdr}

%Format stuff
\pagestyle{fancy}
\headheight 35pt

%Header info
\chead{\Large \textbf{Nonhomeogeneous Linear ODE}}
\lhead{}
\rhead{}

\begin{document}
\section{Superposition}
	Given a linear operator $L$ in terms of the differential operator $D$ (e.g. $L=D^2 + D + 1$), because of its linearity the solution to the nonhomogeneous equation $L(y) = f$ can be written as the sum of $L(y) = 0$ and $L(y) = f$. Thus, the solution to a nonhomogeneous ODE
	\begin{equation}
		L(y) = f
	\end{equation}
	can be written as
	\begin{equation}
		y(x) = y_h + y_p
	\end{equation}
	$y_h$ is the general solution to the homogeneous case, and $y_p$ is the particular solution to the non-homogeneous case.
	
\section{Method of Undetermined Coefficients}
	\subsection{Factored Operators}
		For a nonhomogeneous equation $L(y) = f(x)$, if there exists a linear differential operator such that $M(f(x)) = 0$, then the equation can be expressed and solved as
		\begin{equation}
			M(L(y)) = 0
		\end{equation}
		
		This will yield a solution as the sum of two parts, $y_p$ and $y_h$. Substitute $y(x)=y_p$ into the original ODE to determine the constants for the particular solution.
		
	\subsection{Particular Guess}
		It is possible to guess a general form of a particular solution given $f(x)$, then substitute to determine the coefficients.
		\begin{center}
		\begin{tabular}{c|c}
		$f(x)$ & Guessed $y_p$\\\hline
		$k$ & $c$\\
		$x^m$ & $c_mx^m + c_{m-1}x^{m-1} + \ldots + c_1x + c_0$\\
		$e^{\gamma x}$ & $ce^{\gamma x}$\\
		$\cos(\alpha x + \beta)$ or $\sin(\alpha x + \beta)$ & $c_1 \cos(\alpha x + \beta) + c_2\sin(\alpha x + \beta)$
		\end{tabular}
		\end{center}
		
		By superposition, if $f(x)$ is a combination (addition, multiplication) of any of the above, simply combine the guessed solutions in the same way to get the particular solution and substitute back in to determine the constants.
		
		The above table only holds when \textbf{no term in the particular solution is in the homogeneous solution}. If this is not true, then multiply the the particular solution by $x$ until all terms in the solution are linearly independent again.
		
\section{Variation of Parameters}
	While the method of undertermined coefficients works for general ODE's with constant coefficients, it doesn't work for ODE's with functions as coefficients.
	
	In the constant coefficients case, knowing a nontrivial solution $y_1$ of the associated homogeneous ODE means that the solution to the nonhomogeneous equation can be expressed $y(x) = y_1(x)u(x)$, where $u$ is an arbitrary function to solve for. Substitution will leave a linear differential equation to solve for $u(x)$.
	
	In the general case when there are functions as coefficents, first consider the \textbf{normalized equation}
	\begin{equation}
		y'' + a(x)y' + b(x)y = f(x)
	\end{equation}
	If $y_1$ and $y_2$ are homoegenous solutions, then form the linear combination
	\begin{equation}
		y_p(x) = y_1(x)u_1(x) + y_2(x)u_2(x)
	\end{equation}
	Substitution yields the following two functions
	\begin{equation}
		y_1u'_1 + y_2u'_2 = 0
	\end{equation}
	\begin{equation}
		y'_1u'_1 + y'_2u'_2 = f
	\end{equation}
	Now define the \textbf{Wronskian determinant} of $y_1, y_2$ as
	\begin{equation}
	w(x) = 
	\begin{vmatrix}
		y_1(x) & y_2(x)\\
		y'_1(x) & y'_2(x)
	\end{vmatrix}
	\end{equation}
	With this, the particular solution is
	\begin{equation}
		y_p(x) = y_1(x) \int\frac{-y_2(x) f(x)}{w(x)}dx + y_2(x)\int\frac{y_1(x) f(x)}{w(x)}dx
	\end{equation}
	
\section{Green's Functions}
	A slight modification of the variation of parameters method yields a method to determine $y_p$ for an initial value problem. First, convert the integrals to definite integrals.
	\begin{equation}
		y_p(x) = y_1(x) \int_{x_0}^x\frac{-y_2(t) f(t)}{w(t)}dt + y_2(x)\int_{x_0}^x\frac{y_1(t) f(t)}{w(t)}dt
	\end{equation}
	\begin{equation}
		y_p(x) = \int_{x_0}^x \frac{y_1(t) y_2(x) - y_2(t) y_1(x)}{w(t)}f(t)dt
	\end{equation}
	Next, define \textbf{Green's function} to be
	\begin{equation}
		G(x, t) = \frac{y_1(t) y_2(x) - y_2(t) y_1(x)}{w(t)}
	\end{equation}
	which generates the solution
	\begin{equation}
		y_p(x) = \int_{x_0}^x G(x,t)f(t)dt
	\end{equation}
	for $y'' + a(x)y' + b(x)y = f(x)$. By construction, $y_p(x_0) = y_p'(x_0) = 0$.
	
	For second-order ODE's with constant coefficients, the homogeneous solutions reduce to three types. Because Green's functions are constructed from the solutions to these equations, $G(x, t)$ also has three types:
	\begin{enumerate}
		\item
			\[G(x,t) = \frac{1}{r_1 - r_2}(e^{r_1(x-t)} - e^{r_2(x-t)}), \quad r_1 \neq r_2\]
		\item
			\[G(x,t) = (x-t)e^{r(x-t)}, \quad r_1 = r_2\]
		\item
			\[G(x,t) = \frac{1}{\beta}e^{\alpha(x-t)} \sin\beta(x-t), \quad r = \alpha \pm i\beta\]
	\end{enumerate}
	

%	\begin{center}
%	\begin{tikzpicture}
%		[scale=3,line cap=round,
%		%Styles
%		axes/.style=,
%		important line/.style={very thick},
%		information text/.style={rounded corners,fill=red!10,inner sep=1ex},
%		dot/.style={circle,inner sep=1pt,fill,label={#1},name=#1}			
%		]
%		
%		%Colors
%		\colorlet{anglecolor}{green!50!black}	%angle arcs/lines
%		
%		%The graphic
%	\end{tikzpicture}
%	\end{center}

%	\begin{figure}[htb]
%		\centering
%		\includegraphics[width=0.8\textwidth]{filename.eps}
%		\caption{Caption.}
%		\label{fig:figure}
%	\end{figure}

%		\def\enotesize{\normalsize}
%		\theendnotes
\end{document}