\documentclass[11pt]{article}
\usepackage{amsmath, amssymb, amsthm}
\usepackage[retainorgcmds]{IEEEtrantools}

\usepackage{fancyhdr}

%Format stuff
\pagestyle{fancy}
\headheight 35pt

%Header info
\chead{\Large \textbf{Eigenvalues and Eigenvectors}}
\lhead{}
\rhead{}

\begin{document}
\section{Finding Eigenvalues and Eigenvectors}
	For a matrix $A: \mathbb{R}^n \times \mathbb{R}^n$, there may exist up to $n$ different eigenvalues $\lambda$ and an infinite number of eigenvectors $\mathbf{v} \in \mathbb{R}^n$ such that
	\begin{equation}
		A\mathbf{v} = \lambda \mathbf{v}
	\end{equation}
	
	For each eigenvalue $\lambda_k$, there is an associated subspace of eigenvectors:
	\begin{equation}
		T_{\lambda_k} = \{\mathbf{v} \in \mathbb{R}^n \mid A\mathbf{v} = \lambda_k \mathbf{v} \}
	\end{equation}
	If the sum of the dimensions of all of these subspaces is equal to the dimension of $A$, then $A$ can be \textbf{diagonalized}, meaning there exists an invertible matrix $U = \{\mathbf{v_1}, \mathbf{v_1}, \ldots, \mathbf{v_n}\}$ such that
	\begin{equation}
		U^{-1}AU = \Lambda
	\end{equation}
	where $\Lambda$ is a diagonal matrix with the eigenvalues corresponding to each eigenvector in $U$ along the diagonal.
	
	To find the eigenvalues of $A$:
	\begin{IEEEeqnarray}{rCl}
		A\mathbf{x} & = & \lambda \mathbf{x}\\
		A\mathbf{x} & = & \lambda I \mathbf{x}\\
		(A - \lambda I)\mathbf{x} & = & 0
	\end{IEEEeqnarray}
	This will yield a polynomial in $\lambda$. Solve for all possible values and substitute each eigenvalue back into the original equation one-by-one to determine the corresponding basis of eigenvectors.
	
\section{Solving Systems}
	Given a system
	\begin{equation}
		\frac{d\mathbf{x}}{dt} = A\mathbf{x}
	\end{equation}
	we can extrapolate from the 1-dimensional case that the general solution will be a linear combination of particular solutions. Note that these matrix method is only valid if $\sum \text{dim}T_\lambda = n$.
	
	If the algebraic multiplicity of all eigenvalues is 1, then
	\begin{equation}
		\mathbf{x} = \sum_{k=1}^n c_ke^{\lambda_k t} \mathbf{v_k}
	\end{equation}
	
	However, if there are eigenvalues with multiplicities greater than 1, then the general solution is
	\begin{equation}
		\mathbf{x} = \sum_{k=1}^n c_ke^{\lambda_i t} \mathbf{v_k}
	\end{equation}
	where $\lambda_i$ is the eigenvalue that corresponds to $\mathbf{v_k}$.

%	\begin{center}
%	\begin{tikzpicture}
%		[scale=3,line cap=round,
%		%Styles
%		axes/.style=,
%		important line/.style={very thick},
%		information text/.style={rounded corners,fill=red!10,inner sep=1ex},
%		dot/.style={circle,inner sep=1pt,fill,label={#1},name=#1}			
%		]
%		
%		%Colors
%		\colorlet{anglecolor}{green!50!black}	%angle arcs/lines
%		
%		%The graphic
%	\end{tikzpicture}
%	\end{center}

%	\begin{figure}[htb]
%		\centering
%		\includegraphics[width=0.8\textwidth]{filename.eps}
%		\caption{Caption.}
%		\label{fig:figure}
%	\end{figure}

%		\def\enotesize{\normalsize}
%		\theendnotes
\end{document}