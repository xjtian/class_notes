\documentclass[11pt]{article}
\usepackage{amsmath, amssymb, amsthm}
\usepackage[retainorgcmds]{IEEEtrantools}

\usepackage{fancyhdr}

%Format stuff
\pagestyle{fancy}
\headheight 35pt

%Header info
\chead{\Large \textbf{Nonhonomgeneous Systems}}
\lhead{}
\rhead{}

\begin{document}
For a nonhonogeneous system in the form
\begin{equation}
	\frac{d\mathbf{x}}{dt} = A\mathbf{x} + \mathbf{b}(t)
\end{equation}
solutions are in the form
\begin{equation}
	\mathbf{x}(t) = e^{tA} \int e^{-tA} \mathbf{b}(t) dt + e^{tA}\mathbf{x_0}
\end{equation}
If initial conditions are based on an arbitrary $t_0$ instead of $t = 0$, then
\begin{equation}
	\mathbf{x}(t) = e^{tA} \int_{t_0}^t e^{-tA} \mathbf{b}(t) dt + e^{tA}\mathbf{x_0}
\end{equation}

\subparagraph{Variation of Parameters} For non-constant coefficients in $A$, if $n$ linearly independent solutions to the homogeneous system are already known, then form the \textbf{fundamental matrix} for the system
\begin{equation}
	X(t) = \{\mathbf{x_1}, \mathbf{x_2}, \ldots, \mathbf{x_n}\}
\end{equation}
and the solutions for the system are found
\begin{IEEEeqnarray}{rCl}
	\mathbf{x}(t) & = & \mathbf{x_p} + X(t) \cdot \mathbf{x_0}\\
	\mathbf{x_p} & = & X(t) \cdot \mathbf{v}(t)\\
	\mathbf{v}(t) & = & \int X^{-1}\mathbf{b}(t)dt
\end{IEEEeqnarray}

%	\begin{center}
%	\begin{tikzpicture}
%		[scale=3,line cap=round,
%		%Styles
%		axes/.style=,
%		important line/.style={very thick},
%		information text/.style={rounded corners,fill=red!10,inner sep=1ex},
%		dot/.style={circle,inner sep=1pt,fill,label={#1},name=#1}			
%		]
%		
%		%Colors
%		\colorlet{anglecolor}{green!50!black}	%angle arcs/lines
%		
%		%The graphic
%	\end{tikzpicture}
%	\end{center}

%	\begin{figure}[htb]
%		\centering
%		\includegraphics[width=0.8\textwidth]{filename.eps}
%		\caption{Caption.}
%		\label{fig:figure}
%	\end{figure}

%		\def\enotesize{\normalsize}
%		\theendnotes
\end{document}