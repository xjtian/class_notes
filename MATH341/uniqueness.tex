\documentclass[11pt]{article}
\usepackage{amsmath, amssymb, amsthm}
\usepackage[retainorgcmds]{IEEEtrantools}

\usepackage[pdftex]{graphicx}
\usepackage{tikz}
\usetikzlibrary{intersections}

\usepackage{fancyhdr}

%Listings stuff
\usepackage{listings}
\usepackage{lstautogobble}
\usepackage{color}

\definecolor{gray}{rgb}{0.5,0.5,0.5}
\lstset{
basicstyle={\small\ttfamily},
tabsize=3,
numbers=left,
numbersep=5pt,
numberstyle=\tiny\color{gray},
stepnumber=2,
breaklines=true
}

%Format stuff
\pagestyle{fancy}
\headheight 35pt

%Header info
\chead{\Large \textbf{Existence and Uniqueness}}
\lhead{}
\rhead{}

\begin{document}
Both the following theorems are applied on the IVP
\begin{IEEEeqnarray}{rCl}
	y' & = & F(x, y)\\
	y(x_0) & = & y_0
\end{IEEEeqnarray}

\section{Classical Picard-Lindelof Theorem}
	If on a rectangle described by $x\in[x_0 - A, x_0 + A], y\in[y_0 - B, y_0 + B]$ the following two properties hold:
	\begin{enumerate}
		\item $F(x, y)$ is continuous in $x$
		\item $F(x, y)$ is Lipschitz-continuous in $y$
	\end{enumerate}
	Then there exists $\epsilon > 0$ such that a unique solution exists on the interval $[x_0 - \epsilon, x_0 + \epsilon]$.
	
	\subparagraph{Lipschitz Continuity}
		Essentially, the slope between any two points on the graph is bounded by some real number. $f(x)$ is Lipscitz-continuous if
			\begin{equation}
				\exists k > 0 \mid \forall (a, b), \quad |f(a) - f(b)| \leq k|a - b|
			\end{equation}
	
	For this theorem, the requirement of Lipschitz-continuity can also be replaced by the requirement that $|F_y| \leq k$, that is the partial with respect to $y$ is bounded.	
	
\section{Standard Uniqueness Theorem}
	If $F$ and $F_y$ are both continuous on a rectangle $x\in[x_0 - A, x_0 + A], y\in[y_0 - B, y_0 + B]$, then there exists a unique solution somewhere in the interval. If additionally $F_y$ is bounded, then the solution exists on the entire interval.

%	\begin{center}
%	\begin{tikzpicture}
%		[scale=3,line cap=round,
%		%Styles
%		axes/.style=,
%		important line/.style={very thick},
%		information text/.style={rounded corners,fill=red!10,inner sep=1ex},
%		dot/.style={circle,inner sep=1pt,fill,label={#1},name=#1}			
%		]
%		
%		%Colors
%		\colorlet{anglecolor}{green!50!black}	%angle arcs/lines
%		
%		%The graphic
%	\end{tikzpicture}
%	\end{center}

%	\begin{figure}[htb]
%		\centering
%		\includegraphics[width=0.8\textwidth]{filename.eps}
%		\caption{Caption.}
%		\label{fig:figure}
%	\end{figure}

%		\def\enotesize{\normalsize}
%		\theendnotes
\end{document}