\documentclass[11pt]{article}
\usepackage{amsmath, amssymb, amsthm}
\usepackage[retainorgcmds]{IEEEtrantools}

\usepackage[pdftex]{graphicx}
\usepackage{tikz}
\usetikzlibrary{intersections}

\usepackage{fancyhdr}

%Listings stuff
\usepackage{listings}
\usepackage{lstautogobble}
\usepackage{color}

\definecolor{gray}{rgb}{0.5,0.5,0.5}
\lstset{
basicstyle={\small\ttfamily},
tabsize=3,
numbers=left,
numbersep=5pt,
numberstyle=\tiny\color{gray},
stepnumber=2,
breaklines=true
}

%Format stuff
\pagestyle{fancy}
\headheight 35pt

%Header info
\chead{\Large \textbf{Linear First-Order ODE's}}
\lhead{}
\rhead{}

\begin{document}
Linear ODE's are in the general form
\begin{equation}
	y' + g(x)y = f(x)
\end{equation}

To solve these equations, we multiply the entire equation by an integrating factor $M(x)$ to reduce the left hand side of the equation to the product rule for differentiation.
\begin{IEEEeqnarray}{rCl}
	M(x)y' + M(x)g(x)y & = & M(x)y' + M'(x)y\\
	M'(x) & = & M(x)g(x)\\
	\frac{dM}{dx} & = & M\cdot g(x)\\
	M^{-1}dM & = & g(x)\\
	M & = & ce^{\int g(x)dx} = e^{\int g(x)dx}
\end{IEEEeqnarray}
Drop the $c$ in the end because it ends up canceling/combining with the constant in the later steps. Now to solve the ODE,
\begin{IEEEeqnarray}{rCl}
	(M(x)y(x))' & = & f(x)M(x)\\
	M(x)y(x) & = & F(x) + c\\
	y(x) & = & \frac{F(x) + c}{M(x)}\\
	F(x) & = & \int M(x)f(x)dx\\
	F(x) & = & \int e^{\int g(x)dx} \cdot f(x)dx\\
	y(x) & = & M(x)^{-1} (F(x) + c)
\end{IEEEeqnarray}
Thus, 
\begin{equation}
	y(x) = e^{-\int g(x)dx} (F(x) + c)
\end{equation}

%	\begin{center}
%	\begin{tikzpicture}
%		[scale=3,line cap=round,
%		%Styles
%		axes/.style=,
%		important line/.style={very thick},
%		information text/.style={rounded corners,fill=red!10,inner sep=1ex},
%		dot/.style={circle,inner sep=1pt,fill,label={#1},name=#1}			
%		]
%		
%		%Colors
%		\colorlet{anglecolor}{green!50!black}	%angle arcs/lines
%		
%		%The graphic
%	\end{tikzpicture}
%	\end{center}

%	\begin{figure}[htb]
%		\centering
%		\includegraphics[width=0.8\textwidth]{filename.eps}
%		\caption{Caption.}
%		\label{fig:figure}
%	\end{figure}

%		\def\enotesize{\normalsize}
%		\theendnotes
\end{document}