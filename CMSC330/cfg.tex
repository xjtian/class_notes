\documentclass[11pt]{article}
\usepackage{amsmath, amssymb, amsthm}
\usepackage[retainorgcmds]{IEEEtrantools}

\usepackage{tikz}
\usetikzlibrary{intersections}

\usepackage{fancyhdr}

%Format stuff
\pagestyle{fancy}
\headheight 35pt

%Header info
\chead{\Large \textbf{Context Free Grammars}}
\lhead{}
\rhead{}

\begin{document}
\section{Definition}
	Formally, a CFG is a 4-tuple $(\Sigma, N, P, S)$:
	\begin{itemize}
		\item $\Sigma$ - alphabet of \textbf{terminals}
		\item $N$ - a finite, nonempty set of \textbf{nonterminals} such that $N \cap \Sigma = \emptyset$.
		\item $P$ - a set of \textbf{productions} of the form $N \rightarrow (\Sigma|N)*$. Informally, rules that map a nonterminal to a set of strings of zero or more terminals.
		\item $S\in N$ - the \textbf{start symbol}
	\end{itemize}
	
	\subparagraph{Backus-Naur Form} Also called BNF, a production $S \rightarrow B c D$ is written in BNF as \verb|<A> ::= <B>c<D>|. Nonterminals are enclosed in angle brackets.
	
	CFG's can generate strings by recursive rewriting. Consider the example $S \rightarrow 0S | 1S | \epsilon$, which constructs the language \verb:(0|1)*:. Informally, a string $s$ is accepted by a CFG $S$ (notated as $s \in [[S]]$) if there exists some rewriting (a series of applications of productions) from $S$ that yields $s$. Such a sequence is called a \textbf{derivation} or \textbf{parse}, and discovering the derivation is called \textbf{parsing}.
	
\section{Parsing}
	\subsection{Parse Trees} 
		Parse trees are derivations of strings from grammars. The root node in the tree is the start symbol, every internal node is a nonterminal, and leaf nodes are read left to right to give the string corresponding to the tree. Parse trees show the structure of an expression as it corresponds to the grammar.
	
		For example, the parse tree for the derivation $S\implies aS \implies aT \implies aU \implies acU \implies ac$ for the grammar.
		\begin{IEEEeqnarray}{rCl}
			S & \rightarrow & aS | T\\
			T & \rightarrow & bT | U\\
			U & \rightarrow & cU | \epsilon
		\end{IEEEeqnarray}
		
		\begin{center}
		\begin{tikzpicture}
			[scale=.75,line cap=round,
			%Styles
			axes/.style=,
			important line/.style={very thick},
			information text/.style={rounded corners,fill=red!10,inner sep=1ex},
			dot/.style={circle,inner sep=1pt,fill,label={#1},name=#1}			
			]
			
			%Colors
			\colorlet{anglecolor}{green!50!black}	%angle arcs/lines
			
			%The graphic
			\node {S}
				child {node {a}}
				child {node {S}
					child {node {T}
						child {node {U}
							child {node {c}}
							child {node {U}
								child {node {$\epsilon$}}
							}
						}
					}
				};
		\end{tikzpicture}
		\end{center}
	
	\subsection{Ambiguity}
		A grammar is \textbf{ambiguous} if a string may have multiple derivations (multiple parse trees). Ambiguity leads to ambiguous associativity and precedence of operators. One way to eliminate ambiguity is to turn a grammar into a \textbf{left-recursive} or \textbf{right-recursive} grammar. Left-recursive grammars recurse on the left operand of a production, and right-recursive grammars recurse on the right operand.
		
		For example, $E \rightarrow E + a | E - a | E * a | \epsilon$ is left-recursive and $E \rightarrow a + E | \epsilon$ is right-recursive. Left-recursive grammars correspond to left-associative operators, and right-recursive grammars correspond to right-associative operators.
		
		\subparagraph{Precedence} If a nonterminal has productions for several operators, they effectively have the same precedence. To order operators by precedence, think of productions for nonterminals as a linked list where the head of the list (start symbol) has the lowest precedence and the end of the list has the highest precedence.
%	\begin{figure}[htb]
%		\centering
%		\includegraphics[width=0.8\textwidth]{filename.eps}
%		\caption{Caption.}
%		\label{fig:figure}
%	\end{figure}

%		\def\enotesize{\normalsize}
%		\theendnotes
\end{document}