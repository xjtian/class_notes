\documentclass[11pt]{article}
\usepackage{amsmath, amssymb, amsthm}
\usepackage[retainorgcmds]{IEEEtrantools}

\usepackage[pdftex]{graphicx}
\usepackage{tikz}
\usetikzlibrary{intersections}

\usepackage{marginnote}
\usepackage{endnotes}

\usepackage{fancyhdr}

%Listings stuff
\usepackage{listings}
\usepackage{lstautogobble}
\usepackage{color}

\definecolor{gray}{rgb}{0.5,0.5,0.5}
\lstset{
basicstyle={\small\ttfamily},
tabsize=3,
numbers=left,
numbersep=5pt,
numberstyle=\tiny\color{gray},
stepnumber=2,
breaklines=true
}

%Properly formatted differential 'd'
\newcommand{\ud}{\, \mathrm{d}}

%Format stuff
\pagestyle{fancy}
\headheight 35pt

%Header info
\chead{\Large \textbf{Maximum Subsequence Sum}}
\lhead{}
\rhead{}

\begin{document}
\section{Problem}
	\begin{description}
		\item[Input:] An array \verb|A[1...n]| of integers. Negative values are allowed.
		\item[Output:] The maximum sum among all subsequences in the array, or 0 if all values are negative.
	\end{description}
	
\section{Algorithm}
	\subsection{Naive Implementation}
		A naive implementation is to sum every possible subsequence and keep a running maximum. This approach is $O(n^3)$.
		\begin{lstlisting}[autogobble=true]
			for i in [1, n]:
				for j in [i, n]:
					for k in [i, j]:
						sum += A[k]
					max = max(max, sum)
			return max
		\end{lstlisting}
		
		\subsubsection{Runtime}
			The runtime is upper-bounded by $O(n^3)$, but it is possible to derive the constants. Note the changes of variables to simplify the sums.
			\begin{IEEEeqnarray}{rCl}
				\sum_{i=1}^n \sum_{j=i}^n \sum_{k=i}^j 1 & = & \sum_{i=1}^n \sum_{j=i}^n (j - i + 1)\\
				& = & \sum_{i=1}^n \sum_{j=1}^{n-i+1} j\\
				& = & \frac{1}{2} \sum_{i=1}^n [(n - i + 1)(n - i + 2)]\\
				& = & \frac{1}{2} \sum_{i=1}^n i(i + 1)\\
				& = & \frac{1}{2} n(n+1)(n+2)
			\end{IEEEeqnarray}
			
	\subsection{Improved Naive Implementation}
		We can eliminate the innermost loop by keeping a running sum within the j-loop. This sum represents the sum of the i-j subarray. This approach runs in $O(n^2)$.
		\begin{lstlisting}[autogobble=true]
			for i in [1, n]:
				for j in [i, n]:
					sum += A[j]
					max = max(max, sum)
			return max
		\end{lstlisting}
	

%	\begin{center}
%	\begin{tikzpicture}
%		[scale=3,line cap=round,
%		%Styles
%		axes/.style=,
%		important line/.style={very thick},
%		information text/.style={rounded corners,fill=red!10,inner sep=1ex},
%		dot/.style={circle,inner sep=1pt,fill,label={#1},name=#1}			
%		]
%		
%		%Colors
%		\colorlet{anglecolor}{green!50!black}	%angle arcs/lines
%		
%		%The graphic
%	\end{tikzpicture}
%	\end{center}

%	\begin{figure}[htb]
%		\centering
%		\includegraphics[width=0.8\textwidth]{filename.eps}
%		\caption{Caption.}
%		\label{fig:figure}
%	\end{figure}

%		\def\enotesize{\normalsize}
%		\theendnotes
\end{document}